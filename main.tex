\documentclass[11pt,a4paper,oneside]{book}

% Math typesetting
\usepackage{latexsym}
\usepackage{amsmath} % loads amsbsy, amsopn, amstext
\usepackage{amsthm}
\usepackage{thmtools, thm-restate}
\usepackage{amsfonts}
\usepackage{amssymb}
\usepackage{mathrsfs} % script (\mathscr)
\usepackage{mathtools} % overbracket
\usepackage{fouridx}
\usepackage{algpseudocode}
\usepackage{algorithm}

\input{res/bordermatrix.tex}

\theoremstyle{plain}
\newtheorem{lemm}{Lemma}
\newtheorem{theo}{Theorem}

\theoremstyle{definition}
\newtheorem{defi}{Definition}

\theoremstyle{remark}

% Encoding and Fonts
\usepackage{xunicode} % replaces fontenc
\usepackage{xltxtra} % loads fontspec, metalogo, realscripts; redefine \showhyphens; define \vfrac and \namedglyph.

\DeclareSymbolFont{sfoperators}{OT1}{cmss}{m}{n}
\SetSymbolFont{sfoperators}{normal}{OT1}{cmss}{m}{n}
\makeatletter
\renewcommand{\operator@font}{\mathgroup\symsfoperators}
\makeatother

\setromanfont{CMU Serif}
\setsansfont{CMU Sans Serif}
\setmonofont{CMU Typewriter Text}
\setmathrm{CMU Serif}
\setmathsf{CMU Sans Serif}
\setmathtt{CMU Typewriter Text}

% Pdf pages numbers and links
\usepackage[
  unicode,
  breaklinks,
  hidelinks,
  pdftitle={The Coverability problem for parametric Petri nets},
  pdfauthor={Alexis Reynouard},
  pdfsubject={Formal verification},
  xetex
]{hyperref}
% Edit \autoref texts
\usepackage{cleveref}
\crefname{theo}{theorem}{theorems}
\crefname{lemm}{lemma}{lemmas}
\crefname{defi}{definition}{definitions}
\newcommand*{\fullref}[1]{\hyperref[{#1}]{\autoref*{#1}~\nameref*{#1}}}
\renewcommand\subsectionautorefname{section}
\renewcommand\subsubsectionautorefname{section}
\renewcommand\theoremautorefname{theorem} % TODO

% Typesetting
\usepackage{xspace}
\renewcommand{\textomega}{ω\xspace}

% Layout
\usepackage[hmargin={1.25in,1.25in},vmargin={1.25in,1.25in}]{geometry}
\setlength{\parindent}{0pt}
\setlength{\parskip}{1.5ex}

\usepackage{fancyhdr}
\pagestyle{myheadings}
\fancyhf{}
\rhead[\leftmark]{thepage}

% Bibliography
\usepackage{natbib}
\makeindex

% Index
\usepackage{makeidx}

% Acronyms
\usepackage{acronym}
\acrodef{PN}{Petri net}
\acrodef{PPN}{parametric Petri net}
\acrodef{EEC}{Expand, Enlarge and Check}
\acrodef{WSTS}{well-structured transition system}

% Lists
%\usepackage{enumitem}
%\setlist[itemize]{noitemsep,nolistsep}

% Figures
\usepackage[format=hang]{subfig}

% PGF/TikZ
\usepackage{tikz}
\usetikzlibrary{arrows,decorations,backgrounds,positioning,fit,petri,decorations.pathmorphing,decorations.pathreplacing}
\tikzset{
	x=0.16\linewidth,
	y=0.16\linewidth,
	>=stealth',
	bend angle=30,
	every place/.style={
		draw=black,
		minimum size=0.055\linewidth
	},
	every transition/.style={
		fill=black,
		minimum height=0.055\linewidth,
		inner xsep=0pt,
		minimum width=3pt
	},
  horizontal/.style={
		minimum height=3pt,
		minimum width=0.055\linewidth
  }
}

%%%% Document specific
\title{The Coverability problem for parametric Petri nets}
\author{Alexis Reynouard}
\date{}

% new commands
\usepackage{xparse}
\usepackage{ifthen}

\newcommand{\lang}[1]{{\em{}#1}}

\newcommand{\Ecov}{$\mathscr{E}$-cov\xspace}
\newcommand{\Ucov}{$\mathscr{U}$-cov\xspace}
\newcommand{\oplace}{$\omega$-place\xspace}
\newcommand{\oplaces}{$\omega$-places\xspace}
\newcommand{\noplaces}{non-$\omega$-places\xspace}
\newcommand{\tand}{\text{ and }}
\newcommand{\tif}{\text{ if }}
\newcommand{\tothrw}{\text{ otherwise }}
\newcommand{\tandif}{\text{ and if }}
\newcommand{\tothrwand}{\text{ otherwise and }}

% conventions
% fixed: transition = t
% fixed: place = p
% fixed: node = n
% fixed: root = n_0
% fixed: set = {}
% fixed: vector = ()
% fixed: sequence = ()
% command name prefixes: name, def, body

% PNs
\newcommand{\opn}{$\omega$-\ac{PN}\xspace}
\newcommand{\opns}{$\omega$-\acp{PN}\xspace}
\newcommand{\omark}{$\omega$-marking\xspace}
\newcommand{\omarks}{$\omega$-markings\xspace}
% names
\newcommand{\namePN}[1][] {\ensuremath{\mathcal{N}_{#1}}\xspace}
\newcommand{\namePPN}[1][]{\ensuremath{\mathcal{S}_{{#1}}}\xspace}
% bodies
\newcommand{\bodyPN}[1][] {\tuple{\places_{#1}, \transitions_{#1}, \mar_{0
  \ifthenelse{\equal{\detokenize{#1}}{}}{}{,#1}}}}
\newcommand{\bodyPPN}[1][]{\tuple{\places_{#1}, \transitions_{#1}, \parameters_{#1}, \mar_{0
  \ifthenelse{\equal{\detokenize{#1}}{}}{}{,#1}}}}
\newcommand{\bodyNonMarkedPN}[1][] {\tuple{\places_{#1}, \transitions_{#1}}}
\newcommand{\bodyNonMarkedPPN}[1][]{\tuple{\places_{#1}, \transitions_{#1}, \parameters_{#1}}}
% defs
\newcommand{\defPN}[1][] {\namePN[#1]  = \bodyPN[#1]}
\newcommand{\defPPN}[1][]{\namePPN[#1] = \bodyPPN[#1]}
\newcommand{\defNonMarkedPN}[1][] {\namePN[#1]  = \bodyNonMarkedPN[#1]}
\newcommand{\defNonMarkedPPN}[1][]{\namePPN[#1] = \bodyNonMarkedPPN[#1]}

% Karp and Miller Tree
\newcommand{\nameT}{\ensuremath{\mathcal{T}}\xspace}
\newcommand{\bodyT}{\tuple{\nodes, \edges, n_0, \Lab}}
\newcommand{\defT}{\nameT = \bodyT}
% sets
\newcommand{\nodes}{N}
\newcommand{\edges}{B}
% functions
\newcommand{\lab}[1]{\Lab(#1)}
\newcommand{\labt}[1]{\Labt\left({#1}\right)}
\newcommand{\parent}[1]{#1^{-1}}
\newcommand{\child}[1]{#1^{+1}}
\newcommand{\tts}[2]{\Tts(#1,#2)} % tree to sequence
\newcommand{\na}[1]{\Na(#1)} % newly accelerated
\newcommand{\ab}[1]{\Ab(#1)} % accelerated before

% object
\newcommand{\tuple}[1]{\left\langle#1\right\rangle}
\newcommand{\setComp}[2]{\left\{#1 \mid #2\right\}}

\NewDocumentCommand{\inw}{ O{} O{} }{I_{#1}% %input weight
  \ifthenelse{\equal{\detokenize{#2}}{\detokenize{}}}{}
  {(#2)}}

\NewDocumentCommand{\outw}{ O{} O{} }{O_{#1}% %output weight
  \ifthenelse{\equal{\detokenize{#2}}{\detokenize{}}}{}
  {(#2)}}

\NewDocumentCommand{\effect}{ m O{} }{\Effect\left(#1\right)%
  \ifthenelse{\equal{\detokenize{#2}}{\detokenize{}}}{}
  {(#2)}}

% matrices
\NewDocumentCommand{\inm}{ O{} O{} }{\mathbf{I}_{#1}%
  \ifthenelse{\equal{\detokenize{#2}}{\detokenize{}}}{}
  {(#2)}}
\NewDocumentCommand{\outm}{ O{} O{} }{\mathbf{O}_{#1}%
  \ifthenelse{\equal{\detokenize{#2}}{\detokenize{}}}{}
  {(#2)}}

% sets
\newcommand{\set}{\ensuremath{\mathcal{E}}\xspace}
\newcommand{\places}{P}
\newcommand{\transitions}{T}
\newcommand{\markings}{\mathcal{M}}
\newcommand{\setm}{\markings}
\newcommand{\naturals}{\mathbb{N}}
\newcommand{\integers}{\mathbb{Z}}
\newcommand{\parameters}{\mathbb{P}}
\newcommand{\range}[2]{\{#1, \dots, #2\}}
\newcommand{\ucs}{\mathcal{U}} %upward-closed set
\newcommand{\setp}{\mathcal{Q}} %set of places
\newcommand{\setv}{\mathcal{V}} %set of valuations

\newcommand{\front}{F}

% sequence
\newcommand{\seq}[1][]{\ifthenelse{\equal{\detokenize{#1}}{}}%
  {\mathscr{S}}%
  {\elem{\mathscr{S}}{#1}}}
\newcommand{\seqt}[1][1]{%sequence of transitions
  \ifthenelse{\equal{\detokenize{#1}}{\detokenize{1}}}{\sigma}{\rho}}
\newcommand{\iscs}[1]{\seqt[2]_{#1}} % increasing self-covering sequence

\NewDocumentCommand{\defSeq}{ O{\seq} O{\sit} m m }{#1 = \bodySeq[#2]{#3}{#4}}
\NewDocumentCommand{\defInfSeq}{ O{\seq} O{\sit} O{1} O{2} O{} }{#1 = \bodyInfSeq[#2][#3][#4][#5]}
\NewDocumentCommand{\defSeqt}{ O{\seqt} m m }{#1 = \bodySeqt{#2}{#3}}
\NewDocumentCommand{\defInfSeqt}{ O{\seqt} O{t} O{1} O{2} O{} }{#1 = \bodyInfSeq[#2][#3][#4][#5]}

\newcommand{\bodySeqt}[2]{\bodySeq[\transition]{#1}{#2}}
\newcommand{\bodySeqm}[2]{\bodySeq[\mar]{#1}{#2}}

\NewDocumentCommand{\bodySeq}{ O{\sit} m m }{(\elemsSeq[#1]{#2}{#3})}
\NewDocumentCommand{\bodyInfSeq}{ O{\sit} O{1} O{2} O{} }{(\elemsInfSeq[#1][#2][#3][#4])}

\NewDocumentCommand{\elemsSeq}{ O{\sit} m m }{#1_{#2}, \dots, #1_{#3}}
\NewDocumentCommand{\elemsInfSeq}{ O{\sit} O{1} O{2} O{} O{,} O{\expandafter\dots}}{#1_{#2}#5 #1_{#3}#5 #6{#4}}

\newcommand{\mpath}[2]{#1 \leadsto #2}

% seq functions
\newcommand{\concat}{+}
\newcommand{\oldslice}[3][\seq]{\fourIdx{#2}{}{#3}{}{#1}}
\newcommand{\olddslice}[3][\seq]{#2 \overset{#1}{\leadsto} #3}
\newcommand{\chooseslice}[3][\seq]{\text{\todo{Choose a notation}} \oldslice[#1]{#2}{#3} = \olddslice[#1]{#2}{#3}} 
\newcommand{\slice}[3][\seq]{{#1}[{#2}:{#3}]}
\newcommand{\elem}[2]{#1^{(#2)}}

% units
\newcommand{\mar}{\ensuremath{\mathbf{m}}\xspace}
\newcommand{\mari}{\ensuremath{\mar_0}\xspace}
\newcommand{\marp}{\ensuremath{\mathbf{m'}}\xspace} % deprecated
\newcommand{\transition}{t} % fixed
\newcommand{\place}{p} % fixed
\newcommand{\sit}{e} %sequence item
\newcommand{\param}[1][1]{
  \ifthenelse{\equal{\detokenize{#1}}{\detokenize{1}}}
  {a}{%
    \ifthenelse{\equal{\detokenize{#1}}{\detokenize{2}}}
    {b}{%
      \ifthenelse{\equal{\detokenize{#1}}{\detokenize{3}}}
      {c}{%
        \ifthenelse{\equal{\detokenize{#1}}{\detokenize{4}}}
        {c}{!!!\errmessage{unknown param number}}}}}}


% relations
\newcommand{\rela}{\mathrel{R}}

% operators
\DeclareMathOperator{\Ab}{\zeta}
\DeclareMathOperator{\Cover}{Cover}
\DeclareMathOperator{\Downc}{\downarrow^\preceq}
\DeclareMathOperator{\Effect}{Effect}
\DeclareMathOperator{\KMAcc}{Acceleration}
\DeclareMathOperator{\Labt}{\lambda}
\DeclareMathOperator{\Lab}{\Lambda}
\DeclareMathOperator{\Maxs}{Max^\sqsubseteq}
\DeclareMathOperator{\Max}{Max^\preceq}
\DeclareMathOperator{\Min}{Min^\preceq}
\DeclareMathOperator{\Na}{\alpha}
\DeclareMathOperator{\Oplaces}{\Omega}
\DeclareMathOperator{\Post}{Post}
\DeclareMathOperator{\Pre}{Pre}
\DeclareMathOperator{\Tts}{\varsigma}
\DeclareMathOperator{\Upc}{\uparrow^\preceq}
\DeclareMathOperator{\Cov}{cov}
\DeclareMathOperator{\Vcov}{v-cov}
\newcommand{\Treepath}{\mathscr{T}}
\newcommand{\branch}{\mathscr{T}}

% functions
\newcommand{\card}[1]{\left|#1\right|}
\newcommand{\cover}[1]{\Cover(#1)}
\newcommand{\downc}[1]{\Downc(#1)}
\newcommand{\fire}[2][]{\xrightarrow{#2}_{#1}}
\newcommand{\kmacc}[1]{\KMAcc(#1)}
\newcommand{\maxp}[1]{\Max(#1)}
\newcommand{\maxs}[1]{\Maxs(#1)}
\newcommand{\minp}[1]{\Min(#1)}
\newcommand{\op}[1]{\Oplaces(#1)}
\newcommand{\posts}[2][]{\Post^*_{{#1}}(#2)}
\newcommand{\post}[2][]{\Post_{#1}(#2)}
\newcommand{\pres}[2][]{\Pre^*_{#1}(#2)}
\newcommand{\pre}[2][]{\Pre_{#1}(#2)}
\newcommand{\transpose}[1]{#1^T}
\newcommand{\treepath}[2][]{\slice[\Treepath]{#1}{#2}}
\newcommand{\upc}[1]{\Upc(#1)}
\newcommand{\val}[1][]{v\ifthenelse{\equal{\detokenize{#1}}{}}{}{(#1)}}
\newcommand{\cov}[2]{\Cov(#1,#2)}
\newcommand{\vcov}[2]{\Vcov(#1,#2)}



% old

\newcommand{\overbar}[1]{\overline{#1\mkern-1.5mu}\mkern 1.5mu}



%\newcommand{\N}{\ensuremath{\mathcal{N}}\xspace}
%\newcommand{\PPN}{\ensuremath{\mathcal{S}}\xspace}
%\newcommand{\PPNi}{\ensuremath{\mathcal{S}_1}\xspace}
%\newcommand{\PPNii}{\ensuremath{\mathcal{S}_2}\xspace}
%\newcommand{\PTm}{\ensuremath{\langle P,T, \mari\rangle}\xspace}
%\newcommand{\PT}{\ensuremath{\langle P,T\rangle}\xspace}
%\newcommand{\NPT}{\ensuremath{\N = \PT}\xspace}
%\newcommand{\NPTm}{\ensuremath{\N = \PTm}\xspace}
%\newcommand{\PTP}{\ensuremath{\langle P,T,\mathbb{P}\rangle}\xspace}
%\newcommand{\PTPm}{\ensuremath{\langle P,T,\mathbb{P}, \mari\rangle}\xspace}
%\newcommand{\PTPmi}{\ensuremath{\langle P_1,T_1,\mathbb{P}_1, \mar_{0,1}\rangle}\xspace}
%\newcommand{\PTPmii}{\ensuremath{\langle P_2,T_2,\mathbb{P}_2, \mar_{0,2}\rangle}\xspace}
%\newcommand{\SPTPm}{\ensuremath{\PPN = \PTPm}\xspace}
%\newcommand{\SPTPmi}{\ensuremath{\PPNi = \PTPmi}\xspace}
%\newcommand{\SPTPmii}{\ensuremath{\PPNii = \PTPmii}\xspace}
%\newcommand{\SPTP}{\ensuremath{\PPN = \PTP}\xspace}


\newcommand{\marpi}{\ensuremath{\mathbf{m'}_0}\xspace}
\newcommand{\marpp}{\ensuremath{\mathbf{m''}}\xspace}

\newcommand{\matIN}{\matI_\N}
\newcommand{\matIS}{\matI_\PPN}
\newcommand{\matI}{\mathbf{I}}
\newcommand{\matON}{\matO_\N}
\newcommand{\matOS}{\matO_\PPN}
\newcommand{\matO}{\mathbf{O}}

\newcommand{\sousrel}{\stackrel{\Sous}{\Longrightarrow}}
\newcommand{\surrel}{\stackrel{\Sur}{\Longrightarrow}}

\newcommand{\vect}[1]{\left(#1\right)}

\DeclareMathOperator{\Accelb}{A\overline{cce}l}
\DeclareMathOperator{\Acc}{Acc}
\DeclareMathOperator{\CovSeq}{CovSeq}
\DeclareMathOperator{\Flatten}{Flatten}
\DeclareMathOperator{\Postb}{P\overbar{ost}}
\DeclareMathOperator{\R}{\mathcal{R}}
\DeclareMathOperator{\Sous}{Under}
\DeclareMathOperator{\Sur}{Sur}
\DeclareMathOperator{\back}{Back}
\DeclareMathOperator{\vback}{vBack}
\DeclareMathOperator{\EcovOp}{\mathscr{E}cov}
\DeclareMathOperator{\UcovOp}{\mathscr{U}cov}

% TMP
% draft
\newcommand{\dtodo}[1]{[TODO: {\color{red} #1}]\message{TODO: #1}}
\newcommand{\dremoved}[1]{[REMOVED: {\color{gray} #1}]}
\newcommand{\dnote}[1]{{\em\color{gray} #1}}
% final
\newcommand{\ftodo}[1]{\message{TODO: #1}}
\newcommand{\fremoved}[1]{}%\message{REMOVED: #1}}
\newcommand{\fnote}[1]{{\em\color{gray} #1}}

\newcommand{\todo}[1]{\dtodo{#1}}
\newcommand{\removed}[1]{\dremoved{#1}}
\newcommand{\note}[1]{\dnote{#1}}
\newcommand{\rev}[1]{\drev{#1}}
\newcommand{\mov}[1]{\dmov{#1}}
%\newcommand{\todo}[1]{\ftodo{#1}}
%\newcommand{\removed}[1]{\fremoved{#1}}
%\newcommand{\note}[1]{\fnote{#1}}
%\newcommand{\rev}[1]{\frev{#1}}
%\newcommand{\mov}[1]{\fmov{#1}}


\begin{document}

\frontmatter
\begin{titlepage}
\setcounter{page}{0}
\begin{center}
\textbf{UNIVERSIT\'E LIBRE DE BRUXELLES}\\
\textbf{Faculté des Sciences}\\
\textbf{Département d'Informatique}
\vfill{}\vfill{}

{\Huge  The Coverability problem \vspace*{.5cm}  \linebreak[4] for parametric Petri nets}

{\Huge \par}
\begin{center}{\LARGE Alexis Reynouard}\end{center}{\Huge \par}
\vfill{}\vfill{}
\begin{flushright}{\large \textbf{Promoter :} Prof. Gilles Geeraerts}\hfill{}{\large Master Thesis in Computer Sciences}\\
{\large }\hfill{}{}\end{flushright}{\large\par}
\vfill{}\vfill{}\enlargethispage{3cm}
\textbf{Academic year 2018~--~2019}
\end{center}
\end{titlepage}
\newpage
\thispagestyle{empty}
\null

\newenvironment{vcenterpage}
{\newpage\thispagestyle{empty} 
\vspace*{\fill}}
{\vspace*{\fill}\par\pagebreak}

%\begin{vcenterpage}
%\begin{flushright}
%    \large\em\null\vskip1in 
%    You may want\\
%   to write a dedication here\vfill
%  \end{flushright}
%\end{vcenterpage}
%\thispagestyle{empty}
%\vspace*{5cm}
%
%\begin{quotation}
%\noindent ``\emph{You may also include one or more general quotes related to your topic.}''
%\begin{flushright}\textbf{Name of the author, date}\end{flushright}
%\end{quotation}
%
%\medskip
%
%\begin{quotation}
%\noindent ``\emph{Another quote.}''
%\begin{flushright}\textbf{Name of the author, date}\end{flushright}
%\end{quotation}
%\chapter*{Acknowledgment}
%\thispagestyle{empty} 
%
%\noindent I want to thank ...

\thispagestyle{empty} 
\setcounter{page}{0}
\tableofcontents
\acresetall

\mainmatter
\setcounter{page}{1}

\chapter{Introduction}
\label{sec:introduction}
\acp{PN} are a mathematical and graphical model introduced by Carl Adam Petri in 1962 \citep{Petri62,Petri66}.
It was successfully used to analyse systems in a wide range of domains, and has proven to be particularly successful for the formal verification of asynchronous systems, like manufacturing systems \cite{li2009deadlock}.
\Cref{sec:some-uses-of-pn}

In their standard definition, \acp{PN} are instantiated through many natural numbers\footnote{\ie a \ac{PN} may be represented as a pair of matrices whose the values are from $\naturals$, see \Cref{sec:the-pn-model}.} which may represent, for example, the amount of resource needed for a given action to be carried out.

The introduction of parameters into the model to avoid the need to state these values explicitly%
\footnote{One can find in the literature many other way to use parameters in \acp{PN}. For example, place and\,/\,or transitions may also be parameters in order to dynamically change the network structure, like in \cite{Christensen97}.}
may have several benefits:
it can allow an efficient analysis of a whole family of \acp{PN}, like in \cite{Abdulla13}, or to model dynamic changes in the system, as introduced by \cite{Badouel99} as a subclass of reconfigurable nets.

The use of parameters increases the modelling power of \acp{PN} but also make some basic coverability problems undecidable in the general case \cite{David17}.

We adopt the parametric Petri net model introduced by \cite{David17}, which seems the most intuitive and general, \rev{and we study the existing results and algorithms for plain Petri nets to find out whether they still hold or how to adapt them to the parametric model.}

The rest of the document is as follows:
\rev{In this first part, we define the plain Petri net model (\ie the classical one) and the parametric model.
We then briefly motivate our study, \todo{give concrete examples of applications,} and give an overview of the previous works on parametrisation of \acp{PN}.
Finally, we place the \ac{PN} model in a broader model family: the \acp{WSTS}.\\
In a second part, we recall, first, the classical results that we will study on this new model, second, the results already obtained for the parametrized Petri net model as we have defined them.\\
Then, we focus on the parametric Petri net model to establish whether the results related to the coverability problem in the plain Petri net model still hold or if the algorithms may be adapted to this new model.}

\acresetall


%
\section{Context}
\label{sec:context}

%%
\subsection{Previous Works on Parametrization of \acp{PN}}
\label{sec:previous-works-on-parametrization-of-pn}
The use of parameters in formal verification systems is a well developed topic in the literature.

With regard to \ac{PN}, parameters have been introduced with many different roles.
Some works, like \cite{Christensen97} use parameters as places or transitions, for example to make it possible to change a place into a more complex subnet and thus allow different levels of abstractions to be considered.
In \cite{Lindqvist91}, parameters are used on the markings to obtain concise parametrised reachability trees, but not to realize formal verifications on these parametric systems.

\cite{Badouel99} introduce parameters as the weight of arcs to model changes in a system.
The parameters have a finite valuation domain and verifications are performed on these parametrized systems.
Systems with quantitative parameters with infinite valuation domains are analysed in \cite{Abdulla13}.
Similarly, \cite{Marsan94} study \acp{PN} with parametric initial markings.
%(called \ac{PN} models in contrast to \ac{PN} systems where the initial markings does not have parameters).

Our work is in the line with \cite{David17} which use discrete parameters as arc weights as well as in the markings.
\cite{David17} provide also a proof for the non decidability of \Ucov and \Ecov, and define several subclasses of \ac{PPN} for which these problems are decidable.


\todo{\textomega-markings and \textomega-Petri nets \cite{Geeraerts15}}

% vim: spell spelllang=en :


%%
\subsection{Overview of Similar Models}
\label{sec:overview-of-similar-models}
Many reactive systems are naturally modeled as infinite-state systems.
Some infinite-states models are known to allow some automatic formal verification from model-checking techniques.
Among them are \acp{PN}, but also Lossy FIFO systems, Broadcast protocols...
\todo{refs}
All are \acp{WSTS} (but there exists other famous infinite-states systems, like timed-automata).
\todo{check whether timed-automata are \acp{WSTS}.}

\acp{WSTS} are outside of the range of this work, but we will sometimes refer to them to distinguish between techniques specific to the \ac{PN} model or usable in a wider range of contexts.
\todo{Check english writting.}

In a few words, \acp{WSTS} are transitions systems whose set of states are well-quasi-ordered and whose transition relations is monotonic with respect to the well quasi-ordering.

The monotonicity property differ from the strong monotonicity defined above by the fact that the second state may be found after many steps. Formally:
\begin{defi}[Monotonicity]
  A transition system is said \emph{monotonic} whenever \todo{or `when'?} its transition relation is monotonic.

  A transition relation $\rightarrow \subseteq (S \times S)$ over a $\leq$-well quasi-ordered set $S$ is monotonic if, and only if, for all $s_1$, $s_2$, and $s_3$ from $S$ such that $s_1 \leq s_2$ and $s_1 \rightarrow s_3$ there exists $s_4 \in S$ such that $s_2 \fire{*} s_4$.
\end{defi}


%
\section{Motivations}
\label{sec:motivations}

%%
\subsection{Interests of \acp{PPN}}
\label{sec:interests-of-ppn}
\todo{sources and examples}

Today \acp{PN} are used in a wide range of areas.
They are commonly used either to design a safe system, or to verify an existing one.
These uses require that the system is complete.
That is, for the design of a model, it must be entirely designed to be analysable.
On the other hand, when checking an existing system, if a desired property does not hold, the correction must be done by hand.

With the introduction of parameters some variables unknown at the design stage can be integrated into the model without having to be set arbitrarily. Moreover, if during the verification a desired property turns out not to hold, it is possible to check if the change of parameters alone can solve the problem, or if the Petri net structure must be changed too. Going further, the use of parameters in the model can allow to determine ``the safest values'' for a system, or to synthesize the values that allow to respect a given strategy.

We can therefore say that parameters can simplify the \emph{design} of a system. Indeed, since it is possible to keep unknown values, modelling can be done step by step, with the possibility to check the model at each step.
In addition, the design can be partially automated by parameter synthesis.
This approach gives a new interest in this model in fields as varied as chemistry, construction processes, financial loans...
\cite{David17} contains good illustrative examples.

There are also many advantages of using parameters when it comes to \emph{verification}.
For example, they allow to verify some properties simultaneously on many systems that differs only by parameters values.


%%
\subsection{Interest of the coverability problem in \acp{PPN}}
\label{sec:interest-of-the-coverability-problem-in-ppn}
\todo{sources and examples}

As it provides evidence of safety properties on the studied systems, coverability problem is of primary interest in system design and verification. Therefore, for the reasons given in the previous section, it is worth being able to solve it efficiently on \acp{PPN}.

To give a more concrete intuition on the interest, consider a system that execute a \emph{task} for others systems.
At each instant (whatever an instant is), the system may receive requests to perform the task from many other systems. We say that each request creates a \emph{job}.
We would like to have a system that is not too expensive to implement, but also capable of completing the tasks quickly enough.
To this end, we make our system capable of performing $a$ jobs at the same time, keeping $a$ as low as possible to reduce costs.
This system may be modelled as shown on the figure~\ref{fig:parametric-petri-net-example} with $\mari = (0, \param[1], 0)$ as initial marking.
$p_1$ represents the job queue and $p_3$ the execution unit.\\
We can now formally verify that, whatever the parameter values, the execution unit will not receive more than $a$ jobs to carry out at the same time, that is an instance of the \Ecov for the marking $(0, 0, \param[1]+1)$.
Indeed, it is easy to see that $\Post^*((0, \param[1], 0)) = \setComp{(i, j, k)}{i, j \text{ and } k \in \naturals, j + k = \param[1]}$.
\todo{maybe remove part on `$a$ as low as possible'}

Before recalling the known results on \ac{PPN} and plain \ac{PN} that will be useful for our study, let us give a brief overview of some work already done on \acp{PPN}.
% vim: syn spell toplevel :


\acresetall

\chapter{The Coverability Problem}
\label{sec:the-coverability-problem}

%
\section{The \ac{PN} model}
\label{sec:the-pn-model}
%\paragraph{Plain \ac{PN}}

\begin{defi}[\acl{PN}]
  \label{defi:pn}
  A \acf{PN} \namePN is a weighted oriented bipartite graph, whose the two subsets of vertices define a tuple $\tuple{\places, \transitions}$ where:
  \begin{itemize}
    \item $\places$ is a finite set of places,
    \item $\transitions$ is a finite set of transitions.
  \end{itemize}
  For each transition $t \in \transitions$ are defined (exactly) these two functions:
  \begin{itemize}
    \item $\inw[t] : \places \mapsto \naturals$ associates to each place the weight of the edge to $t$ \emph{(input weight)},
    \item $\outw[t] : \places \mapsto \naturals$ associates to each place the weight of the edge from $t$ \emph{(output weight)}.
  \end{itemize}
  It is denoted by $t = \tuple{\inw[t], \outw[t]}$.
  Because these functions define the edges of the graph, a \ac{PN} is completely defined by the tuple $\tuple{\places, \transitions}$ and so is denoted by $\defNonMarkedPN$.
\end{defi}

\begin{defi}[marking]
  Given a set of places $\places$, a marking of $\places$ is a function $\mar : \places \mapsto \naturals$ that associates $\mar(p)$ tokens to each place $p \in \places$.

  A marking of a \ac{PN} $\defNonMarkedPN$ is a marking of $\places$.
\end{defi}

An order on the markings is essential for the analysis of \acp{PN}.
The order we will define is a well-quasi-order and a partial order.

\begin{defi}[quasi-order]
  A quasi-order, or preorder, on a set $\set$ is a binary relation $\rela$ that is:
  \begin{align*}
    \text{reflexive: }  &&\forall x \in \set,\ & x \rela x \\
    \text{transitive: } &&\forall (x, y, z) \in \set^3,\ & (x \rela y\land y \rela z)\Rightarrow x \rela z
  \end{align*}
\end{defi}

\begin{defi}[well-quasi-order]
  \label{defi:well-quasi-order}
  A \emph{well-founded} quasi-order, or well-quasi-order, on a set $\set$ is a quasi-order $\rela$ on $\set$ such that, for any infinite sequence $\defInfSeq$ of elements from $\set$, there exist indices $i < j$ with $\sit_i \rela \sit_j$. That is, there is no infinite antichain in $\set$ for this relation.
\end{defi}

\begin{defi}[partial order]
  A partial order on a set $\set$ is a quasi-order $\rela$ that is
  \begin{align*}
    \text{antisymmetric: } &&\forall (x, y) \in \set^2,\ & (x \rela y\land y \rela x)\Rightarrow x = y
  \end{align*}
\end{defi}

\begin{defi}[partial order \(\preceq\) on the markings]
  Given a set of places $\places$, the partial order \(\preceq \subseteq \naturals^{\card{\places}} \times \naturals^{\card{P}}\) is such that for all pair of markings \((\mar_1, \mar_2) \in \naturals^{\card{\places}} \times \naturals^{\card{\places}}\) we have that \(\mar_1 \preceq \mar_2\) if and only if for all place \(p \in \places : \mar_1(p) \leq \mar_2(p)\).

  $\mar_2$ is said to \emph{cover} $\mar_1$.
\end{defi}

\(\mar \prec \marp\) denotes that \(\mar \preceq \marp \text{ and } \marp \npreceq \mar\).

\begin{lemm}[\cite{Dickson13}]
  \label{lemm:wqo}
  $\preceq$ is a well-quasi-order.
\end{lemm}

\todo{proof(?)}

%\paragraph{Parametric \ac{PN}}

In this work we will focus on an extension of the \ac{PN} model, the \ac{PPN} model, that is extended thanks to the use of parameters as input and output weights or as number of tokens in some places of the initial marking.

\begin{defi}[\acl{PPN} \citep{David17}]
  A \acf{PPN} $\defNonMarkedPPN$ is a weighted oriented bipartite graph with a finite set $\parameters$ of parameters. The two subsets of vertices are:
  \begin{itemize}
    \item $\places$: a finite set of places,
    \item $\transitions$: a finite set of transitions,
  \end{itemize}
  For each transition $t \in \transitions$ are defined the following functions:
  \begin{itemize}
    \item $\inw[t] : P \mapsto \naturals \cup \parameters$ associates to each place the weight of the edge to $t$ \emph{(input weight)},
    \item $\outw[t] : P \mapsto \naturals \cup \parameters$ associates to each place the weight of the edge from $t$ \emph{(output weight)}.
  \end{itemize}
\end{defi}

As for plain \acp{PN}, this is denoted $t = \tuple{\inw[t], \outw[t]}$.

\begin{defi}[parametric marking]
  Given a set of places $\places$, a parametric marking of $\places$ is a function $\mar : \places \mapsto \naturals \cup \parameters$ that associates $\mar(p)$ tokens to each place $p \in \places$.
\end{defi}

As a marking of a \ac{PN} $\defNonMarkedPN$ is a marking of $\places$,
a marking of a \ac{PPN} $\defNonMarkedPPN$ is a \emph{parametric} marking of $\places$.
Note that a (plain) marking \mar is a parametric marking where $\mar(p) \in \naturals$ for all $p \in \places$.

More generally, we adopt the following convention:
a marking is a function whose the domain is $\places$ and the codomain is a super set of $\naturals$.
Let \set be a set. An \set-marking $\mar : \places \mapsto \naturals \cup \set$ is a marking over $\naturals \cup \set$.
With an element $x$, an $x$-marking $\mar : \places \mapsto \naturals \cup \{x\}$ is a marking over $\naturals \cup \{x\}$.

\begin{defi}[initialized (parametric) \ac{PN}]
  An initialized \ac{PN} $\defPN$ (resp. \ac{PPN} $\defPPN$) is a \ac{PN} (resp. \ac{PPN}) with an initial marking \mari.
\end{defi}

This is sometimes called a \emph{marked (parametric) \ac{PN}}.
We will often refer to an initialized (parametric) \ac{PN} loosely as a (parametric) \ac{PN}.

The figure~\ref{fig:parametric-petri-net-example} shows an example of \ac{PPN} whose $\parameters = \{\param[1], \param[2]\}$ and with an initial marking \mari such that $\mari(p_1) = 1$, $\mari(p_2) = \param[1]$ and $\mari(p_3) = 0$.
The circles represent the places, the rectangles are the transitions, and the dots are the tokens.
If the number of tokens at a given place is parametric (\ie depends on a parameter of $\parameters$), it is written inside the circle.
An arrow from a place $p$ to a transition $t$ denotes that $\inw[t](p) = 1$.
The absence of an arrow from $p$ to $t$ indicates that $\inw[t](p) = 0$.
If $\inw[t](p) > 1$, a label with the value of $\inw[t](p)$ is added to the arrow.
Symmetrically, the arrows from the transitions to the places indicate the output weights.

\begin{figure}[htbp]
  \centering
  \begin{tikzpicture}[auto,x=0.12\linewidth,y=0.11\linewidth]
	\node [place,tokens=1] (d) [label=$p_1$] at (4,2) {};
	\node [place] (l) [label=west:$p_2$] at (4,1) {$a$};
	\node [place] (o) [label=east:$p_3$] at (6,1) {};
	
	\node [transition] (S) [label=$t_1$] at (3,2) {}
  edge [post] node [auto] {$b$} (d);
	\node [transition] (C) [label=$t_2$] at (5,2) {}
	edge [pre]  (d)
	edge [pre,  bend right] (l)
	edge [post, bend left]  (o);
	\node [transition] (F) [label=$t_3$] at (5,0) {}
	edge [pre,  bend right] (o)
	edge [post, bend left]  (l);
\end{tikzpicture}

  \par
  \caption{An initialized \ac{PPN}}
  \label{fig:parametric-petri-net-example}
\end{figure}

We usually set an order on the places.
This allows to view the markings as vectors
(here, \mari is the column vector $\transpose{(1, \param[1], 0)}$, where $\transpose{\cdot}$ is the transpose operator)
as well as the $\inw[t]$ and $\outw[t]$ functions ($t \in \transitions$).
Likewise, we define an order on the transitions.
Therefore, with $t \in \naturals$, $\inw[t]$ and $\outw[t]$ denote respectively the $\inw$ and $\outw$ functions defined for the $t$\textsuperscript{th} transition
(here, $\inw[1] = \transpose{(0, 0, 0)}$ and $\outw[1] = \transpose{(\param[2], 0, 0)}$).
Given a \ac{PPN} $\defNonMarkedPPN$, the backward and forward incidence matrices
$\inm[\namePPN] \in (\naturals \cup \parameters)^{\card{\places} \times \card{\transitions}}$
and
$\outm[\namePPN] \in (\naturals \cup \parameters)^{\card{\places} \times \card{\transitions}}$
are naturally defined by $\inm[\namePPN][p, t] = \inw[t][p]$
and $\outm[\namePPN][p, t] = \outw[t][p]$.
($\namePPN$ is omitted when it is obvious from the context.)
%In addition, it makes the equivalence between \acp{PN} and \emph{vector addition systems} introduced in \cite{Karp69} more explicit and
This allows to use linear algebra for the analysis of \acp{PN}.

\begin{figure}[htbp]
	\[
		\inm = \bordermatrix[{[]}]{%
					& t_1 & t_2 & t_3 \cr
			p_1 & 0   & 1   & 0   \cr
			p_2 & 0   & 1   & 0   \cr
			p_3 & 0   & 0   & 1   }
		\mspace{56mu}
		\outm = \bordermatrix[{[]}]{%
					& t_1       & t_2 & t_3 \cr
      p_1 & \param[2] & 0   & 0   \cr
			p_2 & 0         & 0   & 1   \cr
			p_3 & 0         & 1   & 0   }
	\]
  \caption{The incidence matrices of the \ac{PPN} from figure \ref{fig:parametric-petri-net-example}}
  \label{fig:incidence-matrices-example}
\end{figure}

%\begin{defi}[Vector addition system]
%  A vector addition system of dimension $n$ is a pair $\langle d, W\rangle$ where $d \in \naturals^n$ is called the \emph{start vector} and $W$ is a finite set of vectors $\mathbb{Z}^n$.
%\end{defi}
%This corresponds to the definition of an initialized \ac{PN} and \todo{we will see that the semantic corresponds too}.


%
\section{Operational semantic of \acp{PN}}
\label{sec:operational-semantic-of-pn}
Given a \ac{PN} $\defNonMarkedPN$ and a marking \mar on \namePN, a transition $t \in \transitions$ is said \emph{enabled} by \mar if $\forall p \in \places : \mar(p) \geq \inw[t][p]$.
An enabled transition can be \emph{fired} to produce a new marking \marp such that $\forall p \in \places : \marp(p) = \mar(p) - \inw[t][p] + \outw[t][p]$.
This is denoted by $\mar \fire{t} \marp$.

Here are some additional notations:
\begin{itemize}
  \item $\mar \fire{} \marp$ denotes that there exists $t \in \transitions$ such that $\mar \fire{t} \marp$.
  \item $\mar \fire{\seqt} \marp$ where $\seqt$ is a sequence of transitions $\defSeqt{1}{n-1}, t_i \in \transitions, i \in \range{1}{n-1}$ denotes that there exists a sequence of markings $\bodySeqm{1}{n}$ such that : $\mar = \mar_1 \fire{t_1} \cdots \fire{t_{n-1}} \mar_n = \marp$.
  \item $\mar \fire{*} \marp$ denotes that there exists a sequence of transition $\seqt$ such that $\mar \fire{\seqt} \marp$.
    Note that the $\fire{*}$ relation is the reflexive and transitive closure of the relation $\fire{}$.
\end{itemize}

It is important to note that the effect of a transition is to add or remove a constant number of tokens at each place and does not depend on the marking from which it is fired.
A \ac{PN} transition is said to have a \emph{constant effect}.
Thus, the effect of a transition in a \ac{PN} is a function $\effect{t}$ that maps each place $p$ to $- \inw[t][p] + \outw[t][p]$.
This definition is extended to any sequence of transitions $\seqt$ as $\effect{\seqt}[p] = \sum_{t \in \seqt} -\inw[t][p] + \outw[t][p]$.

The following well-known results will be useful in the sequel.

\begin{lemm}[Strong monotonicity of PN]
  \label{lemm:strong-monotonicity-of-pn}
  The \ac{PN} model is \emph{strongly monotonic with regard to $\preceq$}.
  That is, for all \ac{PN} $\defPN$, for all transition $t \in \transitions$ and for all markings $\mar_1, \mar_2, \mar_3$ of \namePN such that $\mar_1 \preceq \mar_2$ and $\mar_1 \fire{t} \mar_3$, there exists a unique marking $\mar_4$ of \namePN such that $\mar_2 \fire{t} \mar_4$ and $\mar_3 \preceq \mar_4$.
\end{lemm}

\begin{proof}
  $t$ is enabled in $\mar_1$.
	Thus, for all $p \in \places : \inw[t][p] \leq \mar_1(p)$.
	Since $\mar_1 \preceq \mar_2$, we have for all place $p \in \places$ : $\inw[t][p] \leq \mar_1(p) \leq \mar_2(p)$.
	Thus $t$ is enabled in $\mar_2$.
	Since the effect of transitions is constant, $\mar_1 \preceq \mar_2 \Rightarrow \mar_3 \preceq \mar_4$.
\end{proof}

\begin{lemm}[Strict strong monotonicity of PN]
  The \ac{PN} model has \emph{strict strong monotonicity with regard to $\preceq$}.
	That is, for all \ac{PN} $\defPN$, for all transition $t \in \transitions$ and for all markings $\mar_1, \mar_2, \mar_3$ of \namePN such that $\mar_1 \prec \mar_2$ and $\mar_1 \fire{t} \mar_3$, there exists a marking $\mar_4$ of \namePN such that $\mar_2 \fire{t} \mar_4$ and $\mar_3 \prec \mar_4$.
\end{lemm}

The proof is very similar to the one of the \cref{lemm:strong-monotonicity-of-pn}.

We give a general definition of the \emph{monotonicity} property in transitions systems in \cref{defi:monotonicity}.

\begin{defi}
  Given an \ac{PN} $\defNonMarkedPN$ and a marking \mar of \namePN:
  \begin{itemize}
    \item $\post{\mar} = \setComp{\mar'}{\mar \fire{} \mar'}$ is the set of one-step successors of \mar,
    \item $\pre{\mar} = \setComp{\mar'}{\mar' \fire{} \mar}$ is the set of one-step predecessors of \mar,
    \item $\posts{\mar} = \setComp{\marp}{\mar \fire{*} \marp}$ is the set of successors of \mar, in any number of step.
      With $\mari$ the initial marking of \namePN, $\posts{\mari}$ is the \emph{reachability set} of \namePN.%, denoted \reach{\nameNP}.
    \item $\pres{\mar} = \setComp{\marp}{\marp \fire{*} \mar}$ is the set of predecessors of \mar, in any number of step.
  \end{itemize}
\end{defi}

These operators are naturally extended to sets of markings as the union of the sets obtained by applying the operator on each marking of the sets.
That is, with $\markings$ a set of markings of \namePN,
$\post{\markings} = \setComp{\marp}{\exists \mar \in \markings : \mar \fire{} \marp}$.

For example, regarding the \ac{PPN} shown on figure \ref{fig:parametric-petri-net-example},
$\post{\vect{0,1,0}} = \{\vect{b, 1, 0}\}$
and
$\posts{\vect{0,1,0}} = \setComp{\vect{i, 1, 0}}{i \in \naturals} \cup \setComp{\vect{i, 0, 1}}{i \in \naturals}$.

All of this applies to \ac{PPN} through valuations of the parameters:
\begin{defi}[Instantiation of \acp{PPN}]
  Let $\defPPN$ be a \ac{PPN} and $\val : \parameters \mapsto \naturals$ be a \emph{$\naturals$-valuation}, or simply valuation, on $\parameters$.
  Then $\val[\namePPN]$ is defined as the \ac{PN} obtained by replacing each parameter $\param \in \parameters$ by $\val[\param]$.
  Thus, we have $\val[\namePPN] = \tuple{\places, \transitions, \mari'}$ such that:
  \begin{itemize}
    \item $\inm[\val(\namePPN)][p, t] =
      \begin{cases}
        \phantom{\val(}\inm[\namePPN][p, t]  & \text{if } \inm[\namePPN][p, t] \in \naturals \\
                 \val( \inm[\namePPN][p, t]) & \text{if } \inm[\namePPN][p, t] \in \parameters
      \end{cases}$
    \item $\outm[\val(\namePPN)][p, t] =
      \begin{cases}
        \phantom{\val(}\outm[\namePPN][p, t]  & \text{if } \outm[\namePPN][p, t] \in \naturals \\
                 \val( \outm[\namePPN][p, t]) & \text{if } \outm[\namePPN][p, t] \in \parameters
      \end{cases}$
    \item $\mari'(p) =
      \begin{cases}
        \phantom{\val(}\mari(p)  & \text{if } \mari(p) \in \naturals \\
                 \val( \mari(p)) & \text{if } \mari(p) \in \parameters
      \end{cases}$
  \end{itemize}
\end{defi}

Given $\namePPN$ a \ac{PPN} and a valuation $\val$, one can therefore instantiate a \ac{PN} $\val[\namePPN]$ from $\namePPN$ and apply the semantic described above.
When the \ac{PPN} under consideration is clear from the context, $\inm[\val]$ is used to denote $\inm[\val(\namePPN)]$ and $\outm[\val]$ to denote $\outm[\val(\namePPN)]$.
We write $\fire[\val]{t}$,
         $\fire[\val]{}$,
         $\fire[\val]{\sigma}$,
         $\fire[\val]{*}$,
         $\Post_v$,
         $\Pre_v$,
         $\Post^*_v$
     and $\Pre^*_v$
to denote $\fire{t}$,
          $\fire{}$,
          $\fire{\sigma}$,
          $\fire{*}$,
          $\Post$,
          $\Pre$,
          $\Post^*$
      and $\Pre^*$
on the plain \ac{PN} $\val(\namePPN)$.

This makes it possible to formally represent a system and interactions between its components.

Moreover, when working on set of valuations, it is often convenient to set an order on the parameters and to see the valuations as vectors.
Thus, for example with $\parameters = \{a, b\}$, we can consider the parameters in lexicographical order: $\parameters = (a, b)$.
Then the valuation $\val$ such that $\val(a) = 1$ and $\val(b) = 2$ may be seen as the vector $\val = (1,2)$.
Many definitions given for the markings are valid for any finite dimensional vector (that we call \emph{tuples}) whose the domain is a totally ordered countable set%
\footnote{A \emph{countable set} is a set that is finite or has the same size as $\naturals$.}.
So, by analogy, we may use $\preceq$, $\Min$, $\Max$… on valuations.

We will now define some properties that the model may have and that are usually of interest to show that the modelled system meets some requirements.


%
\section{Behavioural properties of \acp{PN}}
\label{sec:behavioural-properties-of-pn}
The markings basically indicate the state of the system.
Knowing if an initialized \ac{PN} may reach a given marking, that represents for example a bad state, is therefore essential to check properties of the modelled system.
This is the \emph{reachability problem}.

\begin{defi}[Reachability]
  Given an initialized \ac{PN} $\defPN$ and a marking \mar of \namePN, \mar is said reachable if $\mari \fire{*} \mar$.
\end{defi}

\begin{defi}[Place boundedness]
  \label{defi:place-boundedness}
  Given a marked PN $\defPN$,
  a place $p \in \places$ such that $\exists c \in \naturals : \nexists \mar \in \posts{\mar_0}, \mar(p) > c$ is said to be bounded.
  That is, if there exists an upper bound on the number of tokens in the place $p$ in the set of the reachable markings.
\end{defi}

However, the verification of safety properties are more often reduced to a \emph{coverability problem}, that is essentially asking if an initialized \ac{PN} can reach or exceed a given marking.

\begin{restatable}[Coverability]{defi}{coverability}
  Given an initialized \ac{PN} $\defPN$ and a marking \mar of \namePN, \mar is said coverable if there exists a marking \marp such that $\mar \preceq \marp$ and $\mari \fire{*} \marp$.

  A set of markings is said coverable whenever one of its markings is coverable.
\end{restatable}

\begin{defi}[Coverability problem]
  Given an initialized \ac{PN} $\defPN$ and a set $\markings$ of markings of \namePN, determine whether $\exists \mar \in \markings \text{ and } \marp \in \posts{\mari} \text{ such that } \mar \preceq \marp$.

  The coverability problem for a marking \mar is the coverability problem for the singleton $\{\mar\}$.
\end{defi}

The behaviour of a \ac{PPN} is defined by the behaviours of all the \acp{PN} that can be obtained by a valuation of its parameters.
So, for an initialized \ac{PPN} $\namePPN$, the coverability problem may be declined in an existential and an universal form.
The existential coverability problem (\Ecov) ask if there exists a valuation $\val$ such that \mar is coverable.
The universal coverability problem (\Ucov) ask if \mar is coverable for all valuations $\val$.

\begin{defi}[Universal and existential coverability problems]
  Given a \ac{PPN} $\defPPN$ and a set $\markings$ of non-parametric markings of $\namePPN$
  \begin{itemize}
    \item the \emph{existential coverability problem} ask if there is a valuation $\val$ for $\parameters$ such that $\markings$ is coverable,
    \item the \emph{universal   coverability problem} ask if $\markings$ is coverable for all valuations of $\parameters$.
  \end{itemize}
\end{defi}

For the sake of simplicity, we will denote that the instance of the coverability problem on the \ac{PN} \namePN for the set $\setm$ of markings of \namePN is positive by: $\cov{\namePN}{ \setm} = \top$, and is negative by $\cov{\namePN}{ \setm} = \bot$. The same apply to \Ecov and \Ucov through the operators $\EcovOp$ and $\UcovOp$ respectively.
% vim: syn spell toplevel :


%
\section{Sequence notations and terminology}
\label{sec:sequence-notations-and-terminology}
We make a parenthesis here to introduce the notations on the sequences we will use during this work.

With $\set$ a set, a sequence of elements of $\set$ is a list of its elements in a given order.
We usually write it within parentheses.

Given two sequences $\defSeq{1}{n}$ and $\defSeq[\seq'][\sit']{1}{n'}$ of elements from $\set$, a natural $c$ and two indices $i$, $j \in \range{1}{n}, i < j$ and an object $\sit$, we use the following notations:
\begin{itemize}
  \item $\seq \concat \seq' = (\elemsSeq{1}{n}, \elemsSeq[\sit']{1}{n'})$ is the concatenation of $\seq$ and $\seq'$,
  \item $c \cdot \seq = (\overbrace{\overbracket{\elemsSeq{1}{n}}, \overbracket{\elemsSeq{1}{n}}, {\dots}}^{c \text{ times}})$ is the concatenation of $\seq$ with itself, repeated $c$ times,
  \item $\seq[i]$ is the $i$th element of $\seq$, so here $\seq[i] = \sit_i$, 
  \item $\set^*$ is the set of all the sequences of elements of $\set$, that is a sequence of zero or more elements from $\set$; by extension,
  \item $\sit^*$ denotes $\{\sit\}^*$, that is a sequence of zero or more repetitions of $\sit$,
  \item $\slice{\sit_i}{\sit_j} = \bodySeq{i}{j}$ is the subsequence of $\seq$ starting from $\sit_i$ and ending with $\sit_j$.\\
    There may be several occurrences of the given elements in $\seq$.
    We consider by convention: for the end bound, its last occurrence; and for the beginning bound, its last occurrence occurring before the end bound.
    If this is not possible, either because one of the given elements is not in the sequence, or because there is no occurrence for the start bound before the end bound, then the denoted sub-sequence is empty,
  \item $\slice{}{\sit_j} = \slice{\seq[1]}{\sit_j}$ stands for the \emph{prefix}, \lang{i.e.} beginning, of the sequence, until the last occurrence of $\sit_j$,
  \item $\slice{\sit_i}{} = \slice{\sit_i}{\seq[\card{\seq}]}$ is the end of the sequence, from the last occurrence of $\sit_i$ to the end.
\end{itemize}


\chapter{Known Results on the Coverability Problem}
\label{sec:known-results-on-the-coverability-problem}
This chapter intends to introduce known results on both \ac{PN} and \ac{PPN}.
They are not extensively described here:
since they are mentioned only as far as they can help the presentation of the results in the next chapter, the intuition is often given without any detail nor proof.
The reader is therefore encouraged to refer to the mentioned sources.


%
\section{Prerequisites}
\label{sec:prerequisites}
%% Intro
%We now present the results related to the coverability problem on the plain \acp{PN} that we think are the most interesting.
To introduce these results, we need some additional definitions.
They will be given for plain \acp{PN}, but most of them are naturally extended to \acp{PPN}.

%% Covering set informal
Given an initialized \ac{PN} \namePN, the \emph{covering set} of \namePN is the set of the markings covered by at least one reachable marking of \namePN.
It is an over-approximation of the reachability set that is precise enough to solve the coverability problem, and is, therefore, interesting for our study.

%% Covering set formal
\begin{defi}[Covering set]
  Let $\defPN$ be an initialized \ac{PN}.
  The \emph{covering set} $\setm$ of \namePN, noted $\cover{\namePN}$, is the set $\setComp{\mar}{\exists \mar' \in \posts{\mari} : \mar \preceq \mar' }$.
\end{defi}

%% Covering set picture
\begin{figure}[htbp]
  \label{fig:reach-and-cover-example}
  \centering
  \subfloat[A \ac{PN} ($\card{\places} = 2$)]{
    \label{fig:two-net}
    \begin{tikzpicture}[auto,x=0.12\linewidth,y=0.11\linewidth]
  \node [place, tokens=4] (y) [label=$y$] at (0,0) {};
  \node [place] (x) [label=$x$] at (2,0) {};

  \node [transition] (1) [label=$t_1$] at (1,0) [transition] {}
    edge [pre] node[midway, above] {2} (y)
    edge [post] (x);
  \node [transition] (2) [label=$t_2$] at (3,0) [transition] {}
    edge [pre] (x);
    %\node at (1.5,-0.75) {\label{2net} (1)\quad Un réseau de Petri ($|P| = 2$)};
\end{tikzpicture}


  }

  \subfloat[The reachable markings]{
    \label{fig:two-reach}
    \begin{tikzpicture}[auto,x=1.1cm,y=1.1cm]
  \tikzset{
    >=stealth',
    axis/.style={thin, ->, line join=miter, color=gray},
    dot/.style={circle,fill=black,minimum size=4pt,inner sep=0pt,
          outer sep=-1pt}
  }
  \draw[axis,<->] (2.5,0) node(xline)[right] {$x$} -|
          (0,4.5) node(yline)[above] {$y$};

  \draw (1,1pt) -- (1,-1pt) node[anchor=north] {$1$};
  \draw (1pt,1) -- (-1pt,1) node[anchor=east] {$1$};

  \node[dot] (04) at (0,4) {};

  \node[dot] (12) at (1,2) {};
  \node[dot] (02) at (0,2) {};

  \node[dot] (20) at (2,0) {};
  \node[dot] (10) at (1,0) {};
  \node[dot] (00) at (0,0) {};

  \draw[->] (04) to node {$t_1$} (12);
  \draw[->] (12) to node {$t_1$} (20);
  \draw[->] (02) to node {$t_1$} (10);

  \draw[->] (12) to node[above] {$t_2$} (02);
  \draw[->] (20) to node[above] {$t_2$} (10);
  \draw[->] (10) to node[above] {$t_2$} (00);

  %\node at (1.2,-1) {\label{2reach} (2)\enspace Les marquages accessibles};
\end{tikzpicture}


  }\qquad
  \subfloat[The covering set]{
    \label{fig:two-cover}
    \begin{tikzpicture}[auto,x=1.1cm,y=1.1cm]
  \tikzset{
    >=stealth',
    axis/.style={thin, ->, line join=miter, color=gray},
    dot/.style={circle,fill=black,minimum size=4pt,inner sep=0pt,
          outer sep=-1pt}
  }

  \draw[axis,<->] (2.5,0) node(xline)[right] {$x$} -|
          (0,4.5) node(yline)[above] {$y$};

  \draw (1,1pt) -- (1,-1pt) node[anchor=north] {$1$};
  \draw (1pt,1) -- (-1pt,1) node[anchor=east] {$1$};

  \node[dot] at (0,4) {};

  \node[dot] at (1,2) {};
  \node[dot] at (0,2) {};

  \node[dot] at (2,0) {};
  \node[dot] at (1,0) {};
  \node[dot] at (0,0) {};

  \node[dot, color=black!60!white] at (0,3) {};
  \node[dot, color=black!60!white] at (0,1) {};
  \node[dot, color=black!60!white] at (1,1) {};

  %\node at (1.2,-1) {\label{2cover} (3)\enspace L'ensemble de couverture};
\end{tikzpicture}


  }
  \caption{Reachability and covering sets}
\end{figure}

\Cref{fig:two-net} shows a marked \ac{PN} with two places.
One can therefore represents the markings as points on a plane.
\Cref{fig:two-reach} shows the reachable markings in the form of an accessibility graph.
In~\ref{fig:two-cover} we see the covering set.

%% Unbounded places informal and self-covering sequence formal
Sometimes the number of tokens in a place is unbounded. %(\lang{c.f.} the place boundedness problem).
In a plain \ac{PN}, this is due to the existence of an increasing self-covering sequence.
\begin{defi}[Self-covering sequence]
  Given an initialized \ac{PN} $\defPN$,
  a self-covering sequence $\seqt$ is a sequence of transitions from $\transitions$ such that
  \(
    \mar_i \fire{\seqt} \mar_j
  \),
  with $\mari \fire{*} \mar_i$ %and $\seqt$ two sequences of transitions of $\transitions$
  and $\mar_i \preceq \mar_j$.
\end{defi}

\begin{lemm}
	\label{theo:self-cov-non-termination}
	The existence of a self-covering sequence is a necessary and sufficient condition for the \emph{non-termination} of a PN (\ie there exists an infinite transition sequence $\seqt$ enabled in $\mar_0$).
\end{lemm}

\begin{proof}
Note that, since $\mar_i \preceq \mar_j$, $\seqt$ is enabled in $\mar_j$.
In addition, the strong monotonicity of \acp{PN} ensures that, with $\mar'_j$ given by $\mar_j \fire{\seqt} \mar'_j$, we have $\mar_j \preceq \mar'_j$.
Thus, we see that it is a sufficient condition for the \emph{non-termination} of the system (the system may be able to fire transitions infinitely often).
Actually, since $\preceq$ is a well-quasi-order (Lemma~\ref{lemm:wqo}), one can find in any infinite sequence $\mari \fire{} \mar_1 \fire{} \dots$ two markings $\mar_i$ and $\mar_j$ such that $\mar_i \preceq \mar_j$.
Therefore, any infinite sequence is self-covering, and the existence of such a sequence is also necessary for the non-termination of the system.
\end{proof}

%% Increasing self-covering sequence formal
\begin{defi}[Increasing self-covering sequence]
  Given an initialized \ac{PN} $\defPN$,
  an increasing self-covering sequence $\seqt$ is a sequence of transitions from $\transitions$ such that
  \(
    \mar_i \fire{\seqt} \mar_j
  \),
  with $\mari \fire{*} \mar_i$
  and $\mar_i \prec \mar_j$.
\end{defi}

\begin{lemm}
	\label{theo:self-cov-non-termination}
	The existence of an increasing self-covering sequence is a necessary and sufficient condition for the existence of unbounded places.
\end{lemm}

\begin{proof}
  This condition is sufficient:
  Let $\setp \subseteq \places$ be the set of places $\setp = \setComp{p \in \places}{ \mar_i(p) < \mar_j(p)}$.
  $\setp \neq \emptyset$ since $\mar_i \prec \mar_j$.\\
  With a reasoning similar to the one above, we see that the strict strong monotonicity of \acp{PN} ensures that, with such a sequence, the \ac{PN} can reach a marking $\mar'_j$ given by $\mar_j \fire{\seqt} \mar'_j$ and such that $\mar_j \prec \mar'_j$.
  Because of the constant transition effect , we know that $\forall p \in \setp : \mar_j(p) < \mar'_j(p)$.
  The unboundedness of the places in $\setp$ follows.

  It is also necessary:
  Since the effect of a transition is bounded ($\forall p \in \places : \forall t \in \transitions : \effect{t}[p] \in \integers$), a place $p'$ may be unbounded only if there exists a particular sequence of transition $\defInfSeqt$ that is:
  \begin{itemize} 
    \item enabled in $\mar_0$,
    \item infinite,
    \item and, with $\bodyInfSeq[\mar]$ the markings given by $\mar_0 \fire{t_1} \mar_1 \fire{t_2} \mar_2 \fire{t_3} \dots$, ($\seqt$ is) such that one can extract from $\bodyInfSeq[\mar]$ a infinite subsequence $\bodyInfSeq[\mar']$ that is strictly increasing in $p'$ : for any pair $(k, l) \in \naturals^2, k < l \Rightarrow \mar_i(p') < \mar_j(p')$.
  \end{itemize}
  Since $\preceq$ is a well-quasi-order on the markings, there is a pair $(i, j) \in \naturals^2$ such that $i < j$ and $\mar'_i \preceq \mar'_j$.
  Furthermore, since $\mar'_i(p') < \mar'_j(p')$, we have $\mar'_i \prec \mar'_j$.
\end{proof}

%% omark informal
An \omark is a way to represent a set of markings which have the same number of tokens in some places, and may have any number of tokens, potentially an infinity, in the other places.

%% omark formal
\begin{defi}[\omark]
  We define \emph{$\omega$} to be such that:
  $\omega \notin \naturals$
  and for any constant $c \in \naturals$:
  \begin{itemize}
    \item $c \leq \omega$
    \item $\omega + c = \omega$
    \item $\omega - c = \omega$
  \end{itemize}

  \emph{An \omark} \mar over a set of places $\places$ is a function $\mar : \places \mapsto \naturals \cup \{\omega\}$ that associates $\mar(p)$ tokens to each place $p \in \places$.

  With $\parameters$ a set of parameters, $\omega \notin \parameters$,
  \emph{a parametric \omark} \mar over a set of places $\places$ is a function $\mar : \places \mapsto \naturals \cup \parameters \cup \{\omega\}$ that associates $\mar(p)$ tokens to each place $p \in \places$.
\end{defi}

Note that an \omark \mar is a parametric \omark where $\mar(p) \in \naturals \cup \{\omega\}$ for all places $p \in \places$.
Similarly, a parametric marking \mar is a parametric \omark where $\mar(p) \neq \omega$ for all places $p \in \places$.
As for parametric markings, we often refer to a parametric \omark simply as \omark.

Given an \omark \mar, $\op{\mar} \subseteq \places$ is the set of places $p$ such that $\mar(p) = \omega$, sometimes referred as \emph{\oplaces of \mar}. $\places \setminus \op{\mar}$ is the set of \emph{\noplaces of \mar}.

%% Coverability set informal
\acp{PN} with unbounded places have an infinite reachability set.
So the covering set is also infinite.
\emph{Coverability sets} are useful to give a finite representation of the covering set thanks to \omarks.
%%
In order to define them formally, we need to know about the maximal markings and the upward- and downward-closure of a marking set.

%% Max markings informal
The maximal markings of a set are those which are not covered by any other marking of the set.
%% Max markings formal
\begin{defi}[Maximal markings]
  Given a marking set  $\setm$, the set of maximal elements of $\setm$ is
  $\maxp{\setm} = \setComp{ \mar \in \setm}{\nexists \mar' \in \setm \text{ s.t. } \mar \prec \mar' }$.
\end{defi}

%% Closure formal
\begin{defi}[Upward- and downward-closure on markings]
  Let $\setm \subseteq \naturals^{\card{\places}}$ be a set of markings on the places $\places$:
  \begin{itemize}
    \item The \emph{upward-closure} of $\setm$, noted $\upc{\setm}$, is the set
      $\setComp{\mar \in \naturals^{\card{\places}}}{\exists \mar' \in \setm : \mar' \preceq \mar}$,
    \item The \emph{downward-closure} of $\setm$, noted $\downc{\setm}$, is the set
      $\setComp{\mar \in \naturals^{\card{\places}}}{\exists \mar' \in \setm : \mar \preceq \mar'}$.
  \end{itemize}
  The closure of a marking \mar is the closure of the singleton $\{\mar\}$.
\end{defi}

Once again, these definitions are applicable to
%any set of tuples of values from a countable set.
a broader classes of sets, namely the sets equipped with a well-quasi-order~\citep{Abdulla96}.

%% Up closure example
For instance, the upward-closure of $\mar = (1, 2, 3)$ is $\upc{\mar} = \setComp{(i, j, k)}{i \geq 1, j \geq 2, k \geq 3}$.

%% Closed set formal
\begin{defi}[Upward- and downward-closed set of markings]
  A set $\setm$ of markings is said \emph{upward-closed} if $\setm = \upc{\setm}$.
  It is said \emph{downward-closed} if $\setm = \downc{\setm}$.
\end{defi}

Note that the downward-closure of a downward-closed set is the set itself.
Thus, by definition, for any $\defPN$, the covering set $\cover{\namePN} = \downc{\posts{\mar_0}}$ is downward-closed.

%% Coverability set formal
\begin{defi}[Coverability set \citep{Finkel87,Finkel90}]
  Given an initialized \ac{PN} $\defPN$, a \emph{coverability set} $\setm$ of \namePN is a marking set  such that $\downc{\setm} = \cover{\namePN}$.
\end{defi}

%% Usefulness of coverability sets and omarks
Notice that, since $\downc{\mar}$ exists and is unique for all \omark \mar, an \omark may always stand for one and only one downward-closed set.
Symmetrically, it is known that any downward-closed set may be represented by a finite set of \omarks \cite{Abdulla96,Geeraerts06}. Indeed,

\begin{lemm}
  \label{theo:repr-downc-sets}
  For all downward-closed set $\setm$ of markings, there exists a finite set of \omarks $\setm'$ such that $\downc{\setm'} = \setm$.
\end{lemm}

\begin{proof}
  \todo{proof}
\end{proof}

In particular, finite coverability sets may effectively represent, thanks to \omarks, any covering set:
Having an \omark \mar in a coverability set of $\namePN$ denotes that for all marking $\mar_1$ such that $\forall p \in \places \setminus \op{\mar}, \mar_1(p) = \mar(p)$, there exists a marking $\mar_2$ in the covering set (and thus in the reachability set) of \namePN such that $\mar_1 \prec \mar_2$.

\begin{lemm}[Finite coverability set]
  \label{theo:finite-coverability-set}
  For all marked PN, there exists a finite coverability set.
\end{lemm}

\begin{proof}
  Since the covering set of a PN is downward-closed, this follows from \cref{theo:repr-downc-sets}.
\end{proof}

%% Max markings and minimal coverability set informal.
Moreover, the set of maximal reachable \omarks of \namePN exists, is finite, and forms a minimal coverability set:

%% Max markings and minimal coverability set formal.

\begin{gather*}
\downc{\maxp{\posts{\mari}}} = \downc{\posts{\mari}} \\*
\tand \\*
\forall \setm \text{ such that } \downc{\setm} = \downc{\posts{\mari}}\text{, we have } \card{\setm} \geq \card{\maxp{\posts{\mari}}}
\end{gather*}

Actually, this is a specific application of the following lemma:

\begin{lemm}
  \label{theo:finite-max-set}
  Given a set $\set$ with a well-quasi-order $\preceq$,
  for any subset $\sset$ of $\set$,
  the maximal elements of $\sset$ (wrt. $\preceq$) form a finite set such that $\downc{\maxp{\sset}} = \downc{\sset}$.
\end{lemm}

\todo{proof (finite and such that...)}

%% coverability problem 2 formal
It is worth noticing that, in the context of \acp{PN}, the coverability problem for a marking set  $\setm$ may be defined as follows:
\begin{defi}[Coverability problem]
  \label{defi:upclocovprblm}
  Given a \ac{PN} \namePN and an upward-closed set $\ucs = \upc{\setm}$ of markings of \namePN, determine whether $\posts{\mari} \cap \ucs \neq \emptyset$.
\end{defi}

\label{text:upward-closed-set-representation}
An upward-closed set is always infinite.
It can be effectively represented through its unique set of minimal elements whose it is the upward-closure:
\begin{lemm}
  \label{theo:upward-closed-set-representation}
  For all upward-closed set $\ucs \subseteq \naturals^{\card{\places}}$, its set of minimal elements $\minp{\ucs} = \{\mar \in \ucs \mid \nexists \mar' \in \ucs : \mar' \prec \mar\}$ is such that $\upc{\minp{\ucs}} = \ucs$.
\end{lemm}

The following lemma ensure that this set is finite:
\begin{lemm}[Dickson's lemma]
  \label{theo:dickson}
  For all $c \in \naturals$, every set of $c$-tuples of natural numbers have finitely many minimal elements.
\end{lemm}
This result may easily be extended to $c$-tuples of values from any ordered countable set.

Moreover, this representation is effective in the sense that the set may be manipulated through it.
\cite{Ganty09} gives a way to perform the operations on upward-closed sets through their minimal elements.

We will also need the following result:
\begin{lemm}[\citep{Abdulla96}]
  \label{theo:pre-upc}
  Given a \ac{PN} and an upward-closed set $\ucs$ of markings of \ac{PN}, $\pre{\ucs}$ is upward-closed too.
\end{lemm}

%\subsection{A backward algorithm \citep{Finkel90, Abdulla96}}
%
%We will now present an algorithm to solve the coverability problem for a marking \mar of a \ac{PN} $\defPN$.
%
%This algorithm was introduced by Abdulla \lang{et al.} \cite{Abdulla96} for well-structured transition systems, a more general class of models which includes \acp{PN}.
%It is close of the one introduced earlier in \cite{Finkel90}.
%
%Recall the definition for a marking of being coverable.
%\coverability*
%
%For convenience, we will use another equivalent definition.
%
%\begin{defi}[Coverability]
%  Given an initialized \ac{PN} $\defPN$ and an upward-closed set $\ucs$, $\ucs$ is said coverable if there exists a marking \mar' such that $\mar' \in \ucs$ and $\mari \fire{*} \mar'$.
%\end{defi}
%
%By choosing $\upc{\mar}$ as $\ucs$, these two definitions set out the same instance of the coverability problem.
%With a set of markings considered in the first definition, $\ucs$ may be the union of their upward-closure in the second.
%
%We say it is a backward algorithm in the sense that it is based on the computation of the set $\pres{\mar}$ and answer by checking whether $\mari \in \pres{\mar}$; unlike a forward approach which would have calculated the reachability set and conclude by checking whether \mar was in it. In other words, it computes all the configurations that can reach $\ucs$ in any number of steps.
%
%The calculation is a fixed point algorithm that compute the increasing sequence, for the inclusion relation, of sets of markings: $(R_n)_{n \in \naturals}$, with $R_0 = \ucs$ and $R_{n+1} = \pre{R_n} \cup R_n$.
%Thus, $R_n$ is the set of markings from which there exists a sequence of at most $n$ transitions which may be fired and that cover $\ucs$.
%Because, with $\ucs$ an upward-closed set of markings, $\pre{\ucs}$ is upward-closed too%
%\footnote{This is due to the monotonicity of \acp{PN}, \todo{see for example cite\{someone\}}},
%and because the union of two upward-closed sets is an upward-closed set,
%$R_n$ is upward-closed for all $n$.
%
%\todo{summarize correctness and termination from \cite{Abdulla96}}


%
\section{Known Results on Plain \acp{PN}}
\label{sec:known-results-on-plain-pn}
%%
\subsection{A general backward algorithm $\back$}
\label{sec:a-general-backward-algorithm-back}
\label{sec:backward-algorithm}
There exists a simple way to solve the coverability problem for all the examples of \acp{WSTS} above mentioned.
This algorithm was introduced by Abdulla \lang{et al.} \citep{Abdulla96}.
It is close of the one introduced earlier in \cite{Finkel90}.
%Even if we do not see how to use it in the context of \ac{PPN}, we mention it here because it helps to grasp, by comparison with the Karp and Miller algorithm presented in the following section, where does lie the difficulties of the coverability problem.

Relying on the definition~\ref{defi:upclocovprblm} of the coverability problem, given an upward-closed set of markings $\ucs$, the algorithm computes $\pres{\ucs}$ by iterating the $\Pre$ operator.
Thus we compute a sequence $(\setm_i)_{i \geq 0}$ where $\setm_i$ is the set of markings from where $\ucs$ is reachable in $i$ or less steps.


\begin{algorithm}
  \caption{$\back$}
  \label{algo:back}

  \begin{algorithmic}
    \Require $\ucs$ the upward-closed goal set of markings
    \State $\setm \gets \ucs$
    \Repeat
      \State $\setm'\gets \setm$
      \State $\setm \gets \setm \cup \pre{\setm}$
    \Until{$\setm' = \setm$}
    \Ensure $\setm = \pres{\ucs}$
  \end{algorithmic}
\end{algorithm}


The termination and correction of the algorithm were proved in \cite{Abdulla96}.% Here are the main elements of the proofs.

The termination is ensured by the existence of a fixed point in the sequence of upward-closed set of markings $(\setm_i)_{i \geq 0}$:
\begin{gather*}
  \setm_0 = \ucs \\
  \forall i > 0 : \setm_i = \setm_{i-1} \cup \pre{\setm_{i-1}}
\end{gather*}
When this fixed point is reached, we have $\pres{\ucs}$.
If $\mari \in \pres{\ucs}$, one can conclude positively.
Otherwise, the result is negative.
We call this algorithm $\back$.

Although the set handled by the algorithm are infinite, 

\begin{lemm}[\cite{Abdulla96}]
  Given a monotonic transition system and an upward-closed set $\ucs$ of its states, $\pre{\ucs}$ is upward-closed too.
\end{lemm}

This approach is elegant and general, but often inefficient in practice.
It is well-known that a forward exploration of the state of space (\lang{i.e.}, in this context, using $\Post$ instead of $\Pre$) is usually more efficient \citep{Henzinger98}.
We now present a forward algorithm, but whose the application is restricted to \ac{PN}.


%%
\subsection{The Karp and Miller algorithm}
\label{sec:the-karp-and-miller-algorithm}
\label{sec:km}
Although it was not originally designed for this purpose, the Karp and Miller algorithm \cite{Karp69} is a classical algorithm to compute a coverability set of an initialized \ac{PN}.
More precisely, \emph{it constructs a coverability tree} and uses an acceleration function to systematically detect self-covering sequences, and thus ensures the termination.

\begin{defi}[Coverability tree]
  Given a \ac{PN} $\defPN$, a coverability tree \nameT of \namePN is a labelled tree $\defT$ where:
  \begin{itemize}
    \item $\nodes$ is the set of nodes of the tree.%, is a set of \omark of \namePN such that $\downc{\nodes} = \cover{\namePN}$.

    \item $n_0 \in \nodes$ is the root of the tree, \ie $\nexists n \in \nodes$ such that $(n, n_0) \in \edges$.

    \item $\Lab : \nodes \mapsto (\naturals \cup \{\omega\})^{\card{\places}}$ is a labelling function that associate to each node an \omark of \namePN.

    \item $\edges \subseteq \nodes \times \nodes$, the set of edges, is such that:
      \begin{itemize}
        \item with $(n_1, n_2) \in \nodes^2$, if there exists an edge $(n_1, n_2) \in \edges$ then there exists a sequence $\seqt$ of transitions of $\transitions$ such that $\lab{n_1} \fire{\seqt} \lab{n_2}$,
        \item for all node $n \in \nodes \setminus \{n_0\}$, there exists a path from the root to $n$, that is, there exists a sequence of edges of $\edges$ of the form $((n_0, n_1), (n_1, n_2), \dots, (n_{i}, n))$, $i \geq 1$, and
        \item there are no cycles, that is, there are no sequences of edges of $\edges$ of the form $((n_1, n_2), (n_2, n_3), \dots, (n_i, n_1))$.
      \end{itemize}
  \end{itemize}
  and such that $\downc{\setComp{\lab{n}}{n \in \nodes}} = \cover{\namePN}$.
\end{defi}

In addition, we will often use the following notations.
\begin{itemize}
  \item We denote by $\treepath{n}$ the path in the tree from the root to $n$, that is the sequence of nodes from $n_0$ to $n$.
    $\treepath{n}$ is called \emph{the branch to $n$} and the nodes of $\treepath{n}$ are the \emph{ancestors} of $n$.
		The \emph{depth} of $n$ is its number of ancestors minus one ($|\treepath{n}| - 1$).

    With $m \in \treepath{n}$, $\treepath[m]{n}$ is the sequence of nodes from $m$ to $n$.
    If $m \notin \treepath{n}$, $\treepath[m]{n}$ is the empty sequence.
  %$\treepath{n}$ is the sequence of nodes from the path, $\treepathe{n}$ is the sequence of edges.
  \item $\forall n \in \nodes \setminus \{n_0\}$, $\parent{n}$ is the direct ancestor, or \emph{parent}, of $n$ in $\nameT$, that is, the last but one node in $\treepath{n}$.
  %\item \removed{given a sequence of nodes, for any node $n$ of the sequence but the last one, $\child{n}$ is the next node in the sequence.}
\end{itemize}

The algorithm (see~\Cref{algo:km}) constructs the tree $\nameT$ as follows:
The root $n_0$ of the tree is labelled with \mari.
A frontier $\front$ is defined to be the set of unprocessed nodes of the tree and is initialised to $\{n_0\}$.
Then, while $\front$ is non-empty, a node $n$ is chosen from $\front$ to be processed:
(1) it is removed from $\front$, and (2) if there is no node $n'$ in $\treepath{n}$ such that $\lab{n} = \lab{n'}$, then for all \omark $\mar \in \post{\lab{n}}$, (2.1) a node labelled with $\kmacc{\mar, \treepath{n}}$ is added to the frontier and (2.2) to the tree as a child of $n$.

\begin{algorithm}
  \caption{The Karp and Miller algorithm}
  \label{algo:km}

  \begin{algorithmic}
    \State $n_0 \gets $ new Node, $\lab{n_0} \gets \mar_0$
    \State $\front \gets \{n_0\}$
    \While{$\front \neq \emptyset$}
      \State $n \gets$ pop$(\front)$ \Comment{(1)}
      \If{$\nexists n' \in \treepath{n} \text{ s.t. } \lab{n} = \lab{n'}$} \Comment{(2)}
        \ForAll{$\mar \in  \post{\lab{n}}$}
          \State $n'' \gets $ new Node, $\lab{n''} \gets \kmacc{\mar, \treepath{n}}$
          \State push$(\front, n'')$ \Comment{(2.1)}
          \State push$($children$(n_0), n'')$ \Comment{(2.2)}
        \EndFor
      \EndIf
    \EndWhile
  \end{algorithmic}
\end{algorithm}


We attach to the Karp and Miller tree $\nameT$ the mapping $\Labt : \nodes \setminus \{n_0\} \mapsto \transitions$ that gives for all the nodes of the tree (but the root) the transition used to create it at step (2) of the Karp and Miller algorithm.\\
By abuse of notation, we will note
$\labt{(n_1, n_2, ..., n_m)}$
%with $n_i, i \in \{1, ..., m\}$ some nodes of $\nameT$
with $n_i \in \nodes$
the sequence
% TODO: remove this false line break
\\
$(\labt{n_2}, ..., \labt{n_m})$.

To keep $\nodes$ finite, the Karp and Miller algorithm exploits the strong monotonicity of \acp{PN} to introduce \omarks through an \emph{acceleration function} $\KMAcc$.
This function takes a marking \mar to accelerate with a set of markings $\setm$ as a base for the acceleration and returns a marking such that:
\[
  \KMAcc(\mar,\setm)(p) =
  \begin{cases}
    \omega  &\text{if } \exists \mar' \in \setm : \mar' \prec \mar \text{ and } \mar'(p) < \mar(p) \\
    \mar(p) &\text{otherwise}
  \end{cases}
\]

The acceleration function is said to accelerate a marking if the first case holds for one or more places.
These places are said \emph{accelerated}.

The correctness and the termination of the algorithm lies on the strong monotonicity of \acp{PN}, and were proved by Karp and Miller in their work \cite{Karp69}.

To prove that the Karp and Miller Tree $\nameT$ is a coverability tree of \namePN, we show:
\begin{enumerate}
  \item that one can construct a sequence from \mari to any marking \mar “covered by the tree”, given a node $n$ such that $\mar \in \downc{\lab{n}}$, and
  \item that any marking of $\cover{\namePN}$ is “covered by at least one node of the tree”.
\end{enumerate}

Finally, we prove that $\nameT$ is finite and that the algorithm terminates.

Although very different in form, the proofs are based on those stated in \cite{Karp69}.
We take this opportunity to give the reader a firm understanding of how the algorithm works.
For a shorter proof, refer to \cite{Karp69}.

\begin{lemm}
  \label{theo:km-correctness}
  Given an initialized \ac{PN} $\defPN$ and a node $n$ of its Karp and Miller tree $\defT$,
  for any marking $\mar \in \downc{\lab{n}}$ there exists a sequence of transitions $\seqt \in \transitions^*$ and a marking $\mar'$ such that $\mari \fire{\seqt} \mar'$ and $\mar \preceq \mar'$.
\end{lemm}

\begin{proof}
  We show that there exists a function $\Tts : \naturals^{\card{\places}} \times \nodes \mapsto \transitions^* : \mar, n \rightarrow \seqt$ that, given a marking $\mar$ and a node $n$ such that $\mar \in \downc{\lab{n}}$, gives such a sequence $\seqt$.

  Before we look at $\Tts$, here is one key observation on the Karp and Miller tree:
  since the Karp and Miller algorithm never remove $\omega$s, we have
  $\forall n \in \nodes \setminus \{n_0\}$, $\op{\lab{\parent{n}}} \subseteq \op{\lab{n}}$,
  \ie all the \oplaces on a marking stay marked with an $\omega$ in all its descendants.

  The idea of the construction of $\tts{\mar}{n}$ is that for any \oplace $p_i$ of $\lab{n}$, one can extract from $\treepath{n}$ a self-covering sequence $\iscs{i}$ strictly increasing in $p_i$:
  $\effect{\iscs{i}}(p_i) > 0$ and $\effect{\iscs{i}}(p) \geq 0$ for all $p \in \places$.
  If $\lab{n}$ does not contain any \oplace, then the sequence given by $\labt{\treepath{n}}$ is an actual execution of the system (since no acceleration have occurred on the branch) and we define $\tts{\mar}{n}$ to be $\labt{\treepath{n}}$.
  Further down in the branch we may encounter the first node with some \oplaces.
  These $\omega$s reveal the presence of increasing self-covering sequences that we can repeat to cover the desired marking.
  If new $\omega$s appear later, the  increasing self-covering sequence is not obvious, since it may be that the effect of the sequence implicitly detected by the acceleration is negative for some places already accelerated before.
  However, this negative effect on this place may be counterbalanced by the repetition of the sequence that allowed this previous acceleration.
  We now detail the different cases.
  %We now show it in greater details.

  %Without loss of generality, we assume that $\mari$ does not map any place to $\omega$, and thus $\tts{\mar}{n_0} = ()$.
  Since $\mari$ does not have any \oplace, $\tts{\mar}{n_0} = ()$.

  %We now recursively show that $\tts{\mar}{n}$ exists for all such $(\mar, n)$ pair where $n \neq n_0$.
  We voluntary adopt a very prolific style here, in order to allow an easy and in-depth understanding of how the algorithm work.

  So assume $\mar$ and $n$ given and $\mar \in \downc{\lab{n}}$.
  When we deal with a given node, we'll need a specific order on the places, so we adopt the following conventions:
  \begin{itemize}
    \item with $p_i \in \op{\lab{n}}$,
      \begin{itemize}
        \item $n^\omega_i$ denotes the first node of $\treepath{n}$ such that $\lab{n_i}(p_i) = \omega$, and
        \item $n_i$ denotes the node used by $\KMAcc$ to accelerate $p_i$ in $\lab{n^\omega_i}$, \ie, the node of $\treepath{n^\omega_i}$ such that $\lab{n_i} \prec \lab{n^\omega_i} \text{ and } \lab{n_i}(p_i) \neq \omega$.
          If there are many such nodes, $n_i$ is the last one in $\treepath{n^\omega_i}$.
      \end{itemize}
    \item with $i$ and $j \in \{1, …, \card{\places}\}$, we have:
      \begin{itemize}
        \item $i < j$ whenever $p_i \in \op{\lab{n}} \wedge p_j \notin \op{\lab{n}}$:
          the \oplaces are before the \noplaces,
        \item $i < j$ whenever $p_i$ and $p_j \in \op{\lab{n}}$ and $\treepath{n^\omega_i}$ is a prefix of $\treepath{n^\omega_j}$:
          the places accelerated first come first,
        \item $i < j$ whenever $p_i$ and $p_j \in \op{\lab{n}}$, $n^\omega_i = n^\omega_j$ and $\treepath{n_i}$ is a prefix of $\treepath{n_j}$:
          the places accelerated thanks to nodes nearer of the root come first,
        \item otherwise, the order between $i$ and $j$ may be fixed arbitrarily.
      \end{itemize}
    \item $\ab{n} = \{1, ..., I\}$ is the possibly empty set of the indices of the places that were accelerated ``before'' $n$: $i \in \ab{n}$ iff $p_i \in \op{\lab{\parent{n}}}$,
    \item $\na{n} = \{I+1, ..., J\}$ is the possibly empty set of the indices of the places accelerated ``at'' $n$: $j \in \na{n}$ iff $p_j \in \op{\lab{n}}$ and $p_j \notin \ab{n}$.%$p_j \notin \op{\lab{\parent{n}}}$.
  \end{itemize}

  In the rest of the proof, we will consistenlty use $I$ and $J$ as the largest indices of $\ab{n}$ and $\na{n}$.

  Thus, with $n$ clear from the context this ordering fixed, $n^\omega_1$ corresponds to the first node of $\treepath{n}$ where an acceleration occured.

  \todo{example}

  Moreover, $\forall i \in \ab{n} \cup \na{n}$, we will denote by $\iscs{i}$ the self-covering sequence increasing in $p_i$ extracted from $\treepath{n}$.
  Actually, the work is twofold: when dealing with a node $n$ we have to determine $\iscs{j}$ for all $j \in \na{n}$, and how many times at least should we repeat $\iscs{i}$ for all $i \in \ab{n} \cup \na{n}$, \lang{i.e}, we are looking for a sequence of the form:
  \[ \tts{\mar}{n} = \labt{\treepath{n_1}} + k_1 \cdot \iscs{1} + \dots + k_J \cdot \iscs{J} + \labt{\treepath[n^\omega_J]{n}} \]

  Recall that
  $\forall n \in \nodes \setminus \{n_0\}$, $\op{\lab{\parent{n}}} \subseteq \op{\lab{n}}$,
  (all the \oplaces on a marking stay marked with an $\omega$ in all its descendants).
  We use this property to realise an induction on the nodes of the branch.
  However, we do not state it explicitly for the sake of clarity.

  First consider that $\ab{n} = \emptyset$.
  If $\na{n} = \emptyset$, there was no acceleration on $\treepath{n}$ and by construction of $\nameT$, $\mari \fire{\labt{\treepath{n}}} \lab{n}$.
  So we define $\tts{\mar}{n}$ to be $\labt{\treepath{n}}$ whenever $\ab{n} = \emptyset$ and $\na{n} = \emptyset$.\\
  This is a special case of the one where $\na{n}$ may be not empty.
  In that case, $\mar''$ given by $\mari \fire{\labt{\treepath{n}}} \mar''$ agrees with $\lab{n}$ in its \noplaces.
  To reach $\mar'$ that covers \mar in all its places, we explicitly repeat the increasing self-covering sequences implicitly detected by the acceleration function.
  By construction of $\nameT$ we know that $\ab{n} = \emptyset$ (thus~$I = 0$), and, for all $j \in \na{n} = \{I+1 = 1, ..., J\}$, $\lab{n_j} \prec \mar''$.
  Therefore:
  \begin{align*}
    &\makebox[0pt][l]{\,\(\labt{\treepath[n_j]{n^\omega_j}}\text{ is enabled in }\mar''\text{ (by strong monotonicity), }\)} \\
    &\text{For all the places, the effect is non-negative:}\\
    &\effect{\labt{\treepath[n_j]{n^\omega_j}}}(p) \geq 0 &&\text{ for all }j \in \na{n}\text{ and }p \in \places \text{, and} \\
    &\text{For all the accelerated places, the effect is positive:}\\
    &\effect{\labt{\treepath[n_j]{n^\omega_j}}}(p_j) > 0  &&\text{ for all }j \in \na{n} \text{.}
  \end{align*}
  (Note that $n^\omega_j = n$.)
  Thus, for all such $j$, $\iscs{j} = \labt{\treepath[n_j]{n^\omega_j}}$ is the increasing self-covering sequence we were looking for.
  Moreover, % whenever $\ab{n} = \emptyset$,
  %$\seqt_j = \labt{\treepath[n_j]{n}}$ for all $j \in \na{n}$
  %and
  there exists some naturals $k_j$ such that we can define $\tts{\mar}{n}$ to be:
  \[ \tts{\mar}{n} = \labt{\treepath{n_1}} + k_1 \cdot \iscs{1} + \dots + k_J \cdot \iscs{J} \quad\text{ if }\ab{n} = \emptyset \]
  Actually, one can easily compute such $k_j$ to obtain the shortest sequence of this form.
  With $\seqt_j = \labt{\treepath{n_1}} + k_1 \cdot \iscs{1} + … + k_j \cdot \iscs{j}$, we have:
  \[
    k_j =
    \begin{cases}
      \left\lceil
        \frac{\mar(p_j) - \mari(p_j) - \effect{\labt{\treepath{n_1}}}(p_j)}
             {\effect{\iscs{j}}}
      \right\rceil
      &\text{ if } j = 1\\
      \\
      \max\left(0,
        \left\lceil
          \frac{\mar(p_j) - \mari(p_j) - \effect{\seqt_{j-1}}(p_j)}
              {\effect{\iscs{j}}}
        \right\rceil
      \right)
      &\text{ if } j \in \{2, ..., J\}
    \end{cases}
  \]

  The chosen order ensures that with $j_1 \leq j_2$ we have $\card{\iscs{j_1}} \geq \card{\iscs{j_2}}$ and so any other order would give a sequence longer or as long as the one computed here.

  Now consider the case where $\ab{n} \neq \emptyset$.
  %By definition, a node in this case have at least one ancestor lying in the base case.
  Once again, let us focus first on the simpler case where $\na{n} = \emptyset$.
  Here one can easily compute the $\preceq$-lowest marking $\mar''$ of $\downc{\lab{\parent{n}}}$ such that $\mar'' \fire{\labt{n}} \mar'$ (with $\mar \preceq \mar'$)
  and that agrees with $\lab{\parent{n}}$ for its \noplaces. It is given by:
  \[
    \mar''(p) = \begin{cases}
      \lab{\parent{n}}(p)
        &\text{ if } p \notin \op{\lab{\parent{n}}} \\
      \max(\mar(p) - \effect{\labt{n}}(p), \inw[\labt{n}][p])
        &\text{ otherwise}
    \end{cases}
  \]

  Indeed, %(1)
  $\labt{n}$ is enabled in $\mar''$:
  for the \noplaces $p$ of $\lab{\parent{n}}$, the Karp and Miller algorithm ensures that $I_{\labt{n}}(p) \leq \mar''(p)$;
  and for the \oplaces $p$ of $\lab{\parent{n}}$, the chosen value is at least $I_{\labt{n}}(p)$.\\
  Moreover, %(2)
  $\mar \preceq \mar'$:
  for the \noplaces $p$ of $\lab{\parent{n}}$, we have $\mar'(p) = \mar''(p) + \effect{\labt{n}}(p) = \lab{n}(p) \geq \mar(p)$ by choice of $n$ and \mar;
  and for the \oplaces $p$ of $\lab{\parent{n}}$, we have $\mar'(p) = \mar''(p) + \effect{\labt{n}}(p) \geq \mar(p)$ by construction of $\mar''$.\\
  Thus, we define
  $\tts{n}{\mar}$ to be $\tts{\parent{n}}{\mar''} + (\labt{n})$
  whenever
  $\na{n} = \emptyset \wedge \ab{n} \neq \emptyset$.
  Note that $\tts{\parent{n}}{\mar''}$ may effectively be computed since we are implicitly defining $\tts$ as an induction on the nodes of the branch.

  In the general case, $\na{n}$ may be not empty.
  There a difficulty arises from the fact that $\labt{\treepath[n_j]{n^\omega_j}}, j \in \na{n}$ may not be a self-covering sequence since its effect may be negative for some places $p_i$ (with $i \in \ab{n}$).
  This difficulty is solved by integrating enough repetitions of $\iscs{i}$ along with $\labt{\treepath[n_j]{n^\omega_j}}$ in $\iscs{j}$ to counterbalance it.

  \todo{example}

  Here are two ways to show that it can be done.
  First, it may be easier to convince oneself of this by noting that, for all such $i$, there is a marking $\mar''$ such that $\tts{\mar''}{n^\omega_i}$ has enough tokens in $p_i$ to undo the negative effect of $\labt{\treepath[n_j]{n^\omega_j}}$ in this place. But the calculation of $\mar''$ is technical.\\
   However, the computation of an actual self-covering sequence increasing in $p_j$, to use instead of $\treepath[n_j]{ n^\omega_j}$, is easy.
   %With only one $i$, it is given by:
   % \todo{à corriger}
   %\[
   % \iscs{j} = \labt{\treepath[n_j]{n_i} + l_{j,i} \cdot \treepath[n_i]{n^\omega_i} + \treepath[n^\omega_i]{n^\omega_j}}
   % \text{ with } l_{j,i} =
   % \left\lceil \frac{-\effect{\labt{\treepath[n_j]{n^\omega_j}}}(p_i)}
   %                  { \effect{\labt{\treepath[n_i]{n^\omega_i}}}(p_i)} \right\rceil \text{.}
   %\]

   %In the case where there are several different $i$'s, the sequence can be obtained by considering the $i$'s one by one.
   Starting from $\treepath[n_j]{n^\omega_j}$, for each $i \in \ab{n}$, we compute a path that undo a potential negative effect in $p_i$.
   Then, the sequence we are looking for is $\iscs{j} = \labt{\seqt_{j,I}}$ given by: $\forall i \in \{0\} \cup \ab{n} = \{0, ..., I\}$
  \[
    \seqt_{j,i} =
    \begin{cases}
      \seqt_{j,0} = \treepath[n_j]{n^\omega_j}
        &\text{ if } i = 0 \\
      \seqt_{j,i} = \seqt_{j,i-1}
        &\text{ if } i > 0 \text{ and } \effect{\labt{\seqt_{j,i-1}}}(p_i) \geq 0 \\
      \seqt_{j,i} = \slice[\seqt_{j,i-1}]{}{u} + l_{j,i} \cdot \iscs{i} + \slice[\seqt_{j,i-1}]{v}{}
        &\text{ otherwise}
    \end{cases}
  \]
  where $u = \elem{\iscs{i}}{1}$ is the first element of $\iscs{i}$ and $v = \elem{\iscs{i}}{\card{\iscs{i}}}$ is its last element,
  and
  \[
    l_{j,i} = 
    \left\lceil \frac{-\effect{\labt{\seqt_{j,i-1}}}(p_i)}
                     { \effect{\labt{\iscs{i}}}(p_i)} \right\rceil \text{.}
  \]
  \todo{example}

  This provides us with a mean to compute an actual increasing self-covering sequence for every accelerated place $p_j, j \in \na{n}$.

  Thus, as before we define $\Tts$ to be
  \[ \tts{\mar}{n} = \labt{\treepath{n_1}} + k_1 \cdot \iscs{1} + \dots + k_J \cdot \iscs{J} \]
  %with $m = n_i : \min_{i \in \range{1}{J}}(\card{\treepath{n_i}})$ is the first node of the branch used in a increasing self-covering sequence $\iscs{}$.
  Here however, since $j_1 < j_2$ does not implies $\card{\iscs{j_1}} \geq \card{\iscs{j_2}}$ anymore, another order may lead to shorter sequences.



  To sum up, given a \ac{PN}, its Karp and Miller tree, a node $n$ from it and a marking $\mar \in \downc{n}$, we have
  \[
    \mari \fire{\tts{\mar}{n}} \mar', \mar \preceq \mar'
  \]
  with
  \[
    \tts{\mar}{n} = \labt{\treepath{n_1}} + k_1 \cdot \iscs{1} + \dots + k_J \cdot \iscs{J} + \labt{\treepath[n^\omega_J]{n}}
  \]
  where each increasing self-covering sequences $\iscs{j}$ is computed in the context of the first node of the branch where it is accelerated: $n^\omega_j$;
  where the $k_j$ are computed in the context of the last node where an acceleration occurred $n^\omega_J$, considering a marking $\mar''$ such that
  \( \mar'' \fire{\labt{\treepath[n^\omega_J]{n}}} \mar' \).
\end{proof}

\begin{lemm}
  Given an initialized \ac{PN} $\defPN$ and any of its coverable marking $\mar \in \cover{\namePN}$,
  there exists a node $n$ of its Karp and Miller tree $\defT$, such that $\mar \preceq \lab{n}$.
\end{lemm}

\begin{proof}
  Let \mar be a marking of $\cover{\namePN}$ and $\defSeqt{0}{n-1}$ a sequence of transitions that witnesses it: \(\mari \fire{\seqt} \mar', \mar \preceq \mar'\).
  Let $\bodySeqm{1}{n}$ be the markings : \(\mari \fire{t_0} \mar_1 \fire{t_1} \dots \fire{t_{n-1}} \mar_n = \mar'\).

  Then, apply the following operations on this sequence of markings:
  while there exists a pair of markings \((\mar_i, \mar_j), i < j, \mar_i \preceq \mar_j\) in the sequence:
  \begin{itemize}
    \item if $\mar_i = \mar_j$, remove all the nodes after $\mar_j$,
    \item otherwise, for each place $p$ such that $\mar_i(p) < \mar_j(p)$, replace $\mar_k(p)$ by $\omega$ for all the markings $\mar_k, k \geq j$ from $\mar_j$ (included) to the end of the sequence.
  \end{itemize}

  At the end, the sequence obtained is the sequence of labels in some path $\treepath{n}$ from the $n_0$ to $n$ and the final label cover $\mar$.
\end{proof}

\begin{lemm}
  \label{theo:km-tree-finiteness}
  For any \ac{PN}, its Karp and Miller tree is finite.
\end{lemm}

Based on the Kőnig's Infinity lemma, we will prove that, to be infinite, a tree needs either an infinite branch or a node with infinitely many children.
From there we will proof the finiteness of the Karp and Miller tree.

\begin{lemm}[Kőnig's Infinity lemma \cite{konig1927schlussweise} applied to trees]
	\label{theo:konig}
	Let $\nameT$ be a tree of infinite depth where the number of nodes at each depth is finite.
	Then $\nameT$ contains an infinite branch.
\end{lemm}

\begin{lemm}
  \label{theo:false-konig}
  Every infinite tree has a node with infinitely many children or contains an infinite branch.
\end{lemm}

\begin{proof}%[\proofname\ of \Cref{theo:false-konig}]
  Let $\nameT$ be an infinite tree where every node has a finite number of children.
  For every $c \in naturals$, let $\set_c$ be the set of nodes at depth $c$.
	Induction on $c$ shows that the sets $\set_c$ are finite.
	Thus, by \Cref{theo:konig}, $\nameT$ contains an infinite branch.
\end{proof}

\begin{proof}[\proofname\ of \Cref{theo:km-tree-finiteness}]
  Suppose that the Karp and Miller tree is infinite
  and recall that, by definition, the set of transitions of a PN is finite.
	Then, by contruction, every node of the Karp and Miller tree has a finite number of children.
  Thus, the tree has an inifinite branch.
  This contradicts \Cref{lemm:wqo}.
\end{proof}

\todo{example and explicits why it contradicts lemma 1}
 
The Karp and Miller tree has a lot of convenient properties that allow, among other, to answer the coverability problem as well as the simultaneous place unboundedness problem.

\begin{defi}[Simultaneous place unboundedness]
  Let $\defPN$ be a marked \ac{PN}.
  Given a set of places $\setp \subseteq \places$, \namePN is said $\setp$-simultaneous unbounded if and only if for any $c \in \naturals$ there exists \mar such that $\mari \fire* \mar$ and $\forall p \in \setp : \mar(p) \geq c$.
\end{defi}

Furthermore, the Karp and Miller algorithm can easily be adapted to some parametric problems \cite{David17}, as we will show in \autoref{sec:known-results-on-ppn}.

However, this tree, although finite, is often much larger than the minimal coverability set, and cannot be constructed in reasonable time.
As a consequence, many improvements were proposed, as well as other algorithms with different approaches.


%%
\subsection{Geeraerts method}
\label{geeraerts-method}
\label{sec:eff}
This is usually called ``an efficient computation method of the coverability set of \acp{PN}''.
It was proposed in \cite{Geeraerts07thesis, Geeraerts07} as another approach to the computation of the coverability set.
It is not based on the Karp and Miller algorithm and is not an alternative to it in the sense that it does not allow to answer the same set of questions than the Karp and Miller tree answers.
However, this technique solves the coverability problem more efficiently in practice.

As in the Karp and Miller algorithm, an acceleration function exploits the strong monotonicity of \acp{PN} to allow termination.
But here, the acceleration of a marking is performed with only one marking as the base (instead of a set of marking).

To choose the base to use, the algorithm works on pair of \omarks.
These pairs allow to record a relationship between the markings.
More precisely, the algorithm constructs a pair of \omarks $(\mar_1, \mar_2)$ only if $\downc{\mar_2} \subseteq \downc{\posts{\mar_1}}$.

To reduce the size of the set of pairs of \omarks, only the pairs where the difference (as defined below) between $\mar_1$ and $\mar_2$ is maximal are kept.
This will be the purpose of the order $\sqsubseteq$ we will define and it is justified by the intuitive idea that two more distant markings produce larger accelerations.
Therefore, if the algorithm builds a pair $(\mar_1, \mar_2)$, it can forget about any other ($\sqsubseteq$-comparable) pair whose the elements are closer because
\begin{itemize}
  \item by monotonicity, all potential successors of the elements of this pair will be covered by successors of $\mar_1$ or $\mar_2$, and
  \item any acceleration that can be created from this pair is covered by an acceleration one can build from $(\mar_1, \mar_2)$.
\end{itemize}

To describe the algorithm more formally, we will need the following definitions:

Given a pair of \omarks $(\mar_1, \mar_2)$, we define:
\begin{itemize}
  \item $\Postb((\mar_1, \mar_2)) = \{(\mar_1, \mar'), (\mar_2, \mar') \mid \mar' \in \post{\mar_2}\}$,
  \item and, with $\mar_1 \prec \mar_2$, $\Accelb(\mar_1, \mar_2) = \{(\mar_2, \kmacc{\{\mar_1\}, \mar_2})\}$.
    $\Accelb(\mar_1, \mar_2)$ is not defined whenever $\mar_1 \nprec \mar_2$,
\end{itemize}

With $R$ a set of pair of markings, we define:
\begin{itemize}
  \item $\Postb(R) = \bigcup_{(\mar_1, \mar_2) \in R} \Postb((\mar_1, \mar_2))$
  \item $\Accelb(R) = \bigcup_{(\mar_1, \mar_2) \in R}^{\mar_1 \prec \mar_2} \Accelb((\mar_1, \mar_2))$
  \item $\Flatten(R) = \{\mar \mid \exists \mar' : (\mar', \mar) \in R\}$
\end{itemize}

The computation of the coverability set of the marked \ac{PN} $\defPN$ lies on the sequence $\CovSeq(\namePN) = (V_i)_{i \geq 0}$ of pair of \omarks, where, for all marked \ac{PN} $\namePN$ we have:
\begin{gather*}
  V_0 = \{(\mari, \mari)\} \text{ and } \\
  \forall i \geq 1 : V_i = V_{i-1} \cup \Postb(V_{i-1}) \cup \Accelb(V_{i-1})
\end{gather*}

One can show that,
first, for all node $n$ of the Karp and Miller tree, there exists a value $k \geq 0$ of $i$ such that $\lab{n} \in \Flatten(V_k)$,
second, all the markings produced by $\Postb$ and $\Accelb$ are in the coverability set of \namePN.
\todo{Indeed...}

These two results lead to the following lemma:
\begin{lemm}[\cite{Geeraerts07}]
  Given a marked \ac{PN} \namePN such that $\CovSeq(\namePN) = (V_i)_{i \geq 0}$,
  there exists $k \geq 0$ such that for all $l \in \{0, ..., k-1\}$ we have that $\downc{\Flatten(V_l)} \subset \downc{\Flatten(V_{l+1})}$
  and for all $l \geq k : \downc{\Flatten(V_l)} = \cover{\namePN}$.
\end{lemm}

Thus, the algorithm idea is to compute $\CovSeq$ until it stabilizes, \ie to the lowest $l$ such that $\downc{\Flatten(V_l)} = \downc{\Flatten(V_{l-1})}$ and to return $\downc{\Flatten(V_l)}$.

To perform it efficiently, one can use a well-chosen order $\sqsubseteq$ on the pair of markings.
This order intents to sort the pairs of markings according to the distance in between, and is used to keep only ``the more distant'' pairs.
Let us denote by $\ominus$ the componentwise difference between two markings and to extend it to \omarks.
Formally, given two \omarks $\mar_1$ and $\mar_2$ on a set of places $\places$, $(\mar_1 \ominus \mar_2)(p)$ is defined for all $p \in \places$ as:
\[
  \begin{cases}
    \omega & \text{ whenever } \mar_1(p) = \omega \\
    -\omega & \text{ whenever } \mar_2(p) = \omega \text{ and } \mar_1(p) \neq \omega \\
    \mar_1(p) - \mar_2(p) & \text{ otherwise}
  \end{cases}
\]

Now we can define $\sqsubseteq$.
Given two pairs $(\mar_1, \mar_2)$ and $(\marp_1, \marp_2)$ of \omarks over a set of places $\places$:
\[
  (\mar_1, \mar_2) \sqsubseteq (\marp_1, \marp_2) \Leftrightarrow
  \begin{cases}
    & \mar_1 \preceq \marp_1 \\
    \wedge & \mar_2 \preceq \marp_2 \\
    \wedge & \forall p \in \places : (\mar_2 \ominus \mar_1)(p) \leq (\marp_2 \ominus \marp_1)(p)
  \end{cases}
\]

For a set of pair of \omarks $R$, $\maxs{R} = \{ r \in R \mid \nexists r' \in R, r \sqsubseteq r'\}$ is the set of highest \omark of $R$ with respect to $\sqsubseteq$.

This order has properties \citep{Geeraerts07} that allows to keep the sets of markings of $\CovSeq$ small.
Thus, one can compute $\cover{\namePN}$ of a \ac{PN} $\defPN$ by computing the sequence $(V_i)_{i \geq 0}$ defined below until $\downc{\Flatten(V_i)} = \downc{\Flatten(V_{i-1})}$.
\begin{gather*}
  V_0 = \{(\mari, \mari)\} \text{ and } \\
  \forall i \geq 1 : V_i = \maxs{V_{i-1} \cup \Postb(V_{i-1}) \cup \Accelb(V_{i-1})}
\end{gather*}

At the end, we have that $\downc{\Flatten(V_i)} = \cover{\namePN}$.

The correction and termination of the algorithm as well as useful properties of $\sqsubseteq$ can be found in \cite{Geeraerts07, Ganty09}.


%%
\subsection{The \ac{EEC} algorithm}
\label{the-ecc-algorithm}
\label{sec:eec}
\ac{EEC}, introduced in \cite{Geeraerts07thesis, Geeraerts06}, is an iterative algorithm that allows, among other, to solve the coverability problem for \ac{PN}.
We present it restricted to this context, but it may be used for a wide range of well-structured transitions systems, which \acp{PN} is part of, because it relies only on the monotonicity, and not on the strong monotonicity, of these models.

The idea is to compute and refine simultaneously an over- and an under-approximation of the covering set of the \ac{PN} until one or the other allows to conclude.

The under-approximation is computed as follows:
We define $(C_i)_{i \geq 0}$ to be the sequence of finite set of markings holding no more than $i$ tokens in each place (plus \mari):
\[
  \forall i \in \naturals : C_i = \{0, ..., i\}^{\card{\places}} \cup \{\mari\}
\]
At step $i$, the algorithm computes $\Sous(\namePN, C_i)$ defined as the graph $\langle C_i, \mari, \sousrel \rangle$ which is the transition system induced by the \ac{PN} \namePN restricted to the markings of $C_i$, \lang{i.e.} $(\mar_1, \mar_2) \in \sousrel$ if, and only if, $\mar_1 \rightarrow \mar_2$.
The under-approximation sought is the set of markings reachable through $\sousrel$ from \mari and is denoted $\R(\Sous(\namePN, C_i))$.

At step $i$, the algorithm also uses $L_i$ from the sequence $(L_i)_{i \geq 0}$ of finite set of \omarks such that $L_i = \{0, ..., i, \omega \}^{\card{\places}} \cup \{\mari\}$.
That is, $L_i$ contains all the markings with at most $i$ tokens in any place, or $\omega$ (plus \mari).
This set is used to construct the graph $\Sur(\namePN, C_i)$ defined as the graph $\langle L_i, \mari, \surrel \rangle$ where $(\mar_1, \mar_2) \in \surrel$ if, and only if:
\begin{itemize}
  \item either $\mar_1 \rightarrow \mar_2$,
  \item either $\mar_1 \rightarrow \marp_2, \marp_2 \notin L_i, \marp_2 \preceq \mar_2, \text{ and } \nexists \marpp_2 \in L_i \text{ such that } \marp_2 \prec \marpp_2 \prec \mar_2$.
\end{itemize}
In other words: if $\mar_2 \notin L_i$, it is replaced by the lowest marking of $L_i$ which over-approximate it.
Note that this is an \omark which exists and is unique. \todo{Indeed...}
Then the over-approximation is the set of markings of $L_i$ reachable through $\surrel$ from \mari. It is denoted $\R(\Sur(\namePN, L_i))$.

We can say that they are under- and over-approximations thanks to the following lemmata:
\begin{lemm}[Under-approximation \cite{Ganty09}]
  For all \ac{PN} $\defPN$, for all upward-closed set $\ucs \subseteq \naturals^{\card{\places}}$, and for all $i \geq 0: \R(\Sous(\namePN,C_i)) \cap \ucs \neq \emptyset \Rightarrow \posts{\mari} \cap \ucs \neq \emptyset$.
\end{lemm}
\begin{lemm}[Over-approximation \cite{Ganty09}]
  For all \ac{PN} $\defPN$, for all upward-closed set $\ucs \subseteq \naturals^{\card{\places}}$, and for all $i \geq 0: \downc{\R(\Sur(\namePN,L_i))} \cap \ucs = \emptyset \Rightarrow \posts{\mari} \cap \ucs = \emptyset$.
\end{lemm}

One can prove that one of the conditions mentioned in the lemmata will eventually happen.
This ensures the termination of the algorithm.

Indeed, let $\setm$ be the set of markings we want to cover and let $\ucs$ be $\upc{\setm}$.
If $\ucs$ is reachable, we will eventually get a $C_i$ that contains all the markings of a path from $\mari$ to $\ucs$.
As this path will be present in $\Sous(\namePN, C_i)$, we will have that $\R(\Sous(\namePN,C_i)) \cap \ucs \neq \emptyset$.\\
Symmetrically, let $j$ be such that $L_j$ contains the maximal elements of the covering set of \namePN.
Such a $j$ exists, and we have that $\downc{\R(\Sur(\namePN,L_j))} = \cover{\namePN}$.
Thus, we know that, for a negative instance of the problem, $\downc{\R(\Sur(\namePN,L_i))} \cap \ucs = \emptyset$ will eventually happen for an $i \leq j$.


\removed{A backward algorithm \citep{Finkel90, Abdulla96}}

%
\section{Known Results on \acp{PPN}}
\label{sec:known-results-on-ppn}
By reduction from the halting problem as well as the counter boundedness problem, \cite{David17} has shown that \Ucov and \Ecov are undecidable on \ac{PPN}.
This motivates the introduction of two natural subclasses of \acp{PPN} where parametric coverability problems are decidable.

Namely,
PreT-PPNs are \acp{PPN} where parameters are used only in $\matI$;
PostT-PPNs are \acp{PPN} where parameters are used only in $\matO$.
\cite{David17} provides an adaptation of the Karp and Miller Algorithm to solve the \Ucov problem on PreT-\acp{PPN}, and another to solve the \Ecov problem on PostT-\acp{PPN}.

% vim: set spell spelllang=en :


\chapter{Contributions}
\label{sec:contributions}
In this part we will study the possibilities to adapt the existing results for plain \ac{PN} to the \ac{PPN} model.

% vim: set spell spelllang=en :


\section{Parametric coverability problems and specific valuations}
\label{sec:parametric-coverability-problems-and-specific-valuations}
We present four theorem that basically come from the fact that
higher valuations on output arcs and lower valuations on input arcs lead to greater markings and, to the contrary,
lower valuations on output arcs and higher valuations on input arcs restrict the covering set $\Post^*(\mari)$ set.

First, when the parameters are restricted to the output arcs, a higher valuation leads to markings that are greater and thus more permissive (\lang{i.e.} that enable more transitions).
Going further, we can state the following theorem:
\begin{theo}
  \label{theo:post-star-val}
  Given a PostT-\ac{PPN} \SPTPm and an upward-closed set $U$ of markings of \PPN, \[\EcovOp(\PPN, U) = \top \Leftrightarrow \covOp(v_*(\PPN), U) = \top\] where $v_*$ is the *-valuation that maps every parameter to $*$.
\end{theo}

This was stated in \cite{David17} without a formal proof.
$\covOp(v_*(\PPN), U) \Rightarrow \EcovOp(\PPN, U)$ is trivial.
We therefore provide a proof for the other direction.

\todo{notation: v of a parameter and v of a PPN}

\begin{proof}
  Let $v$ be a *-valuation of $\PPN$ such that $\covOp(v(\PPN), U) = \top$.
  If $v$ is $v_*$, we are done.

  If $v$ is not $v_*$, let $\sigma$ be a sequence of transitions that witnesses the property above: with $\mar \in U$, we have $\mari \fire{\sigma}_v \mar$.
  Since that the n'umber of transitions is finite and that parameters are restricted to output arcs, it is easy to see that any other valuation $v'$ that maps each parameter $p$ to a value greater or equal to $v(p)$ allows $\sigma$ to lead to a marking $\mar_1$ greater than $\mar$ according to $\preceq$.
  In particular, this implies that $\mari \fire{\sigma}_{v_*} \mar_2$ with $\mar \preceq \mar_2$.
  Thus, since $\mar \in U$ and $U$ is upward-closed, we have $\mar_2 \in U$.
\end{proof}

Second, on PostT-\acp{PN}, a lower valuation leads to markings that are lower and thus less permissive.
This leads us to this theorem:
\begin{theo}
  \label{theo:post-zero-val}
  Given a PostT-\ac{PPN} \SPTPm and an upward-closed set $U$ of markings of \PPN, \[\UcovOp(\PPN, U) = \top \Leftrightarrow \covOp(v_0(\PPN), U) = \top\] where $v_0$ is the valuation that maps every parameter to $0$.
\end{theo}

\(\UcovOp(\PPN, U) = \top \Rightarrow \covOp(v_0(\PPN), U) = \top\) is trivial.
The reasoning to prove the other direction is similar to the proof given for the \autoref{theo:post-star-val}.

\begin{proof}
  Let $\sigma$ be such that $\mari \fire{\sigma}_{v_0} \mar$ with $\mar \in U$.
  We prove that for any valuation $v$, $\mari \fire{\sigma}_v \marp$ with $\marp \in U$.

  Because the parameters are bound to the output arcs and since the codomain of a valuation is $\mathbb{N}$, for any transition $t$ and any marking $\mar_1$ of $\PPN$, $\mar_1 \fire{\sigma}_{v_0} \mar_2$ implies that:
  \begin{enumerate}
    \item $t$ is enabled in $\mar_1$ in any valuation $v$: $\mar_1 \fire{t}_v \mar_3$ (because the firing condition is left unchanged), and
    \item $t$ leads to a greater marking according to $\preceq$: $\mar_2 \preceq \mar_3$.
  \end{enumerate}

  By monotonicity of \acp{PN}, this apply to a sequence of transition too.
  Thus, $\mari \fire{\sigma}_v \marp$.
  And since $U$ is upward-closed, $\marp \in U$.
\end{proof}

On the other hand, the values of the input arcs define not only the number of tokens that are removed but also the firing condition and thus the increase in these values restricts the set of firable transitions and ultimately the covering set.

\begin{theo}
  \label{theo:pre-star-val}
  Given a PreT-\ac{PPN} \SPTPm and an upward-closed set $U$ of markings of \PPN, \[\UcovOp(\PPN, U) = \top \Leftrightarrow \covOp(v_*(\PPN), U) = \top\] where $v_*$ is the *-valuation that maps every parameter to $*$.
\end{theo}

\begin{proof}
  $\UcovOp(\PPN, U) \Rightarrow \covOp(v_*(\PPN), U)$ is trivial.

  To prove that $\covOp(v_*(\PPN), U) \Rightarrow \UcovOp(\PPN, U)$, let $\sigma$ be a sequence of transitions such that $\mari \fire{\sigma}_{v_*} \mar_*$ with $\mar_* \in U$.
  Let $t$ be the first transition of $\sigma$.
  Since $\forall c \in \mathbb{N}, c < *$, we see that:
  \begin{itemize}
    \item for any valuation $v$, $t$ is enabled in $\mari$: $\mari \fire{t}_{v}$, and
    \item for any valuation $v$, any marking $\mar', \mar'' \text{ and } \mar''_*$ and any transition $t'$ such that $\mar' \fire{t'}_{v_*} \mar''_*$ and $\mar' \fire{t'}_v \mar''$ we have: $\mar''_* \preceq \mar''$.
  \end{itemize}

  Thus, by monotonicity of \acp{PN}, $\sigma$ is enabled in \mari for any valuation $v$ and, with $\mari \fire{\sigma}_v \mar$, we have $\mar_* \preceq \mar$.
  And since $U$ is upward-closed, $\mar \in U$.
\end{proof}

Finally, regarding values of input arcs, the smallest values are the best to reach high markings:

\begin{theo}
  \label{theo:pre-zero-val}
  Given a PreT-\ac{PPN} \SPTPm and an upward-closed set $U$ of markings of \PPN, \[\EcovOp(\PPN, U) = \top \Leftrightarrow \covOp(v_0(\PPN), U) = \top\] where $v_0$ is the valuation that maps every parameter to $0$.
\end{theo}

\begin{proof}
  $\covOp(v_0(\PPN), U) \Rightarrow \EcovOp(\PPN, U)$ is trivial.

  To prove that $\EcovOp(\PPN, U) \Rightarrow \covOp(v_0(\PPN), U)$, let $v$ be a valuation that allows to cover $U$ and $\sigma$ be a sequence of transitions such that $\mari \fire{\sigma}_v \mar$ with $\mar \in U$.
  Note that:
  \begin{itemize}
    \item all transition of \PPN enabled in a marking under any valuation are enabled in this marking under the valuation $v_0$, and
    \item for any valuation $v'$, any marking $\mar', \mar'' \text{ and } \mar''_{v_0}$ and any transition $t'$ such that $\mar' \fire{t'}_{v_0} \mar''_{v_0}$ and $\mar' \fire{t'}_{v'} \mar''$ we have: $\mar'' \preceq \mar''_{v_0}$.
  \end{itemize}

  Thus, using these observations on $\sigma$, and by monotonicity of \acp{PN}, we see that $\sigma$ is enabled in \mari under $v_0$ and, with $\mari \fire{\sigma}_{v_0} \mar_{v_0}$, we have $\mar \preceq \mar_{v_0}$.
  Since $U$ is upward-closed, $\mar \in U$.
\end{proof}

These different results in hand, it will be easier to give the following adaptations of known algorithms to the parametric coverability problems.


\section{Adapting the Karp and Miller procedure for \Ecov on P-\ac{PPN}}
\label{sec:adapting-the-karp-and-miller-procedure-for-ecov-on-p-ppn}
From \cite{David17}, it is easy to solve the \Ecov problem on P-PPN by using the adapted Karp and Miller procedure (\cref{sec:km-ecov-postt-ppn}) on an equivalent PostT-PPN (\cref{sec:p-ppn-to-postt-ppn}).
Actually, even if it is not explicitly stated in \cite{David17}, it is clear that one can use,
%with very few modifications,
the Karp and Miller algorithm adapted for PostT-PPN directly on the P-\ac{PPN}:
%\todo{Mais j'ai bien l'impression que ce sera moins performant de toute façon.}

%We describe \emph{Karp and Miller algorithm for P-PPNs}.
Given a marked P-PPN $\defPPN$, one can construct a coverability tree \nameT of \namePPN by applying the Karp and Miller algorithm to the PN $\val[\namePPN]$ with $\Acc$ as the acceleration function, where $\val$ is the $*$-valuation that maps all the parameters of $\parameters$ to $*$.
\todo{(Acc is defined in ..  and * in ..)}

The correctness and the termination of the algorithm comes directly from the bisimulation relation described in \cref{sec:p-ppn-to-postt-ppn}.
Indeed, the state space of the built PostT-PPN can be obtained by simply adding $\mar(p_i) = 0$ to every marking $\mar$ of the state space of the P-PPN (in addition to the created initial marking).

%Note that this method is the same as the one presented in \todo{} for the use of Karp and Miller on a PostT-PPN.

\todo{STOP HERE}


\section{Adapting the general backward algorithm}
\label{sec:adapting-the-general-backward-algorithm}
The backward algorithm does not seem to be easy to adapt to the parametric coverability problems.
Indeed, it works only thanks to the iteration of the $\Pre$ operator.

In the case of PostT-\ac{PPN} and of the \Ecov problem, the idea would be to look for \mari from the markings to cover.
We have seen that, with a forward approach, one can consider only high valuations, that are better because they allow to cover more markings: the search from $\mari$ may go ``too far'' without being an issue.
Here however a high valuation may lead to a premature stop in a backward search, because it restricts ``more quickly'' the set of transitions that is enabled at each step.
In other words, contrary to forward approach, the backward approach has the deals with the 0 bound of the transition precondition. 

However, it may be the case that one can answer the \Ucov problem by applying the backward algorithm on the plain \ac{PN} obtained by the 0-valuation of all its parameter.
This will be a direction of our study.

The situation is symmetrical for PreT-\ac{PPN}, where the algorithm does not seem to be of much help for the \Ucov problem but where considering a 0-valuation may lead to a result about the \Ecov problem.

% vim: set spell spelllang=en :


\section{Adapting \ac{EEC}}% for PreT-\acp{PPN}}
\label{sec:adapting-eec}
In this section we provide an adaption of \ac{EEC} to work with \acp{PPN}.
It is inspired by the \todo{KM\textsuperscript{+}} algorithm from \cite{David17}.

The algorithm is the same as the one presented in the section~\ref{sec:eec}, with the difference that the condition to add a marking to the over-approximation is more restrictive.


% vim: set spell spelllang=en :


\section{Adapting Geeraerts method to PostT-\acp{PPN}}
\label{sec:adapting-geeraerts-method-to-postt-ppn}
Geeraerts method is much more complex.
However, as it uses both a Post operator and an acceleration function, it seems to be easily adapted to \ac{PPN} in the same way that the Karp and Miller algorithm was adapted.

% vim: set spell spelllang=en :



\section{The synthesis problem}
\label{sec:the-synthesis-problem}
\todo{change introduction and many other parts as this was not planned at first}

The use of parameter opens the way to the computation of parameters values ensuring some properties in the system.
This is called the \emph{synthesis problem}.

\begin{defi}[Coverability synthesis problem]
  \label{defi:cov-synth-prblm}
  Given a \ac{PPN} $\defPPN$ and an upward-closed set $\ucs$ of markings on $\places$,
  compute the set $\vcov{\namePPN}{\ucs}$ of valuations of $\parameters$
  such that
  $\cov{\val[\namePPN]}{\ucs} = \top \Leftrightarrow \val \in \vcov{\namePPN}{\ucs}$.
\end{defi}

\subsection{$\vback$ to solve the coverability synthesis problem on P-PPNs}

We present a simple algorithm based on $\back$ to compute $\Vcov$ for all P-PPNs \namePPN and goal sets of markings.
%It is close to the algorithm introduced in \cite{David17} for the ``direct computation of the coverability synthesis set for PreT-PPNs''.

Given a P-PPN $\defPPN$ and a goal upward-closed set of markings $\ucs$,
first compute $\pres{\ucs}$.
Then we have
\(
  \vcov{\namePPN}{\ucs} =
  \setComp{\val}{
    \val[\mar_0] \in \pres{\ucs}
    %\cap \downc{\val_\omega(\mar_0)}
  }
\).

The termination and the correctness follow from the termination and the correctness of $\back$.

%The termination comes from the termination of $\back$.
%
%\begin{lemm}[Correctness]
%  Given a P-PPN \namePPN and an upward-closed set $\ucs$ of markings of \namePPN,
%  then
%  %this two propositions are equivalent:\todo{check words proposition and equivalent}
%  \[
%    \text{(1) } \cov{\val[\namePPN]}{\ucs}
%    \Leftrightarrow
%    \text{(2) } \val[\mar_0] \in \pres{\ucs} \cap \downc{\val_\omega(\mar_0)}
%  \]
%\end{lemm}
%
%\begin{proof}
%  We prove (1) $\Rightarrow$ (2) by contradiction.
%  Suppose $\cov{\val[\namePPN]}{\ucs} = \top$ and
%  $\val[\mar_0] \notin \pres{\ucs} \cap \downc{\val_\omega(\mar_0)}$.
%  By definition of $\omega$,
%  for all valuation $\val$,
%  $\val[\mar_0] \in \downc{\val_\omega(\mar_0)}$.
%  Thus, $\val[\mar_0] \notin \pres{\ucs}$.
%  However, from the proof of the correctness of $\back$ \todo{cite/ref}, we have that
%  $\cov{\val[\namePPN]}{\ucs} \Leftrightarrow \val[\mar_0] \in \pres{\ucs}$.
%  This contradicts our initial supposition.
%
%  To prove that (2) $\Rightarrow$ (1), suppose that there exists $\val$ such that
%  $\val[\mar_0] \in \pres{\ucs} \cap \downc{\val_\omega(\mar_0)}$.
%  Thus, $\val[\mar_0] \in \pres{\ucs}$.
%  From the proof of the correctness of $\back$ \todo{cite/ref}, we have that
%  $\cov{\val[\namePPN]}{\ucs} \Leftrightarrow \val[\mar_0] \in \pres{\ucs}$.
%\end{proof}

\paragraph{Effectiveness}
The $\Vcov$ set may be infinite.
However,
%by using the vector representation of valuations,
by \Cref{theo:pre-upc} we obtain an upward-closed set that can be represented by its minimal elements (\Cref{theo:upward-closed-set-representation}).
%$\downc{\val_\omega(\mar_0)}$ is downward-closed and

Indeed,
$\pres{\ucs}$ are the markings that allow to cover $\ucs$.
%From this set, those which agree with $\mar_0$ on its non-parametric places form an over-approximation of 
Since $\ucs$ is upward-closed, by monotonicity, $\pres{\ucs}$ is upward-closed too (\Cref{theo:pre-upc}) \todo{need a proof?}.
Let $\minp{\pres{\ucs}} = \{\mar_1, \dots, \mar_k\}$.
Then,
\begin{align*}
  \mar_0 \in \pres{\ucs} \Leftrightarrow \ &
    \mar_1 \preceq \mar_0 \vee \dots \vee \mar_k \preceq \mar_0 \\
  \Leftrightarrow \ &
    (\mar_1(p_1) \leq \mar_0(p_1) \wedge \mar_1(p_2) \leq \mar_0(p_2) \wedge \dots) \\
  &\vee \dots \\
  &\vee
    (\mar_k(p_1) \leq \mar_0(p_1) \wedge \mar_k(p_2) \leq \mar_0(p_2) \wedge \dots)
\end{align*}

From this expression, we can remove the conjunctions that can not be true for any valuation, that is, those related to a minimal marking that can not be covered by $\mar_0$ due to a non-parametric place: $\mar \in \minp{\pres{\ucs}}$ with
\(
  \exists p \in \places : \mar_0(p) \notin \parameters \text{ and } \mar(p) > \mar_0(p)
\).
The other conjunctions provide us with the minimal valuations of $\vcov{\namePPN}{\ucs}$:
for each conjunction, each contained proposition is either a tautology (whenever $\mar_0(p) \notin \parameters$), or a lower bound on a parameter (whenever $\mar_0(p) \in \parameters$).
\todo{It is easy to see that the set of valuation thus obtained is correct and complete.}

\todo{thanks to the bisimulation relation described on section .. one can use this method to compute vcov on a PreTPPN}.

\todo{thanks to the cosimulation relation described on section .. one can use this method to compute .. on a PostTPPN}

%%\paragraph{Effectivness}
%The $\Vcov$ set may be infinite.
%However,
%by using the vector representation of valuations, we obtain an upward-closed set that can be represented by its minimal elements.
%%$\downc{\val_\omega(\mar_0)}$ is downward-closed and
%
%Indeed,
%$\pres{\ucs}$ are the markings that allow to cover $\ucs$.
%%From this set, those which agree with $\mar_0$ on its non-parametric places form an over-approximation of 
%Since $\ucs$ is upward-closed, by monotonicity, $\pres{\ucs}$ is upward-closed too \todo{need a proof?}.
%Let $\minp{\pres{\ucs}} = \{\mar_1, \dots, \mar_k\}$.
%That is,
%\begin{align*}
%  \mar \in \pres{\ucs} \Leftrightarrow \ &
%    \mar_1 \preceq \mar \vee \dots \vee \mar_k \preceq \mar \\
%  \Leftrightarrow \ &
%    (\mar_1(p_1) \leq \mar(p_1) \wedge \mar_1(p_2) \leq \mar(p_2) \wedge \dots) \\
%  &\vee
%    (\mar_2(p_1) \leq \mar(p_1) \wedge \mar_2(p_2) \leq \mar(p_2) \wedge \dots) \\
%  &\vee \dots
%\end{align*}
%$\forall \param \in \parameters$,
%let $\mar_0^{-1}(\param)$ be the set places $p$ such that $\mar_0(p) = \param$.
%Without loss of generality, consider that the parameters $\parameters = \{\param_1, \dots, \param_i\}$ in $\mar_0$ are used in, and only in, the $j$ first places.
%Then,
%\begin{align*}
%  \val[\mar_0] \in \pres{\ucs} \Leftrightarrow \ &
%    \mar_1 \preceq \val[\mar_0] \vee \dots \vee \mar_k \preceq \val[\mar_0] \\
%  \Leftrightarrow \ &
%    (\mar_1(p_1) \leq \val[\mar_0](p_1) \wedge \mar_1(p_2) \leq \val[\mar_0](p_2) \wedge \dots) \\
%  &\vee
%    (\mar_2(p_1) \leq \val[\mar_0](p_1) \wedge \mar_2(p_2) \leq \val[\mar_0](p_2) \wedge \dots) \\
%  &\vee \dots
%\end{align*}
%
%
%
%Since parameters are used on the initial marking only,
%we are looking for the valuations such that 
%
%
%\cref{lemm:wqo}


\vspace*{0.5cm}
\acresetall

\chapter*{Conclusions}

In the first chapter we defined the context of our study: \ac{PN} and \ac{PPN} and the coverability problems on them.
The second chapter presented the known results as a basis for our study.
The last chapter give the direction of our study and the result we would like to include in the last version of the work.

\appendix

\backmatter

\printindex % use makeindex to generate the index

\bibliographystyle{plain}

\bibliography{info} %use bibtex to generate the bibliography

\end{document}
