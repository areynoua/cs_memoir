\documentclass[11pt,a4paper,oneside]{book}

% Math typesetting
\usepackage{amsmath} % loads amsbsy, amsopn, amstext
\usepackage{amsthm}
\usepackage{thmtools, thm-restate}
\usepackage{amsfonts}
\usepackage{amssymb}
\usepackage{mathrsfs} % script (\mathscr)
\usepackage{mathtools} % overbracket
\usepackage{fouridx}

\input{res/bordermatrix.tex}

\theoremstyle{plain}
\newtheorem{lemm}{Lemma}
\newtheorem{theo}{Theorem}
\theoremstyle{definition}
\newtheorem{defi}{Definition}
\theoremstyle{remark}

% Encoding and Fonts
\usepackage{xunicode} % replaces fontenc
\usepackage{xltxtra} % loads fontspec, metalogo, realscripts; redefine \showhyphens; define \vfrac and \namedglyph.

\DeclareSymbolFont{sfoperators}{OT1}{cmss}{m}{n}
\SetSymbolFont{sfoperators}{normal}{OT1}{cmss}{m}{n}
\makeatletter
\renewcommand{\operator@font}{\mathgroup\symsfoperators}
\makeatother

\setromanfont{CMU Serif}
\setsansfont{CMU Sans Serif}
\setmonofont{CMU Typewriter Text}
\setmathrm{CMU Serif}
\setmathsf{CMU Sans Serif}
\setmathtt{CMU Typewriter Text}

% Pdf pages numbers and links
\usepackage[
  unicode,
  breaklinks,
  hidelinks,
  pdftitle={The Coverability problem for parametric Petri nets},
  pdfauthor={Alexis Reynouard},
  pdfsubject={Formal verification},
  xetex
]{hyperref}
% Edit \autoref texts
\newcommand*{\fullref}[1]{\hyperref[{#1}]{\autoref*{#1}~\nameref*{#1}}}
\renewcommand\subsectionautorefname{section}
\renewcommand\subsubsectionautorefname{section}
\renewcommand\theoremautorefname{theorem} % TODO

% Typesetting
\usepackage{xspace}
\renewcommand{\textomega}{ω\xspace}

% Layout
\usepackage[hmargin={1.25in,1.25in},vmargin={1.25in,1.25in}]{geometry}
\setlength{\parindent}{0pt}
\setlength{\parskip}{1.5ex}

\usepackage{fancyhdr}
\pagestyle{myheadings}
\fancyhf{}
\rhead[\leftmark]{thepage}

% Bibliography
\usepackage{natbib}
\makeindex

% Index
\usepackage{makeidx}

% Acronyms
\usepackage{acronym}
\acrodef{PN}{Petri net}
\acrodef{PPN}{parametric Petri net}
\acrodef{EEC}{Expand, Enlarge and Check}
\acrodef{WSTS}{well-structured transition system}

% Lists
%\usepackage{enumitem}
%\setlist[itemize]{noitemsep,nolistsep}

% Figures
\usepackage[format=hang]{subfig}

% PGF/TikZ
\usepackage{tikz}
\usetikzlibrary{arrows,decorations,backgrounds,positioning,fit,petri,decorations.pathmorphing,decorations.pathreplacing}
\tikzset{
	x=0.16\linewidth,
	y=0.16\linewidth,
	>=stealth',
	bend angle=30,
	every place/.style={
		draw=black,
		minimum size=0.055\linewidth
	},
	every transition/.style={
		fill=black,
		minimum height=0.055\linewidth,
		inner xsep=0pt,
		minimum width=3pt
	}
}

%%%% Document specific
\title{The Coverability problem for parametric Petri nets}
\author{Alexis Reynouard}
\date{}

% new commands
\usepackage{xparse}
\usepackage{ifthen}

\newcommand{\lang}[1]{{\em{}#1}}

\newcommand{\Ecov}{$\mathscr{E}$-cov\xspace}
\newcommand{\Ucov}{$\mathscr{U}$-cov\xspace}
\newcommand{\oplace}{$\omega$-place\xspace}
\newcommand{\oplaces}{$\omega$-places\xspace}
\newcommand{\noplaces}{non-$\omega$-places\xspace}

% conventions
% fixed: transition = t
% fixed: place = p
% fixed: node = n
% fixed: root = n_0
% fixed: set = {}
% fixed: vector = ()
% fixed: sequence = ()
% command name prefixes: name, def, body

% PNs
\newcommand{\opn}{$\omega$-\ac{PN}\xspace}
\newcommand{\opns}{$\omega$-\acp{PN}\xspace}
\newcommand{\omark}{$\omega$-marking\xspace}
\newcommand{\omarks}{$\omega$-markings\xspace}
% names
\newcommand{\namePN}[1][] {\ensuremath{\mathcal{N}_{#1}}\xspace}
\newcommand{\namePPN}[1][]{\ensuremath{\mathcal{S}_{#1}}\xspace}
% bodies
\newcommand{\bodyPN}[1][] {\tuple{\places_{#1}, \transitions_{#1}, \mar_{0,#1}}}
\newcommand{\bodyPPN}[1][]{\tuple{\places_{#1}, \transitions_{#1}, \parameters_{#1}, \mar_{0,#1}}}
\newcommand{\bodyNonMarkedPN}[1][] {\tuple{\places_{#1}, \transitions_{#1}}}
\newcommand{\bodyNonMarkedPPN}[1][]{\tuple{\places_{#1}, \transitions_{#1}, \parameters_{#1}}}
% defs
\newcommand{\defPN}[1][] {\namePN[#1]  = \bodyPN[#1]}
\newcommand{\defPPN}[1][]{\namePPN[#1] = \bodyPPN[#1]}
\newcommand{\defNonMarkedPN}[1][] {\namePN[#1]  = \bodyNonMarkedPN[#1]}
\newcommand{\defNonMarkedPPN}[1][]{\namePPN[#1] = \bodyNonMarkedPPN[#1]}

% Karp and Miller Tree
\newcommand{\nameT}{\ensuremath{\mathcal{T}}\xspace}
\newcommand{\bodyT}{\tuple{\nodes, \edges, n_0, \Lab}}
\newcommand{\defT}{\nameT = \bodyT}
% sets
\newcommand{\nodes}{N}
\newcommand{\edges}{B}
% functions
\newcommand{\lab}[1]{\Lab(#1)}
\newcommand{\labt}[1]{\Labt\left({#1}\right)}
\newcommand{\parent}[1]{#1^{-1}}
\newcommand{\child}[1]{#1^{+1}}
\newcommand{\tts}[2]{\Tts(#1,#2)} % tree to sequence
\newcommand{\na}[1]{\Na(#1)} % newly accelerated
\newcommand{\ab}[1]{\Ab(#1)} % accelerated before

% object
\newcommand{\tuple}[1]{\left\langle#1\right\rangle}
\newcommand{\setComp}[2]{\left\{#1 \mid #2\right\}}

\NewDocumentCommand{\inw}{ O{} O{} }{I_{#1}% %input weight
  \ifthenelse{\equal{\detokenize{#2}}{\detokenize{}}}{}
  {(#2)}}

\NewDocumentCommand{\outw}{ O{} O{} }{O_{#1}% %output weight
  \ifthenelse{\equal{\detokenize{#2}}{\detokenize{}}}{}
  {(#2)}}

\NewDocumentCommand{\effect}{ m O{} }{\Effect\left(#1\right)%
  \ifthenelse{\equal{\detokenize{#2}}{\detokenize{}}}{}
  {(#2)}}

% matrices
\NewDocumentCommand{\inm}{ O{} O{} }{\mathbf{I}_{#1}%
  \ifthenelse{\equal{\detokenize{#2}}{\detokenize{}}}{}
  {(#2)}}
\NewDocumentCommand{\outm}{ O{} O{} }{\mathbf{O}_{#1}%
  \ifthenelse{\equal{\detokenize{#2}}{\detokenize{}}}{}
  {(#2)}}

% sets
\newcommand{\set}{\ensuremath{\mathcal{E}}\xspace}
\newcommand{\places}{P}
\newcommand{\transitions}{T}
\newcommand{\markings}{\mathcal{M}}
\newcommand{\setm}{\markings}
\newcommand{\naturals}{\mathbb{N}}
\newcommand{\parameters}{\mathbb{P}}
\newcommand{\range}[2]{\{#1, \dots, #2\}}
\newcommand{\ucs}{\mathcal{U}} %upward closed set
\newcommand{\setp}{\mathcal{Q}} %set of places

\newcommand{\front}{F}

% sequence
\newcommand{\seq}[1][]{\mathscr{S}%
  \ifthenelse{\equal{\detokenize{#1}}{}}{}{^{(#1)}}}
\newcommand{\seqt}[1][1]{%sequence of transitions
  \ifthenelse{\equal{\detokenize{#1}}{\detokenize{1}}}{\sigma}{\rho}}
\newcommand{\iscs}[1]{\seqt[2]_{#1}} % increasing self-covering sequence

\NewDocumentCommand{\defSeq}{ O{\seq} O{\sit} m m }{#1 = \bodySeq[#2]{#3}{#4}}
\NewDocumentCommand{\defInfSeq}{ O{\seq} O{\sit} O{1} O{2} }{#1 = \bodyInfSeq[#2][#3][#4]}
\NewDocumentCommand{\defSeqt}{ O{\seqt} m m }{#1 = \bodySeqt{#2}{#3}}

\newcommand{\bodySeqt}[2]{\bodySeq[\transition]{#1}{#2}}
\newcommand{\bodySeqm}[2]{\bodySeq[\mar]{#1}{#2}}

\NewDocumentCommand{\bodySeq}{ O{\sit} m m }{(\elemsSeq[#1]{#2}{#3})}
\NewDocumentCommand{\bodyInfSeq}{ O{\sit} O{1} O{2} }{(\elemsSeq[#1][#2][#3])}

\NewDocumentCommand{\elemsSeq}{ O{\sit} m m }{#1_{#2}, \dots, #1_{#3}}
\NewDocumentCommand{\elemsInfSeq}{ O{\sit} O{1} O{2} }{#1_{#2}, #1_{#3}, \dots}

% seq functions
\newcommand{\concat}{+}
\newcommand{\slice}[3][\seq]{\fourIdx{#2}{}{#3}{}{#1}}


% units
\newcommand{\mar}{\ensuremath{\mathbf{m}}\xspace}
\newcommand{\mari}{\ensuremath{\mar_0}\xspace}
\newcommand{\marp}{\ensuremath{\mathbf{m'}}\xspace} % deprecated
\newcommand{\transition}{t} % fixed
\newcommand{\place}{p} % fixed
\newcommand{\sit}{e} %sequence item
\newcommand{\param}[1][1]{
  \ifthenelse{\equal{\detokenize{#1}}{\detokenize{1}}}
  {a}{%
    \ifthenelse{\equal{\detokenize{#1}}{\detokenize{2}}}
    {b}{%
      \ifthenelse{\equal{\detokenize{#1}}{\detokenize{3}}}
      {c}{%
        \ifthenelse{\equal{\detokenize{#1}}{\detokenize{4}}}
        {c}{!!!\errmessage{unknown param number}}}}}}


% relations
\newcommand{\rela}{\mathrel{R}}

% operators
\DeclareMathOperator{\Ab}{\zeta}
\DeclareMathOperator{\Cover}{Cover}
\DeclareMathOperator{\Downc}{\downarrow^\preceq}
\DeclareMathOperator{\Effect}{Effect}
\DeclareMathOperator{\KMAcc}{Acceleration}
\DeclareMathOperator{\Labt}{\lambda}
\DeclareMathOperator{\Lab}{\Lambda}
\DeclareMathOperator{\Maxs}{Max^\sqsubseteq}
\DeclareMathOperator{\Max}{Max^\preceq}
\DeclareMathOperator{\Min}{Min^\preceq}
\DeclareMathOperator{\Na}{\alpha}
\DeclareMathOperator{\Oplaces}{\Omega}
\DeclareMathOperator{\Post}{Post}
\DeclareMathOperator{\Pre}{Pre}
\DeclareMathOperator{\Tts}{\varsigma}
\DeclareMathOperator{\Upc}{\uparrow^\preceq}
\newcommand{\Treepath}{\mathscr{T}}

% functions
\newcommand{\card}[1]{\left|#1\right|}
\newcommand{\cover}[1]{\Cover(#1)}
\newcommand{\downc}[1]{\Downc(#1)}
\newcommand{\fire}[2][]{\xrightarrow{#2}_{#1}}
\newcommand{\kmacc}[1]{\KMAcc(#1)}
\newcommand{\maxp}[1]{\Max(#1)}
\newcommand{\maxs}[1]{\Maxs(#1)}
\newcommand{\minp}[1]{\Min(#1)}
\newcommand{\op}[1]{\Oplaces(#1)}
\newcommand{\posts}[2][]{\Post^*_{#1}(#2)}
\newcommand{\post}[2][]{\Post_{#1}(#2)}
\newcommand{\pres}[2][]{\Pre^*_{#1}(#2)}
\newcommand{\pre}[2][]{\Pre_{#1}(#2)}
\newcommand{\transpose}[1]{#1^T}
\newcommand{\treepath}[2][]{\slice[\Treepath]{#1}{#2}}
\newcommand{\upc}[1]{\Upc(#1)}
\newcommand{\val}[1][]{v\ifthenelse{\equal{\detokenize{#1}}{}}{}{(#1)}}




% old

\newcommand{\overbar}[1]{\overline{#1\mkern-1.5mu}\mkern 1.5mu}



%\newcommand{\N}{\ensuremath{\mathcal{N}}\xspace}
%\newcommand{\PPN}{\ensuremath{\mathcal{S}}\xspace}
%\newcommand{\PPNi}{\ensuremath{\mathcal{S}_1}\xspace}
%\newcommand{\PPNii}{\ensuremath{\mathcal{S}_2}\xspace}
%\newcommand{\PTm}{\ensuremath{\langle P,T, \mari\rangle}\xspace}
%\newcommand{\PT}{\ensuremath{\langle P,T\rangle}\xspace}
%\newcommand{\NPT}{\ensuremath{\N = \PT}\xspace}
%\newcommand{\NPTm}{\ensuremath{\N = \PTm}\xspace}
%\newcommand{\PTP}{\ensuremath{\langle P,T,\mathbb{P}\rangle}\xspace}
%\newcommand{\PTPm}{\ensuremath{\langle P,T,\mathbb{P}, \mari\rangle}\xspace}
%\newcommand{\PTPmi}{\ensuremath{\langle P_1,T_1,\mathbb{P}_1, \mar_{0,1}\rangle}\xspace}
%\newcommand{\PTPmii}{\ensuremath{\langle P_2,T_2,\mathbb{P}_2, \mar_{0,2}\rangle}\xspace}
%\newcommand{\SPTPm}{\ensuremath{\PPN = \PTPm}\xspace}
%\newcommand{\SPTPmi}{\ensuremath{\PPNi = \PTPmi}\xspace}
%\newcommand{\SPTPmii}{\ensuremath{\PPNii = \PTPmii}\xspace}
%\newcommand{\SPTP}{\ensuremath{\PPN = \PTP}\xspace}


\newcommand{\marpi}{\ensuremath{\mathbf{m'}_0}\xspace}
\newcommand{\marpp}{\ensuremath{\mathbf{m''}}\xspace}

\newcommand{\matIN}{\matI_\N}
\newcommand{\matIS}{\matI_\PPN}
\newcommand{\matI}{\mathbf{I}}
\newcommand{\matON}{\matO_\N}
\newcommand{\matOS}{\matO_\PPN}
\newcommand{\matO}{\mathbf{O}}

\newcommand{\sousrel}{\stackrel{\Sous}{\Longrightarrow}}
\newcommand{\surrel}{\stackrel{\Sur}{\Longrightarrow}}

\newcommand{\vect}[1]{\left(#1\right)}

\DeclareMathOperator{\Accelb}{A\overline{cce}l}
\DeclareMathOperator{\Acc}{Acc}
\DeclareMathOperator{\CovSeq}{CovSeq}
\DeclareMathOperator{\Flatten}{Flatten}
\DeclareMathOperator{\Postb}{P\overbar{ost}}
\DeclareMathOperator{\R}{\mathcal{R}}
\DeclareMathOperator{\Sous}{Sous}
\DeclareMathOperator{\Sur}{Sur}
\DeclareMathOperator{\back}{Back}
\DeclareMathOperator{\covOp}{cov}
\DeclareMathOperator{\EcovOp}{\mathscr{E}cov}
\DeclareMathOperator{\UcovOp}{\mathscr{U}cov}

% TMP
% draft
\newcommand{\dtodo}[1]{[TODO: {\color{red} #1}]\message{TODO: #1}}
\newcommand{\dremoved}[1]{[REMOVED: {\color{gray} #1}]}
\newcommand{\dnote}[1]{{\em\color{gray} #1}}
% final
\newcommand{\ftodo}[1]{\message{TODO: #1}}
\newcommand{\fremoved}[1]{}%\message{REMOVED: #1}}
\newcommand{\fnote}[1]{{\em\color{gray} #1}}

\newcommand{\todo}[1]{\dtodo{#1}}
\newcommand{\removed}[1]{\dremoved{#1}}
\newcommand{\note}[1]{\dnote{#1}}
\newcommand{\rev}[1]{\drev{#1}}
\newcommand{\mov}[1]{\dmov{#1}}
%\newcommand{\todo}[1]{\ftodo{#1}}
%\newcommand{\removed}[1]{\fremoved{#1}}
%\newcommand{\note}[1]{\fnote{#1}}
%\newcommand{\rev}[1]{\frev{#1}}
%\newcommand{\mov}[1]{\fmov{#1}}


\begin{document}

\frontmatter
\begin{titlepage}
\setcounter{page}{0}
\begin{center}
\textbf{UNIVERSIT\'E LIBRE DE BRUXELLES}\\
\textbf{Faculté des Sciences}\\
\textbf{Département d'Informatique}
\vfill{}\vfill{}

{\Huge  The Coverability problem \vspace*{.5cm}  \linebreak[4] for parametric Petri nets}

{\Huge \par}
\begin{center}{\LARGE Alexis Reynouard}\end{center}{\Huge \par}
\vfill{}\vfill{}
\begin{flushright}{\large \textbf{Promoter :} Prof. Gilles Geeraerts}\hfill{}{\large Master Thesis in Computer Sciences}\\
{\large }\hfill{}{}\end{flushright}{\large\par}
\vfill{}\vfill{}\enlargethispage{3cm}
\textbf{Academic year 2018~--~2019}
\end{center}
\end{titlepage}
\newpage
\thispagestyle{empty}
\null

\newenvironment{vcenterpage}
{\newpage\thispagestyle{empty} 
\vspace*{\fill}}
{\vspace*{\fill}\par\pagebreak}

%\begin{vcenterpage}
%\begin{flushright}
%    \large\em\null\vskip1in 
%    You may want\\
%   to write a dedication here\vfill
%  \end{flushright}
%\end{vcenterpage}
%\thispagestyle{empty}
%\vspace*{5cm}
%
%\begin{quotation}
%\noindent ``\emph{You may also include one or more general quotes related to your topic.}''
%\begin{flushright}\textbf{Name of the author, date}\end{flushright}
%\end{quotation}
%
%\medskip
%
%\begin{quotation}
%\noindent ``\emph{Another quote.}''
%\begin{flushright}\textbf{Name of the author, date}\end{flushright}
%\end{quotation}
%\chapter*{Acknowledgment}
%\thispagestyle{empty} 
%
%\noindent I want to thank ...

\thispagestyle{empty} 
\setcounter{page}{0}
\tableofcontents
\acresetall

\mainmatter
\setcounter{page}{1}

\chapter{Introduction}
\acp{PN} are a mathematical and graphical model introduced by Carl Adam Petri in 1962 \citep{Petri62,Petri66}.
It was successfully used to analyse systems in a wide range of domains, and has proven to be particularly successful for the formal verification of asynchronous systems, like manufacturing systems \cite{li2009deadlock}.
\Cref{sec:some-uses-of-pn}

In their standard definition, \acp{PN} are instantiated through many natural numbers\footnote{\ie a \ac{PN} may be represented as a pair of matrices whose the values are from $\naturals$, see \Cref{sec:the-pn-model}.} which may represent, for example, the amount of resource needed for a given action to be carried out.

The introduction of parameters into the model to avoid the need to state these values explicitly%
\footnote{One can find in the literature many other way to use parameters in \acp{PN}. For example, place and\,/\,or transitions may also be parameters in order to dynamically change the network structure, like in \cite{Christensen97}.}
may have several benefits:
it can allow an efficient analysis of a whole family of \acp{PN}, like in \cite{Abdulla13}, or to model dynamic changes in the system, as introduced by \cite{Badouel99} as a subclass of reconfigurable nets.

The use of parameters increases the modelling power of \acp{PN} but also make some basic coverability problems undecidable in the general case \cite{David17}.

We adopt the parametric Petri net model introduced by \cite{David17}, which seems the most intuitive and general, \rev{and we study the existing results and algorithms for plain Petri nets to find out whether they still hold or how to adapt them to the parametric model.}

The rest of the document is as follows:
\rev{In this first part, we define the plain Petri net model (\ie the classical one) and the parametric model.
We then briefly motivate our study, \todo{give concrete examples of applications,} and give an overview of the previous works on parametrisation of \acp{PN}.
Finally, we place the \ac{PN} model in a broader model family: the \acp{WSTS}.\\
In a second part, we recall, first, the classical results that we will study on this new model, second, the results already obtained for the parametrized Petri net model as we have defined them.\\
Then, we focus on the parametric Petri net model to establish whether the results related to the coverability problem in the plain Petri net model still hold or if the algorithms may be adapted to this new model.}

\acresetall

\section{Definitions}
\begin{defi}[\acl{PN}]
  A \acf{PN} \PN is a weighted oriented bipartite graph, whose the two subsets of vertices define a tuple \PT where:
  \begin{itemize}
    \item $P$ is a finite set of places,
    \item $T$ is a finite set of transitions.
  \end{itemize}
  For each transition $t \in T$ are defined (exactly) these two functions:
  \begin{itemize}
    \item $I_t : P \mapsto \mathbb{N}$ associates to each place the weight of the edge to $t$ \emph{(input weight)},
    \item $O_t : P \mapsto \mathbb{N}$ associates to each place the weight of the edge from $t$ \emph{(output weight)}.
  \end{itemize}
  It is denoted by $t = \langle I_t, O_t \rangle$.
  Because these functions define the edges of the graph, a \ac{PN} is completely defined by the tuple \PT and so is denoted by \NPT.
\end{defi}

\begin{defi}[marking]
  Given a set of place $P$, a marking of $P$ is a function $\mar : P \mapsto \mathbb{N}$ that associates $\mar(p)$ tokens to each place $p \in P$.
\end{defi}

An order on the markings is essential for the analysis of \acp{PN}. The order we will define is a well quasi-order and a partial order.

\begin{defi}[quasi-order]
  A quasi-order on a set $\mathcal{E}$ is a binary relation $R$ that is:
  \begin{align*}
    \text{reflexive: } &&\forall x \in \mathcal{E},\ & x \mathrel{R} x \\
    \text{transitive: } &&\forall (x, y, z) \in \mathcal{E}^3,\ & (x \mathrel{R} y\land y \mathrel{R} z)\Rightarrow x \mathrel{R} z
  \end{align*}
\end{defi}

\begin{defi}[well quasi-order]
  A well quasi-order $\sqsubseteq$ on a set $\mathcal{E}$ is a quasi-order on $\mathcal{E}$ such that, for any infinite sequence $s = e_0, e_1, e_2, \dots$ of elements from $\mathcal{E}$, there exist indices $i < j$ with $e_i \sqsubseteq e_j$. That is, there is no infinite antichain in $\mathcal{E}$ for this relation.
\end{defi}

\begin{defi}[partial order]
  A partial order on a set $\mathcal{E}$ is a quasi-order $R$ that is
  \begin{align*}
    \text{antisymmetric: } &&\forall (x, y) \in \mathcal{E}^2,\ & (x \mathrel{R} y\land y \mathrel{R} x)\Rightarrow x = y
  \end{align*}
\end{defi}

\begin{defi}[partial order \(\preceq\) on the markings]
  Given a set of places $P$, the partial order \(\preceq \subseteq \mathbb{N}^{|P|} \times \mathbb{N}^{|P|}\) is such that for all pair of markings \((\mar_1, \mar_2) \in \mathbb{N}^{|P|} \times \mathbb{N}^{|P|}\) we have that \(\mar_1 \preceq \mar_2\) if and only if for all place \(p \in P : \mar_1(p) \leq \mar_2(p)\).
\end{defi}

\(\mar \prec \marp\) denotes that \(\mar \preceq \marp \text{ and } \marp \npreceq \mar\).

\begin{lemm}[\cite{Dickson13}]
  \label{lemm:wqo}
  $\preceq$ is a well quasi-order.
\end{lemm}

The following result will be useful in the sequel.

\begin{lemm}[\cite{Brams83}]
  The \ac{PN} model is \emph{strongly monotonic with regard to $\preceq$}. That is, for all \ac{PN} $\PN = \PTm$, for all transition $t \in T$ and for all markings $\mar_1, \mar_2, \mar_3$ of \PN such that $\mar_1 \preceq \mar_2$ and $\mar_1 \fire{t} \mar_3$, there exists a marking $\mar_4$ of \PN such that $\mar_2 \fire{t} \mar_4$ and $\mar_3 \preceq \mar_4$.
\end{lemm}

In this work we will focus on an extension of the \ac{PN} model, the \ac{PPN} model, that is extended thanks to the use of parameters as input and output weights.

\begin{defi}[\acl{PPN} \citep{David17}]
  A \acf{PPN} \SPTP is a weighted oriented bipartite graph with a finite set $\mathbb{P}$ of parameters. The two subsets of vertices are:
  \begin{itemize}
    \item $P$: a finite set of places,
    \item $T$: a finite set of transitions,
  \end{itemize}
  For each transition $t \in T$ are defined the following functions:
  \begin{itemize}
    \item $I_t : P \mapsto \mathbb{N} \cup \mathbb{P}$ associates to each place the weight of the edge to $t$ \emph{(input weight)},
    \item $O_t : P \mapsto \mathbb{N} \cup \mathbb{P}$ associates to each place the weight of the edge from $t$ \emph{(output weight)}.
  \end{itemize}
\end{defi}

As for plain \acp{PN}, this is denoted $t = \langle I_t, O_t \rangle$.

\begin{defi}[parametric marking]
  Given a set of place $P$, a parametric marking of $P$ is a function $\mar : P \mapsto \mathbb{N} \cup \mathbb{P} $ that associates $\mar(p)$ tokens to each place $p \in P$.
\end{defi}

A marking of a \ac{PN} \NPT is a marking of $P$.
A marking of a \ac{PPN} \SPTP is a \emph{parametric} marking of $P$.
Note that a marking \mar is a parametric marking where $\mar(p) \in \mathbb{N}$ for all $p \in P$.

More generally, given a set

\begin{defi}[initialized (parametric) \ac{PN}]
  An initialized \ac{PN} \NPTm (resp. \ac{PPN} \SPTPm) is a \ac{PN} (resp. \ac{PPN}) with an initial marking \mari.
\end{defi}

This is sometimes called a \emph{marked (parametric) \ac{PN}}.
We will often refer to an initialized (parametric) \ac{PN} loosely as a (parametric) \ac{PN}.

The figure~\ref{fig:parametric-petri-net-example} shows an example of \ac{PPN} whose $\mathbb{P} = \{a, b\}$ and with an initial marking \mari such that $\mari(p_1) = 1$, $\mari(p_2) = a$ and $\mari(p_3) = 0$. The circles represent the places, the rectangles are the transitions, and the dots are the tokens. If the number of token at a given place is parametric (\lang{i.e.} depends on a parameter of $\mathbb{P}$), it is written inside the circle. An arrow from a place $p$ and to a transition $t$ denotes that $I_t(p) = 1$. The absence of an arrow from $p$ to $t$ indicates that $I_t(p) = 0$. If $I_t(p) \notin \{0, 1\}$, a label with the value of $I_t(p)$ is added to the arrow.
Symmetrically, the arrows from the transitions to the places indicate the output weights.

\begin{figure}[h]
  \centering
  \begin{tikzpicture}[auto,x=0.12\linewidth,y=0.11\linewidth]
	\node [place,tokens=1] (d) [label=$p_1$] at (4,2) {};
	\node [place] (l) [label=west:$p_2$] at (4,1) {$a$};
	\node [place] (o) [label=east:$p_3$] at (6,1) {};
	
	\node [transition] (S) [label=$t_1$] at (3,2) {}
  edge [post] node [auto] {$b$} (d);
	\node [transition] (C) [label=$t_2$] at (5,2) {}
	edge [pre]  (d)
	edge [pre,  bend right] (l)
	edge [post, bend left]  (o);
	\node [transition] (F) [label=$t_3$] at (5,0) {}
	edge [pre,  bend right] (o)
	edge [post, bend left]  (l);
\end{tikzpicture}

  \par
  \caption{An initialized \ac{PPN}}
  \label{fig:parametric-petri-net-example}
\end{figure}

We usually set an order on the places.
This allows to view the markings as vectors (here, \mari is the column vector $(1, a, 0)^T$, where $\cdot^T$ is the transpose operator) as well as the $I$ and $O$ functions.
Likewise, we define an order on the transitions.
Therefore, $I_t$ and $O_t$ denote respectively the $I$ and $O$ functions defined for the $t$\textsuperscript{th} transition (here, $I_1 = (0, 0, 0)^T$ and $O_1 = (b, 0, 0)^T$).
Given a \ac{PPN} \SPTP, the backward and forward incidence matrices $\matIS \in (\mathbb{N} \cup \mathbb{P})^{|P|\times|T|}$ and $\matOS \in (\mathbb{N} \cup \mathbb{P})^{|P|\times|T|}$ are naturally defined by $\matIS(p, t) = I_t(p)$ and $\matOS(p, t) = O_t(p)$.
($\PPN$ is omitted when it is obvious from the context.)
%In addition, it makes the equivalence between \acp{PN} and \emph{vector addition systems} introduced in \cite{Karp69} more explicit and
This allows to use linear algebra to analyse \acp{PN}.

\begin{figure}[h]
	\[
		\matI = \bordermatrix[{[]}]{%
					& t_1 & t_2 & t_3 \cr
			p_1 & 0   & 1   & 0   \cr
			p_2 & 0   & 1   & 0   \cr
			p_3 & 0   & 0   & 1   }
		\mspace{56mu}
		\matO = \bordermatrix[{[]}]{%
					& t_1 & t_2 & t_3 \cr
			p_1 & b   & 0   & 0   \cr
			p_2 & 0   & 0   & 1   \cr
			p_3 & 0   & 1   & 0   }
	\]
  \caption{The incidence matrices of the \ac{PN} from figure \ref{fig:parametric-petri-net-example}}
  \label{fig:incidence-matrices-example}
\end{figure}

%\begin{defi}[Vector addition system]
%  A vector addition system of dimension $n$ is a pair $\langle d, W\rangle$ where $d \in \mathbb{N}^n$ is called the \emph{start vector} and $W$ is a finite set of vector $\mathbb{Z}^n$.
%\end{defi}
%This corresponds to the definition of an initialized \ac{PN} and \todo{we will see that the semantic corresponds too}.

\subsection{Operational semantic of \acp{PN}}

Given a \ac{PN} \NPT and a marking \mar on \PN, a transition $t \in T$ is said \emph{enabled} by \mar if $\forall p \in P : \mar(p) \geq I_t(p)$. An enabled transition can be \emph{fired} to produce a new marking \marp such that $\forall p \in P : \marp(p) = \mar(p) - I_t(p) + O_t(p)$. This is denoted by $\mar \fire{t} \marp$.
It is important to note that the effect of a transition is to add or remove a constant number of tokens at each place and does not depend on the marking from which it is fired. A \ac{PN} transition is said to have a \emph{constant effect}.

Here are some additional notations:
\begin{itemize}
  \item $\mar \rightarrow \marp$ denotes that there exists $t \in T$ such that $\mar \fire{t} \marp$.
  \item $\mar \fire{\sigma} \marp$ where $\sigma$ is a sequence of transitions $\sigma = (t_1, \dots, t_{n-1}), t_i \in T, i \in \{1, \dots, n-1\}$ denotes that there exists a sequence of markings $\mar_1, \dots, \mar_n$ such that : $\mar = \mar_1 \fire{t_1} \cdots \fire{t_{n-1}} \mar_n = \marp$.
  \item $\mar \fire{*} \marp$ denotes that there exists a sequence of transition $\sigma$ such that $\mar \fire{\sigma} \marp$.
    Note that the $\fire{*}$ relation is the reflexive and transitive closure of the relation $\rightarrow$.
\end{itemize}

\begin{defi}
  Given an \ac{PN} \NPT and a marking \mar of \PN:
  \begin{itemize}
    \item $\Post(\mar) = \{\marp \mid \mar \rightarrow \marp\}$ is the set of one-step successors of \mar,
    \item $\Pre(\mar) = \{\marp \mid \marp \rightarrow \mar\}$ is the set of one-step predecessors of \mar,
    \item $\Post^*(\mar) = \{\marp \mid \mar \fire{*} \marp\}$ is the set of successors of \mar, in any number of step.
      With $\mari$ the initial marking of \PN, $\Post^*(\mari)$ is the \emph{reachability set} of \PN.
    \item $\Pre^*(\mar) = \{\marp \mid \marp \fire{*} \mar\}$ is the set of predecessors of \mar, in any number of step.
  \end{itemize}
\end{defi}

These operators are naturally extended to sets of markings as the union of the sets obtained by applying the operator on each marking of the sets.
That is, with $M$ a set of markings of \PN,
$\Post(M) = \{\marp \mid \exists \mar \in M : \mar \rightarrow \marp\}$.

For example, regarding the \ac{PPN} shown on figure \ref{fig:parametric-petri-net-example},
$\Post((0,1,0)) = \{(b, 1, 0)\}$
and
$\Post^*((0,1,0)) = \{(i, 1, 0) \mid i \in \mathbb{N}\} \cup \{(i, 0, 1) \mid i \in \mathbb{N}\}$.

All of this applies to \ac{PPN} through valuations of the parameters:
\begin{defi}[Instantiation of \acp{PPN}]
  Let \SPTPm be a \ac{PPN} and $v : \mathbb{P} \mapsto \mathbb{N}$ be a \emph{$\mathbb{N}$-valuation}, or simply valuation, on $\mathbb{P}$.
  Then $v(\PPN)$ is defined as the \ac{PN} obtained by replacing each parameter $a \in \mathbb{P}$ by $v(a)$.
  Thus, we have $v(\PPN) = \langle P, T, \marpi\rangle$ such that :
  \begin{itemize}
    \item $\matI_{v(\PPN)}(p, t) =
      \begin{cases}
        \phantom{v(}\matIS(p, t) & \text{if } \matIS(p, t) \in \mathbb{N} \\
                 v(\matIS(p, t)) & \text{if } \matIS(p, t) \in \mathbb{P}
      \end{cases}$
    \item $\matO_{v(\PPN)}(p, t) =
      \begin{cases}
        \phantom{v(}\matOS(p, t) & \text{if } \matOS(p, t) \in \mathbb{N} \\
                 v(\matOS(p, t)) & \text{if } \matOS(p, t) \in \mathbb{P}
      \end{cases}$
    \item $\marpi(p) =
      \begin{cases}
        \phantom{v(}\mari(p) & \text{if } \mari(p) \in \mathbb{N} \\
                 v(\mari(p)) & \text{if } \mari(p) \in \mathbb{P}
      \end{cases}$
  \end{itemize}
\end{defi}

Given \PPN a \ac{PPN} and a valuation $v$, one can thus instantiate a \ac{PN} $v(\PPN)$ from \PPN and apply the semantic described above.  When the \ac{PPN} under consideration is clear from the context, $\matI_v$ is used to denote $\matI_{v(\PPN)}$ and $\matO_v$ to denote $\matO_{v(\PPN)}$. We write $\firev{t}$, $\rightarrow_v$, $\firev{\sigma}$, $\firev{*}$, $\Post_v$, $\Pre_v$, $\Post^*_v$ and $\Pre^*_v$ to denote $\fire{t}$, $\rightarrow$, $\fire{\sigma}$, $\fire{*}$, $\Post$, $\Pre$, $\Post^*$ and $\Pre^*$ on the plain \ac{PN} $v(\PPN)$.

This makes it possible to formally represent a system and interactions between its components. We will now define some properties that the model may have and that are usually of interest to show that the modelled system meets some requirements.

\subsection{Behavioural properties of \acp{PN}}

The markings basically indicate the state of the system. Thus, knowing if an initialized \ac{PN} may reach a given marking, that represents for example a bad state, is essential to check properties of the modelled system. This is the \emph{reachability problem}.

\begin{defi}[Reachability]
  Given an initialized \ac{PN} \NPTm and a marking \mar of \PN, \mar is said reachable if $\mari \fire{*} \mar$.
\end{defi}

However, the verification of safety properties are more often reduced to a \emph{coverability problem}, that is essentially asking if a \ac{PN} can reach or exceed a given marking.

\begin{restatable}[Coverability]{defi}{coverability}
  Given an initialized \ac{PN} \NPTm and a marking \mar of \PN, \mar is said coverable if there exists a marking \marp such that $\mar \preceq \marp$ and $\mari \fire{*} \marp$.

  A set of markings is said coverable whenever one of them is coverable.
\end{restatable}

\begin{defi}[Coverability problem]
  Given an initialized \ac{PN} \NPTm and a set $M$ of markings of \PN, determine whether $\exists \mar \in M \text{ and } \marp \in \Post^*(\mari) \text{ such that } \mar \preceq \marp$.

  The coverability problem for a marking \mar is the coverability problem for the singleton $\{\mar\}$.
\end{defi}

The behaviour of a \ac{PPN} is defined by the behaviours of all the \acp{PN} that can be obtained by a valuation of its parameters.
So, for an initialized \ac{PPN} \PPN, the coverability problem may be declined in an existential and an universal form.
The existential coverability problem (\Ecov) ask if there exists a valuation $v$ such that \mar is coverable.
The universal coverability problem (\Ucov) ask if \mar is coverable for all valuations $v$.

\begin{defi}[Universal and existential coverability problems]
  Given a \ac{PPN} \SPTPm and a set $M$ of non-parametric markings of \PPN
  \begin{itemize}
    \item the \emph{existential coverability problem} ask if there is a valuation $v$ for $\mathbb{P}$ such that $M$ is coverable,
    \item the \emph{universal   coverability problem} ask if $M$ is coverable for all valuations of $\mathbb{P}$.
  \end{itemize}
\end{defi}


% vim: spell spelllang=en :

\section{Motivations}
\subsection{Interests of \acp{PPN}}

\todo{sources and examples}

Today \acp{PN} are used in a wide range of areas.
They are commonly used either to design a safe system, or to verify an existing one.
These uses require that the system is complete.
That is, for the design of a model, it must be entirely designed to be analyzable.
On the other hand, when checking an existing system, if a desired property does not hold, the correction must be made ``by hand''.

With the introduction of parameters some variables unknown at the design stage can be integrated into the model without having to be set arbitrarily. Moreover, if during the verification a desired property turns out not to hold, it is possible to check if the change of parameters alone can solve the problem, or if the Petri net structure must be changed too. Going further, the use of parameters in the model can allow to determine ``the safest values'' for a system, or to synthesize the values that allow to respect a given strategy.

We can therefore say that parameters can simplify the \emph{design} of a system. Indeed, since it is possible to keep unknown values, modelling can be done step by step, with the possibility to check the model at each step.
In addition, the design can be partially automated by parameter synthesis.
This approach gives a new interest in this model in fields as varied as chemistry, construction processes, financial loans...
\cite{David17} contains good illustrative examples.

There are also many advantages of using parameters when it comes to \emph{verification}.
For example, they allow to verify some properties simultaneously on many systems that differs only by parameters values.

\subsection{Interest of the coverability problem in \acp{PPN}}

\todo{sources and examples}

As it provides evidence of safety properties on the studied systems, coverability problem is of primary interest in system design and verification. Therefore, for the reasons given in the previous section, it is worth being able to solve it efficiently on \acp{PPN}.

To give a more concrete intuition on the interest, consider a system that execute a \emph{task} for others systems.
At each instant (whatever an instant is), the system may receive requests to perform the task from many other systems. We say that each request creates a \emph{job}.
We would like to have a system that is not too expensive to implement, but also capable of completing the tasks quickly enough.
For this we make our system capable of performing $a$ jobs at the same time, keeping $a$ as low as possible to reduce costs.
This system may be modeled as shown on the figure~\ref{fig:parametric-petri-net-example} with $\mari = (0, a, 0)$ as initial marking.
$p_1$ represents the job queue and $p_3$ the execution unit.\\
We can now formally verify that, whatever the parameter values, the execution unit will not receive more than $a$ jobs to perform at the same time, that is an instance of the \Ecov for the marking $(0, 0, a+1)$. Indeed, it is easy to see that $\Post^*((0, a, 0)) = \{(i, j, k) \mid i, j \text{ and } k \in \mathbb{N}, j + k = a\}$.
\todo{maybe remove part on `$a$ as low as possible'}

Before recalling the known results on \ac{PPN} and plain \ac{PN} that will be useful for our study, let us give a brief overview of some work already done on \acp{PPN}.

% vim: set spell spelllang=en :


%\chapter{Situation}
\section{Previous works on parametrization of \acp{PN}}
The use of parameters in formal verification systems is a well developed topic in the literature.

With regard to \ac{PN}, parameters have been introduced with many different roles.
Some works, like \cite{Christensen97} use parameters as places or transitions, for example to make it possible to change a place into a more complex subnet and thus allow different levels of abstractions to be considered.
In \cite{Lindqvist91}, parameters are used on the markings to obtain concise parametrised reachability trees, but not to realize formal verifications on these parametric systems.

\cite{Badouel99} introduce parameters as the weight of arcs to model changes in a system.
The parameters have a finite valuation domain and verifications are performed on these parametrized systems.
Systems with quantitative parameters with infinite valuation domains are analysed in \cite{Abdulla13}.
Similarly, \cite{Marsan94} study \acp{PN} with parametric initial markings.
%(called \ac{PN} models in contrast to \ac{PN} systems where the initial markings does not have parameters).

Our work is in the line with \cite{David17} which use discrete parameters as arc weights as well as in the markings.
\cite{David17} provide also a proof for the non decidability of \Ucov and \Ecov, and define several subclasses of \ac{PPN} for which these problems are decidable.


\todo{\textomega-markings and \textomega-Petri nets \cite{Geeraerts15}}

% vim: spell spelllang=en :

\section{Overview of similar models}
Many reactive systems are naturally modeled as infinite-state systems.
Some infinite-states models are known to allow some automatic formal verification from model-checking techniques.
Among them are \acp{PN}, but also Lossy FIFO systems, Broadcast protocols...
\todo{refs}
All are \acp{WSTS} (but there exists other famous infinite-states systems, like timed-automata).
\todo{check whether timed-automata are \acp{WSTS}.}

\acp{WSTS} are outside of the range of this work, but we will sometimes refer to them to distinguish between techniques specific to the \ac{PN} model or usable in a wider range of contexts.
\todo{Check english writting.}

In a few words, \acp{WSTS} are transitions systems whose set of states are well-quasi-ordered and whose transition relations is monotonic with respect to the well quasi-ordering.

The monotonicity property differ from the strong monotonicity defined above by the fact that the second state may be found after many steps. Formally:
\begin{defi}[Monotonicity]
  A transition system is said \emph{monotonic} whenever \todo{or `when'?} its transition relation is monotonic.

  A transition relation $\rightarrow \subseteq (S \times S)$ over a $\leq$-well quasi-ordered set $S$ is monotonic if, and only if, for all $s_1$, $s_2$, and $s_3$ from $S$ such that $s_1 \leq s_2$ and $s_1 \rightarrow s_3$ there exists $s_4 \in S$ such that $s_2 \fire{*} s_4$.
\end{defi}


\chapter{Preliminary results}
This chapter intends to introduce known results on both \ac{PN} and \ac{PPN}.
They are not extensively described here:
since they are mentioned only as far as they can help the presentation of the results in the next chapter, the intuition is often given without any detail nor proof.
The reader is therefore encouraged to refer to the mentioned sources.

\note{This chapter is intended to contain more details and proofs as they are found useful for writing the next chapter.}

\section{Known results for coverability on plain \acp{PN}}
We now present the results related to the coverability problem on the plain \acp{PN} which seem to us the most interesting.

To introduce these results, we need some additional definitions.
They will be given for plain \acp{PN}, but most of them are naturally extended to \acp{PPN}.

The \emph{coverability set} of an initialized \ac{PN} is an over-approximation of the reachability set that is precise enough to solve the coverability problem, and is, therefore, interesting for our study.
In order to define it formally, we need to know about the upward and downward closure of a (set of) marking(s).

\begin{defi}[Upward- and downward-closure on markings]
  Let $S \subseteq \mathbb{N}^{|P|}$ be a set of markings on the places $P$:
  \begin{itemize}
    \item The \emph{upward-closure} of $S$, noted $\upc(S)$, is the set
      $\{\marq \in \mathbb{N}^{|P|} \mid \exists \marqp \in S : \marqp \pleq \marq\}$,
    \item The \emph{downward-closure} of $S$, noted $\downc(S)$, is the set
      $\{\marq \in \mathbb{N}^{|P|} \mid \exists \marqp \in S : \marq \pleq \marqp\}$.
  \end{itemize}
  The closure of a marking \marq is the closure of the singleton $\{\marq\}$.
\end{defi}

For instance, with $\marq = (1, 2, 3)$, we have that its upward-closure is $\upc(\marq) = \{(i, j, k) \mid i \geq 1, j \geq 2, k \geq 3\}$.

\begin{defi}[Upward- and downward-closed set of markings]
  A set $S$ of markings is said \emph{upward-closed} if $S = \upc(S)$.
  It is said \emph{downward-closed} if $S = \downc(S)$.
\end{defi}

\begin{defi}[Coverability set \citep{Finkel87,Finkel90}]
  Given an initialized \ac{PN} $\tupleN = \PTm$, a \emph{coverability set} $S$ of \tupleN is a set of markings such that $\downc(S) = \downc(\Post^*(\marqi))$.
  
  Obviously, the minimal coverability set is thus $\downc(\Post^*(\marqi))$ and is noted $\Cover(\tupleN)$.
\end{defi}

\begin{figure}[htbp]
  \label{fig:reach-and-cover-example}
  \centering
  \subfloat[A \ac{PN} ($|P| = 2$)]{
    \label{fig:two-net}
    \begin{tikzpicture}[auto,x=0.12\linewidth,y=0.11\linewidth]
  \node [place, tokens=4] (y) [label=$y$] at (0,0) {};
  \node [place] (x) [label=$x$] at (2,0) {};

  \node [transition] (1) [label=$t_1$] at (1,0) [transition] {}
    edge [pre] node[midway, above] {2} (y)
    edge [post] (x);
  \node [transition] (2) [label=$t_2$] at (3,0) [transition] {}
    edge [pre] (x);
    %\node at (1.5,-0.75) {\label{2net} (1)\quad Un réseau de Petri ($|P| = 2$)};
\end{tikzpicture}


  }

  \subfloat[The reachable markings]{
    \label{fig:two-reach}
    \begin{tikzpicture}[auto,x=1.1cm,y=1.1cm]
  \tikzset{
    >=stealth',
    axis/.style={thin, ->, line join=miter, color=gray},
    dot/.style={circle,fill=black,minimum size=4pt,inner sep=0pt,
          outer sep=-1pt}
  }
  \draw[axis,<->] (2.5,0) node(xline)[right] {$x$} -|
          (0,4.5) node(yline)[above] {$y$};

  \draw (1,1pt) -- (1,-1pt) node[anchor=north] {$1$};
  \draw (1pt,1) -- (-1pt,1) node[anchor=east] {$1$};

  \node[dot] (04) at (0,4) {};

  \node[dot] (12) at (1,2) {};
  \node[dot] (02) at (0,2) {};

  \node[dot] (20) at (2,0) {};
  \node[dot] (10) at (1,0) {};
  \node[dot] (00) at (0,0) {};

  \draw[->] (04) to node {$t_1$} (12);
  \draw[->] (12) to node {$t_1$} (20);
  \draw[->] (02) to node {$t_1$} (10);

  \draw[->] (12) to node[above] {$t_2$} (02);
  \draw[->] (20) to node[above] {$t_2$} (10);
  \draw[->] (10) to node[above] {$t_2$} (00);

  %\node at (1.2,-1) {\label{2reach} (2)\enspace Les marquages accessibles};
\end{tikzpicture}


  }\qquad
  \subfloat[The minimal coverability set]{
    \label{fig:two-cover}
    \begin{tikzpicture}[auto,x=1.1cm,y=1.1cm]
  \tikzset{
    >=stealth',
    axis/.style={thin, ->, line join=miter, color=gray},
    dot/.style={circle,fill=black,minimum size=4pt,inner sep=0pt,
          outer sep=-1pt}
  }

  \draw[axis,<->] (2.5,0) node(xline)[right] {$x$} -|
          (0,4.5) node(yline)[above] {$y$};

  \draw (1,1pt) -- (1,-1pt) node[anchor=north] {$1$};
  \draw (1pt,1) -- (-1pt,1) node[anchor=east] {$1$};

  \node[dot] at (0,4) {};

  \node[dot] at (1,2) {};
  \node[dot] at (0,2) {};

  \node[dot] at (2,0) {};
  \node[dot] at (1,0) {};
  \node[dot] at (0,0) {};

  \node[dot, color=black!60!white] at (0,3) {};
  \node[dot, color=black!60!white] at (0,1) {};
  \node[dot, color=black!60!white] at (1,1) {};

  %\node at (1.2,-1) {\label{2cover} (3)\enspace L'ensemble de couverture};
\end{tikzpicture}


  }
  \caption{Reachability and minimal coverability sets}
\end{figure}

The figure~\ref{fig:two-net} shows a marked \ac{PN} with two places.
One can therefore represents the markings as points on a plane.
The figure~\ref{fig:two-reach} shows the reachable markings in the form of an accessibility graph.
In~\ref{fig:two-cover} we see the minimal coverability set.

Sometimes the number of token in a place is unbounded (\lang{c.f.} the place boundedness problem).
Therefore, the reachability and coverability sets are infinite.
In a plain \ac{PN}, this is due to the existence of an increasing self-covering sequence.
\begin{defi}[Self-covering sequence]
  Given an initialized \ac{PN} $\tupleN = \PTm$,
  a self-covering sequence is a sequence of the form:
  \[
    \marqi \fire{\rho} \marq_i \fire{\sigma} \marq_j
  \]
  with $\rho$ and $\sigma$ two sequences of transitions of $T$
  and with $\marq_i \pleq \marq_j$.
\end{defi}

Note that, since $\marq_i \pleq \marq_j$, $\sigma$ is firable from $\marq_j$.
In addition, the monotonicity of \acp{PN} ensure that, with $\marqp_j$ given by $\marq_j \fire{\sigma} \marqp_j$, we have $\marq_j \pleq \marqp_j$.
Thus, we see that it is a sufficient condition for the \emph{non-termination} of the system (the system may be able to fire transitions infinitely often).
In fact, because $\pleq$ is a well quasi order (Lemma~\ref{lemm:wqo}), one can find in any infinite sequence $\marqi \fire{} \marq_1 \fire{} \dots$ two markings $\marq_i$ and $\marq_j$ such that $\marq_i \pleq \marq_j$.
And so, any infinite sequence is self-covering, and the existence of such a sequence is also necessary for the non-termination of the system.

\begin{defi}[Increasing self-covering sequence]
  Given an initialized \ac{PN} $\tupleN = \PTm$,
  an increasing self-covering sequence is a sequence of the form:
  \[
    \marqi \fire{\rho} \marq_i \fire{\sigma} \marq_j
  \]
  with $\rho$ and $\sigma$ two sequences of transitions of $T$
  and with $\marq_i \prec \marq_j$.
\end{defi}

Let $Q \subseteq P$ be the set of places $Q = \{q \in P \mid \marq_i(q) < \marq_j(q)\}$.
$Q \neq \emptyset$ since $\marq_i \prec \marq_j$.

With a similar reasoning to the above, we see that having such a sequence ensure that one can reach a marking $\marqp_j$ given by $\marq_j \fire{\sigma} \marqp_j$ such that $\marq_j \prec \marqp_j$.
Because of the constant effect of transitions, we know that $\forall q \in Q : \marq_j(q) < \marqp_j(q)$.
The unboundedness of the places in $Q$ follows.
\todo{A more formal proof that the existence of an increasing self-covering sequence is a necessary and sufficient condition for unboundedness on places is either to be done here or to reference.}



An \omark is a way to represent a set of markings which have the same number of tokens in some places, and may have any number of tokens, potentially an infinity, in the other places.
They are useful to effectively represent potentially infinite downward-closed set, like a coverability set or the sets of markings of an increasing self-covering sequence.

\begin{defi}[\omark]
  We define $\omega$ to be such that:
  $\omega \notin \mathbb{N}$
  and for all constant $c \in \mathbb{N}$:
  \begin{itemize}
    \item $c \leq \omega$
    \item $\omega + c = \omega$
    \item $\omega - c = \omega$
  \end{itemize}

  \emph{An \omark} \marq over a set of places $P$ is a function $\marq : P \mapsto \mathbb{N} \cup \{\omega\}$ that associates $\marq(p)$ tokens to each place $p \in P$.

  With $\mathbb{P}$ a set of parameters, $\omega \notin \mathbb{P}$,
  \emph{a parametric \omark} \marq over a set of places $P$ is a function $\marq : P \mapsto \mathbb{N} \cup \mathbb{P} \cup \{\omega\}$ that associates $\marq(p)$ tokens to each place $p \in P$.
\end{defi}

Note that an \omark \marq is a parametric \omark where $\marq(p) \in \mathbb{N} \cup \{\omega\}$ for all places $p \in P$.
Similarly, a parametric marking \marq is a parametric \omark where $\marq(p) \neq \omega$ for all places $p \in P$.
As for parametric markings, we often refer to a parametric \omark simply as \omark.

Having an \omark $\marq \in \Cover(\tupleN)$ denotes that for all marking $\marq_1$ such that $\marq_1(p) = \marq(p) \forall p \in \{p \mid \marq(p) \neq \omega\}$, there exists an marking $\marq_2$ in the reachability set of \tupleN such that $\marq_1 \pleq \marq_2$.
Notice also that an \omark stands for one and only one downward closed set.

\subsection{Karp and Miller procedure}

The Karp and Miller algorithm \cite{Karp69} is a classical algorithm to compute a coverability set of an initialized \ac{PN}.
More precisely, it constructs a coverability tree and uses an acceleration function to systematically detect self-covering sequences, and thus ensure the termination.

\begin{defi}[Coverability tree]
  Given a \ac{PN} $\tupleN = \PTm$, a coverability tree $\mathcal{T}$ of \tupleN is a labelled tree $\mathcal{T} = \langle N, B, n_0, \Lambda\rangle$ where:
  \begin{itemize}
    \item $N$ is the set of nodes of the tree.%, is a set of \omark of \tupleN such that $\downc(N) = \Cover(\tupleN)$.
    \item $n_0 \in N$ is the root of the tree, \lang{i.e.} $\nexists n \in N$ such that $(n, n_0) \in B$.
    \item $\Lambda : N \mapsto (\mathbb{N} \cup \{\omega\})^{|P|}$ is a labelling function that associate to each node a \textomega-marking of \tupleN.
    \item $B \subseteq N \times N$, the set of edges, is such that:
      \begin{itemize}
        \item with $(n_1, n_2) \in N^2$, if there exists an edge $(n_1, n_2) \in B$ then there exists a sequence $\sigma$ of transitions of $T$ such that $\Lambda(n_1) \fire{\sigma} \Lambda(n_2)$, and
        \item for all node $n \in N \setminus \{n_0\}$, there exists a path from the root to $n$, that is, there exists a sequence of edges of $B$ of the form $((n_0, n_1), (n_1, n_2), \dots, (n_{i}, n))$, $i > 1$, and
        \item there is no cycles, that is, there is no sequences of edges of $B$ of the form $((n_1, n_2), (n_2, n_3), \dots, (n_i, n_1))$.
      \end{itemize}
  \end{itemize}
  and such that $\downc(\{\Lambda(n) \mid n \in N\}) = \Cover(\tupleN)$.
\end{defi}

To keep $N$ finite, the Karp and Miller procedure exploits the monotonicity of \acp{PN} to introduce \omark{}s through an \emph{acceleration function} $\KMAcc$.
This function takes a marking \marq to accelerate and a set of markings $S$ as a base \todo{?} for the acceleration and returns a marking $\marq_\omega$ such that:
\[
  \marq_\omega(p) = 
  \begin{cases}
    \omega    &\text{if } \exists \marqp \in S : \marqp \prec \marq \text{ and } \marqp(p) < \marq(p) \\
    \marq(p)  &\text{otherwise}
  \end{cases}
\]

We denote by $\mathscr{T}_n$ the path in the tree from the root to $n$.

The algorithm constructs the tree $\mathcal{T}$ as follows:
The root $n_0$ of the tree is labelled with \marqi.
A frontier $F$ is defined to be the set of unprocessed nodes of the tree and is initialised to $\{n_0\}$.
Then, while $F$ is non empty, a node $n$ is chosen to be processed:
it is removed from $F$ and, if there is no node $n'$ in $\mathscr{T}_n$ such that $\Lambda(n) = \Lambda(n')$, for all \omark in $\{\KMAcc(\marq, \mathscr{T}_n) \mid \marq \in \Post(\Lambda(n))\}$, a node labelled with $\KMAcc(\marq, \mathscr{T}_n)$ is added to the frontier and to the tree as a child of $n$.

The correctness and the termination of the algorithm lies on the strict monotonicity of \acp{PN}, and was proved by Karp and Miller in their work \cite{Karp69}.

The Karp and Miller tree has a lot of convenient properties, and allows to answer coverability and \todo{} problems.
Furthermore, the Karp and Miller procedure can easily be adapted to some parametric problems \cite{David17}, as we will show in section~\todo{}.

However, this tree, although finite, is often much larger than the minimal coverability set, and cannot be constructed in reasonable time.
As a consequence, many improvement were proposed, as well as other algorithms with different approaches.

\subsection{An efficient computation method of the coverability set of Petri nets}
\label{sec:eff}

\citep{Geeraerts07thesis, Geeraerts07} propose another approach to the computation of the coverability set.
It is not based on the Karp and Miller algorithm and is not an alternative to the Karp and Miller procedure in the sense that it does not allow to answer the same set of questions that the Karp and Miller tree solves.
However, this techniques solve coverability problems more efficiently in practice.

As in the Karp and Miller algorithm, an acceleration function exploits the strict monotonicity of \acp{PN} to allow termination.
But here, the acceleration of a marking is performed with only one marking as the base (instead of a set of marking).

To choose the base to use, the algorithm work on pair of \omarks.
Thus we will need the following definitions:

Given a pair of \omarks $(\marq_1, \marq_2)$, we define:
\begin{itemize}
  \item $\Postb((\marq_1, \marq_2)) = \{(\marq_1, \marqp), (\marq_2, \marqp) \mid \marqp \in \Post(\marq_2)\}$,
  \item with $\marq_1 \prec \marq_2$, $\Accelb(\marq_1, \marq_2) = \{(\marq_2, \KMAcc(\{\marq_1\}, \marq_2))\}$.
    $\Accelb(\marq_1, \marq_2)$ is not defined whenever $\marq_1 \nprec \marq_2$,
\end{itemize}

With $R$ a set of pair of markings, we define:
\begin{itemize}
  \item $\Postb(R) = \bigcup_{(\marq_1, \marq_2) \in R} \Postb((\marq_1, \marq_2))$
  \item $\Accelb(R) = \bigcup_{(\marq_1, \marq_2) \in R}^{\marq_1 \prec \marq_2} \Accelb((\marq_1, \marq_2))$
  \item $\Flatten(R) = \{\marq \mid \exists \marqp : (\marqp, \marq) \in R\}$
\end{itemize}

The efficient computation of the coverability set of the marked \ac{PN} $\NPTm$ lies on the sequence $\CovSeq(\tupleN) = (V_i)_{i \geq 0}$ of pair of \omarks, where, for all marked \ac{PN} $\tupleN$ we have:
\begin{gather*}
  V_0 = \{(\marqi, \marqi)\} \text{ and } \\
  \forall i \geq 1 : V_i = V_{i-1} \cup \Postb(V_{i-1}) \cup \Accelb(V_{i-1})
\end{gather*}

One can shows that,
first, for all node $n$ of the Karp and Miller tree, there exists a value $k \geq 0$ of $i$ such that $\Lambda(n) \in \Flatten(V_k)$,
second, all the markings produced by $\Postb$ and $\Accelb$ are in the coverability set of \tupleN.
\todo{Indeed...}

This two results lead us to the following lemma:
\begin{lemm}[\cite{Geeraerts07}]
  Given a marked \ac{PN} \tupleN such that $\CovSeq(\tupleN) = (V_i)_{i \geq 0}$,
  there exists $k \geq 0$ such that for all $l \in \{0, ..., k-1\}$ we have that $\downc(\Flatten(V_l)) \subset \downc(\Flatten(V_{l+1}))$
  and for all $l \geq k : \downc(\Flatten(V_l)) = \Cover(\tupleN)$.
\end{lemm}

Thus, the algorithm idea is to compute $\CovSeq$ until it stabilizes, \lang{i.e.} to the lowest $l$ such that $\downc(\Flatten(V_l)) = \downc(\Flatten(V_{l-1}))$ and to return $\downc(\Flatten(V_l))$. \todo{Est-ce que $V_l = V_{l-1}$?}

To perform it efficiently, one can use an order $\sqsubseteq$ on the pair of markings to keep only the highest pair with respect to this order.
Let us denote by $\ominus$ the componentwise difference between two markings and to extend it to \omarks.
Formally, given two \omarks $\marq_1$ and $\marq_2$ on a set of places $P$, $(\marq_1 \ominus \marq_2)(p)$ is defined for all $p \in P$ as:
\[
  \begin{cases}
    \omega & \text{ whenever } \marq_1(p) = \omega \\
    -\omega & \text{ whenever } \marq_2(p) = \omega \text{ and } \marq_1(p) \neq \omega \\
    \marq_1(p) - \marq_2(p) & \text{ otherwise}
  \end{cases}
\]

For a set of \omarks $S$, $\maxs(S) = \{ s \in S \mid \nexists s' \in S, s \sqsubseteq s'\}$ is the set of highest \omark of $S$ with respect to $\sqsubseteq$.

Now we can define $\sqsubseteq$.
Given two pairs $(\marq_1, \marq_2)$ and $(\marqp_1, \marqp_2)$ of \omarks over a set of places $P$:
\[
  (\marq_1, \marq_2) \sqsubseteq (\marqp_1, \marqp_2) \Leftrightarrow
  \begin{cases}
    & \marq_1 \pleq \marqp_1 \\
    \wedge & \marq_2 \pleq \marqp_2 \\
    \wedge & \forall p \in P : (\marq_2 \ominus \marq_1)(p) \leq (\marqp_2 \ominus \marqp_1)(p)
  \end{cases}
\]

This order has properties \citep{Geeraerts07} that allows to keep the sets of markings of $\CovSeq$ small.
Thus, one can compute $\Cover(\tupleN)$ of a \ac{PN} \NPTm by computing the sequence $(V_i)_{i \geq 0}$ defined below until $\downc(\Flatten(V_i)) = \downc(\Flatten(V_{i-1}))$.
\begin{gather*}
  V_0 = \{(\marqi, \marqi)\} \text{ and } \\
  \forall i \geq 1 : V_i = \maxs(V_{i-1} \cup \Postb(V_{i-1}) \cup \Accelb(V_{i-1}))
\end{gather*}

At the end, we have that $\downc(\Flatten(V_i)) = \Cover(\tupleN)$.

The correction and termination of the algorithm as well as useful properties of $\sqsubseteq$ are presented in \citep{Geeraerts07, Ganty09}.

\subsection{\ac{EEC} algorithm}
\label{sec:eec}

\citep{Geeraerts07thesis, Geeraerts06}

\removed{A backward algorithm \citep{Finkel90, Abdulla96}}
%\subsection{A backward algorithm \citep{Finkel90, Abdulla96}}
%
%We will now present an algorithm to solve the coverability problem for a marking \marq of a \ac{PN} $\tupleN = \PTm$.
%
%This algorithm was introduced by Abdulla \lang{et al.} \cite{Abdulla96} for well structured transition systems, a more general class of models which includes \acp{PN}.
%It is close of the one introduced earlier in \cite{Finkel90}.
%
%Recall the definition for a marking of being coverable.
%\coverability*
%
%For convenience, we will use an other equivalent definition.
%
%\begin{defi}[Coverability]
%  Given an initialized \ac{PN} \NPTm and an upward-closed set $U$, $U$ is said coverable if there exists a marking \marqp such that $\marqp \in U$ and $\marqi \fire{*} \marqp$.
%\end{defi}
%
%By choosing $\upc(\marq)$ as $U$, these two definitions set out the same instance of the coverability problem.
%With a set of markings considered in the first definition, $U$ may be the union of their upward-closure in the second.
%
%We say it is a backward algorithm in the sense that it is based on the computation of the set $\Pre^*(\marq)$ and answer by checking whether $\marqi \in \Pre^*(\marq)$; unlike a forward approach which would have calculated the reachability set and conclude by checking whether \marq was in it. In other words, it compute all the configurations that can reach $U$ in any number of steps.
%
%The calculation is a fixed point algorithm that compute the increasing sequence, for the inclusion relation, of sets of markings: $(R_n)_{n \in \mathbb{N}}$, with $R_0 = U$ and $R_{n+1} = \Pre(R_n) \cup R_n$.
%Thus, $R_n$ is the set of markings from which there exists a sequence of at most $n$ transitions which may be fired and that cover $U$.
%Because, with $U$ an upward-closed set of markings, $\Pre(U)$ is upward-closed too%
%\footnote{This is due to the monotonicity of \acp{PN}, \todo{see for example cite\{someone\}}},
%and because the union of two upward-closed sets is an upward-closed set,
%$R_n$ is upward-closed for all $n$.
%
%\todo{summarize correctness and termination from \cite{Abdulla96}}

% vim: set spell spelllang=en :

\section{Known results on \acp{PPN}}
By reduction from the halting problem as well as the counter boundedness problem, \cite{David17} has shown that \Ucov and \Ecov are undecidable on \ac{PPN}.
This motivates the introduction of two natural subclasses of \acp{PPN} where parametric coverability problems are decidable.

Namely,
PreT-PPNs are \acp{PPN} where parameters are used only in $\matI$;
PostT-PPNs are \acp{PPN} where parameters are used only in $\matO$.
\cite{David17} provides an adaptation of the Karp and Miller Algorithm to solve the \Ucov problem on PreT-\acp{PPN}, and another to solve the \Ecov problem on PostT-\acp{PPN}.

% vim: set spell spelllang=en :


\chapter{Contributions}
In this part we will study the possibilities to adapt the existing results for plain \ac{PN} to the \ac{PPN} model.

% vim: set spell spelllang=en :

\section{Parametric coverability problems and specific valuations}
We present four theorem that basically come from the fact that
higher valuations on output arcs and lower valuations on input arcs lead to greater markings and, to the contrary,
lower valuations on output arcs and higher valuations on input arcs restrict the covering set $\Post^*(\mari)$ set.

First, when the parameters are restricted to the output arcs, a higher valuation leads to markings that are greater and thus more permissive (\lang{i.e.} that enable more transitions).
Going further, we can state the following theorem:
\begin{theo}
  \label{theo:post-star-val}
  Given a PostT-\ac{PPN} \SPTPm and an upward-closed set $U$ of markings of \PPN, \[\EcovOp(\PPN, U) = \top \Leftrightarrow \covOp(v_*(\PPN), U) = \top\] where $v_*$ is the *-valuation that maps every parameter to $*$.
\end{theo}

This was stated in \cite{David17} without a formal proof.
$\covOp(v_*(\PPN), U) \Rightarrow \EcovOp(\PPN, U)$ is trivial.
We therefore provide a proof for the other direction.

\todo{notation: v of a parameter and v of a PPN}

\begin{proof}
  Let $v$ be a *-valuation of $\PPN$ such that $\covOp(v(\PPN), U) = \top$.
  If $v$ is $v_*$, we are done.

  If $v$ is not $v_*$, let $\sigma$ be a sequence of transitions that witnesses the property above: with $\mar \in U$, we have $\mari \fire{\sigma}_v \mar$.
  Since that the n'umber of transitions is finite and that parameters are restricted to output arcs, it is easy to see that any other valuation $v'$ that maps each parameter $p$ to a value greater or equal to $v(p)$ allows $\sigma$ to lead to a marking $\mar_1$ greater than $\mar$ according to $\preceq$.
  In particular, this implies that $\mari \fire{\sigma}_{v_*} \mar_2$ with $\mar \preceq \mar_2$.
  Thus, since $\mar \in U$ and $U$ is upward-closed, we have $\mar_2 \in U$.
\end{proof}

Second, on PostT-\acp{PN}, a lower valuation leads to markings that are lower and thus less permissive.
This leads us to this theorem:
\begin{theo}
  \label{theo:post-zero-val}
  Given a PostT-\ac{PPN} \SPTPm and an upward-closed set $U$ of markings of \PPN, \[\UcovOp(\PPN, U) = \top \Leftrightarrow \covOp(v_0(\PPN), U) = \top\] where $v_0$ is the valuation that maps every parameter to $0$.
\end{theo}

\(\UcovOp(\PPN, U) = \top \Rightarrow \covOp(v_0(\PPN), U) = \top\) is trivial.
The reasoning to prove the other direction is similar to the proof given for the \autoref{theo:post-star-val}.

\begin{proof}
  Let $\sigma$ be such that $\mari \fire{\sigma}_{v_0} \mar$ with $\mar \in U$.
  We prove that for any valuation $v$, $\mari \fire{\sigma}_v \marp$ with $\marp \in U$.

  Because the parameters are bound to the output arcs and since the codomain of a valuation is $\mathbb{N}$, for any transition $t$ and any marking $\mar_1$ of $\PPN$, $\mar_1 \fire{\sigma}_{v_0} \mar_2$ implies that:
  \begin{enumerate}
    \item $t$ is enabled in $\mar_1$ in any valuation $v$: $\mar_1 \fire{t}_v \mar_3$ (because the firing condition is left unchanged), and
    \item $t$ leads to a greater marking according to $\preceq$: $\mar_2 \preceq \mar_3$.
  \end{enumerate}

  By monotonicity of \acp{PN}, this apply to a sequence of transition too.
  Thus, $\mari \fire{\sigma}_v \marp$.
  And since $U$ is upward-closed, $\marp \in U$.
\end{proof}

On the other hand, the values of the input arcs define not only the number of tokens that are removed but also the firing condition and thus the increase in these values restricts the set of firable transitions and ultimately the covering set.

\begin{theo}
  \label{theo:pre-star-val}
  Given a PreT-\ac{PPN} \SPTPm and an upward-closed set $U$ of markings of \PPN, \[\UcovOp(\PPN, U) = \top \Leftrightarrow \covOp(v_*(\PPN), U) = \top\] where $v_*$ is the *-valuation that maps every parameter to $*$.
\end{theo}

\begin{proof}
  $\UcovOp(\PPN, U) \Rightarrow \covOp(v_*(\PPN), U)$ is trivial.

  To prove that $\covOp(v_*(\PPN), U) \Rightarrow \UcovOp(\PPN, U)$, let $\sigma$ be a sequence of transitions such that $\mari \fire{\sigma}_{v_*} \mar_*$ with $\mar_* \in U$.
  Let $t$ be the first transition of $\sigma$.
  Since $\forall c \in \mathbb{N}, c < *$, we see that:
  \begin{itemize}
    \item for any valuation $v$, $t$ is enabled in $\mari$: $\mari \fire{t}_{v}$, and
    \item for any valuation $v$, any marking $\mar', \mar'' \text{ and } \mar''_*$ and any transition $t'$ such that $\mar' \fire{t'}_{v_*} \mar''_*$ and $\mar' \fire{t'}_v \mar''$ we have: $\mar''_* \preceq \mar''$.
  \end{itemize}

  Thus, by monotonicity of \acp{PN}, $\sigma$ is enabled in \mari for any valuation $v$ and, with $\mari \fire{\sigma}_v \mar$, we have $\mar_* \preceq \mar$.
  And since $U$ is upward-closed, $\mar \in U$.
\end{proof}

Finally, regarding values of input arcs, the smallest values are the best to reach high markings:

\begin{theo}
  \label{theo:pre-zero-val}
  Given a PreT-\ac{PPN} \SPTPm and an upward-closed set $U$ of markings of \PPN, \[\EcovOp(\PPN, U) = \top \Leftrightarrow \covOp(v_0(\PPN), U) = \top\] where $v_0$ is the valuation that maps every parameter to $0$.
\end{theo}

\begin{proof}
  $\covOp(v_0(\PPN), U) \Rightarrow \EcovOp(\PPN, U)$ is trivial.

  To prove that $\EcovOp(\PPN, U) \Rightarrow \covOp(v_0(\PPN), U)$, let $v$ be a valuation that allows to cover $U$ and $\sigma$ be a sequence of transitions such that $\mari \fire{\sigma}_v \mar$ with $\mar \in U$.
  Note that:
  \begin{itemize}
    \item all transition of \PPN enabled in a marking under any valuation are enabled in this marking under the valuation $v_0$, and
    \item for any valuation $v'$, any marking $\mar', \mar'' \text{ and } \mar''_{v_0}$ and any transition $t'$ such that $\mar' \fire{t'}_{v_0} \mar''_{v_0}$ and $\mar' \fire{t'}_{v'} \mar''$ we have: $\mar'' \preceq \mar''_{v_0}$.
  \end{itemize}

  Thus, using these observations on $\sigma$, and by monotonicity of \acp{PN}, we see that $\sigma$ is enabled in \mari under $v_0$ and, with $\mari \fire{\sigma}_{v_0} \mar_{v_0}$, we have $\mar \preceq \mar_{v_0}$.
  Since $U$ is upward-closed, $\mar \in U$.
\end{proof}

These different results in hand, it will be easier to give the following adaptations of known algorithms to the parametric coverability problems.

\section{Adapting the general backward algorithm}
The backward algorithm does not seem to be easy to adapt to the parametric coverability problems.
Indeed, it works only thanks to the iteration of the $\Pre$ operator.

In the case of PostT-\ac{PPN} and of the \Ecov problem, the idea would be to look for \mari from the markings to cover.
We have seen that, with a forward approach, one can consider only high valuations, that are better because they allow to cover more markings: the search from $\mari$ may go ``too far'' without being an issue.
Here however a high valuation may lead to a premature stop in a backward search, because it restricts ``more quickly'' the set of transitions that is enabled at each step.
In other words, contrary to forward approach, the backward approach has the deals with the 0 bound of the transition precondition. 

However, it may be the case that one can answer the \Ucov problem by applying the backward algorithm on the plain \ac{PN} obtained by the 0-valuation of all its parameter.
This will be a direction of our study.

The situation is symmetrical for PreT-\ac{PPN}, where the algorithm does not seem to be of much help for the \Ucov problem but where considering a 0-valuation may lead to a result about the \Ecov problem.

% vim: set spell spelllang=en :

\section{Adapting \ac{EEC}}% for PreT-\acp{PPN}}
In this section we provide an adaption of \ac{EEC} to work with \acp{PPN}.
It is inspired by the \todo{KM\textsuperscript{+}} algorithm from \cite{David17}.

The algorithm is the same as the one presented in the section~\ref{sec:eec}, with the difference that the condition to add a marking to the over-approximation is more restrictive.


% vim: set spell spelllang=en :

\section{Adapting Geeraerts method to PostT-\acp{PPN}}
Geeraerts method is much more complex.
However, as it uses both a Post operator and an acceleration function, it seems to be easily adapted to \ac{PPN} in the same way that the Karp and Miller algorithm was adapted.

% vim: set spell spelllang=en :


\vspace*{0.5cm}
\acresetall
\chapter*{Conclusions}

In the first chapter we defined the context of our study: \ac{PN} and \ac{PPN} and the coverability problems on them.
The second chapter presented the known results as a basis for our study.
The last chapter give the direction of our study and the result we would like to include in the last version of the work.

\appendix

\backmatter

\printindex % use makeindex to generate the index

\bibliographystyle{plain}

\bibliography{info} %use bibtex to generate the bibliography

\end{document}

% vim: set spell spelllang=en :
