\subsection{Interests of \acp{PPN}}

\todo{sources and examples}

Today \acp{PN} are used in a wide range of areas.
They are commonly used either to design a safe system, or to verify an existing one.
These uses require that the system is complete.
That is, for the design of a model, it must be entirely designed to be analyzable.
On the other hand, when checking an existing system, if a desired property does not hold, the correction must be made ``by hand''.

With the introduction of parameters some variables unknown at the design stage can be integrated into the model without having to be set arbitrarily. Moreover, if during the verification a desired property turns out not to hold, it is possible to check if the change of parameters alone can solve the problem, or if the Petri net structure must be changed too. Going further, the use of parameters in the model can allow to determine ``the safest values'' for a system, or to synthesize the values that allow to respect a given strategy.

We can therefore say that parameters can simplify the \emph{design} of a system. Indeed, since it is possible to keep unknown values, modelling can be done step by step, with the possibility to check the model at each step.
In addition, the design can be partially automated by parameter synthesis.
This approach gives a new interest in this model in fields as varied as chemistry, construction processes, financial loans...
\cite{David17} contains good illustrative examples.

There are also many advantages of using parameters when it comes to \emph{verification}.
For example, they allow to verify some properties simultaneously on many systems that differs only by parameters values.

\subsection{Interest of the coverability problem in \acp{PPN}}

\todo{sources and examples}

As it provide evidence of safety properties on the studied systems, coverability problem is of primary interest in system design and verification. Therefore, for the reasons given in the previous section, it is worth being able to solve it efficiently on \acp{PPN}.

To give a more concrete intuition on the interest, consider a system that execute a \emph{task} for others systems.
At each instant (whatever an instant is), the system may receive requests to perform the task from many other systems. We say that each request creates a \emph{job}.
We would like to have a system that is not too expensive to implement, but also capable of completing the tasks quickly enough.
For this we make our system capable of performing $a$ jobs at the same time, keeping $a$ as low as possible to reduce costs.
This system may be modeled as shown on the figure~\ref{fig:parametric-petri-net-example} with $\mari = (0, a, 0)$ as initial marking.
$p_1$ represents the job queue and $p_3$ the execution unit.\\
We can now formally verify that, whatever the parameter values, the execution unit will not receive more than $a$ jobs to perform at the same time, that is an instance of the \Ecov for the marking $(0, 0, a+1)$. Indeed, it is easy to see that $\Post^*((0, a, 0)) = \{(i, j, k) \mid i, j \text{ and } k \in \mathbb{N}, j + k = a\}$.
\todo{maybe remove part on `$a$ as low as possible'}

Before recalling the known results on \ac{PPN} and plain \ac{PN} that will be useful for our study, let us give a brief overview of some work already done on \acp{PPN}.

% vim: set spell spelllang=en :
