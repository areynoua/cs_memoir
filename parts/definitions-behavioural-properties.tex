The markings basically indicate the state of the system.
Knowing if an initialized \ac{PN} may reach a given marking, that represents for example a bad state, is therefore essential to check properties of the modelled system.
This is the \emph{reachability problem}.

\begin{defi}[Reachability]
  Given an initialized \ac{PN} $\defPN$ and a marking \mar of \namePN, \mar is said reachable if $\mari \fire{*} \mar$.
\end{defi}

\begin{defi}[Place boundedness]
  \label{defi:place-boundedness}
  Given a marked PN $\defPN$,
  a place $p \in \places$ such that $\exists c \in \naturals : \nexists \mar \in \posts{\mar_0}, \mar(p) > c$ is said to be bounded.
  That is, if there exists an upper bound on the number of tokens in the place $p$ in the set of the reachable markings.
\end{defi}

However, the verification of safety properties are more often reduced to a \emph{coverability problem}, that is essentially asking if an initialized \ac{PN} can reach or exceed a given marking.

\begin{restatable}[Coverability]{defi}{coverability}
  Given an initialized \ac{PN} $\defPN$ and a marking \mar of \namePN, \mar is said coverable if there exists a marking \marp such that $\mar \preceq \marp$ and $\mari \fire{*} \marp$.

  A set of markings is said coverable whenever one of its markings is coverable.
\end{restatable}

\begin{defi}[Coverability problem]
  Given an initialized \ac{PN} $\defPN$ and a set $\markings$ of markings of \namePN, determine whether $\exists \mar \in \markings \text{ and } \marp \in \posts{\mari} \text{ such that } \mar \preceq \marp$.

  The coverability problem for a marking \mar is the coverability problem for the singleton $\{\mar\}$.
\end{defi}

The behaviour of a \ac{PPN} is defined by the behaviours of all the \acp{PN} that can be obtained by a valuation of its parameters.
So, for an initialized \ac{PPN} $\namePPN$, the coverability problem may be declined in an existential and an universal form.
The existential coverability problem (\Ecov) ask if there exists a valuation $\val$ such that \mar is coverable.
The universal coverability problem (\Ucov) ask if \mar is coverable for all valuations $\val$.

\begin{defi}[Universal and existential coverability problems]
  Given a \ac{PPN} $\defPPN$ and a set $\markings$ of non-parametric markings of $\namePPN$
  \begin{itemize}
    \item the \emph{existential coverability problem} ask if there is a valuation $\val$ for $\parameters$ such that $\markings$ is coverable,
    \item the \emph{universal   coverability problem} ask if $\markings$ is coverable for all valuations of $\parameters$.
  \end{itemize}
\end{defi}

For the sake of simplicity, we will denote that the instance of the coverability problem on the \ac{PN} \namePN for the set $\setm$ of markings of \namePN is positive by: $\cov{\namePN}{ \setm} = \top$, and is negative by $\cov{\namePN}{ \setm} = \bot$. The same apply to \Ecov and \Ucov through the operators $\EcovOp$ and $\UcovOp$ respectively.
