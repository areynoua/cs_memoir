The use of parameters in formal verification systems is a well-developed topic in the literature.

With regard  to \acp{PN}, parameters have been introduced with many roles.
Some works, like \cite{Christensen97}, use parameters as places or transitions, for example to make it possible to change a place into a more complex subnet and thus allow different levels of abstractions to be considered.
In \cite{Lindqvist91}, parameters are used on the markings to obtain concise parametrised reachability trees, but not to realize formal verifications on these parametric systems.

\cite{Badouel99} introduces parameters as the weight of arcs to model changes in a system.
The parameters have a finite valuation domain and verifications are performed on these parametrised systems.
In \cite{watel2017parameterized}, parametrised arc weights are used to solve optimisations problem, specifically in chemical reactions.

Systems with quantitative parameters with infinite valuation domains are analysed in \cite{Abdulla13}.
Similarly, \cite{Marsan94} studies \acp{PN} with parametric initial markings which represent sets of possible initial markings.
%(called \ac{PN} models in contrast to \ac{PN} systems where the initial marking does not have parameters).

Our work is in the line with \cite{David17} which use discrete parameters as arc weights as well as in the markings.
\cite{David17} provides a proof for the non decidability of \Ucov and \Ecov, and define several subclasses of \ac{PPN} for which these problems are decidable.

Close to this \ac{PPN} model, \opn \citep{Geeraerts15} allows input and output weights to be $\omega$.
In this case, the transition consumes or produces a non-deterministic number of tokens.
%Note that in this model, the transitions does not have a constant effect any more.
