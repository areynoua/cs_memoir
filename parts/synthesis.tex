\todo{change introduction and many other parts as this was not planned at first}

The use of parameter opens the way to the computation of parameters values ensuring some properties in the system.
This is called the \emph{synthesis problem}.

\begin{defi}[Coverability synthesis problem]
  \label{defi:cov-synth-prblm}
  Given a \ac{PPN} $\defPPN$ and an upward-closed set $\ucs$ of markings on $\places$,
  compute the set $\vcov{\namePPN}{\ucs}$ of valuations of $\parameters$
  such that
  $\cov{\val[\namePPN]}{\ucs} = \top \Leftrightarrow \val \in \vcov{\namePPN}{\ucs}$.
\end{defi}

\subsection{$\vback$ to solve the coverability synthesis problem on P-PPNs}

We present a simple algorithm based on $\back$ to compute $\Vcov$ for all P-PPNs \namePPN and goal sets of markings.
%It is close to the algorithm introduced in \cite{David17} for the ``direct computation of the coverability synthesis set for PreT-PPNs''.

Given a P-PPN $\defPPN$ and a goal upward-closed set of markings $\ucs$,
first compute $\pres{\ucs}$.
Then we have
\(
  \vcov{\namePPN}{\ucs} =
  \setComp{\val}{
    \val[\mar_0] \in \pres{\ucs}
    %\cap \downc{\val_\omega(\mar_0)}
  }
\).

The termination and the correctness follow from the termination and the correctness of $\back$.

%The termination comes from the termination of $\back$.
%
%\begin{lemm}[Correctness]
%  Given a P-PPN \namePPN and an upward-closed set $\ucs$ of markings of \namePPN,
%  then
%  %this two propositions are equivalent:\todo{check words proposition and equivalent}
%  \[
%    \text{(1) } \cov{\val[\namePPN]}{\ucs}
%    \Leftrightarrow
%    \text{(2) } \val[\mar_0] \in \pres{\ucs} \cap \downc{\val_\omega(\mar_0)}
%  \]
%\end{lemm}
%
%\begin{proof}
%  We prove (1) $\Rightarrow$ (2) by contradiction.
%  Suppose $\cov{\val[\namePPN]}{\ucs} = \top$ and
%  $\val[\mar_0] \notin \pres{\ucs} \cap \downc{\val_\omega(\mar_0)}$.
%  By definition of $\omega$,
%  for all valuation $\val$,
%  $\val[\mar_0] \in \downc{\val_\omega(\mar_0)}$.
%  Thus, $\val[\mar_0] \notin \pres{\ucs}$.
%  However, from the proof of the correctness of $\back$ \todo{cite/ref}, we have that
%  $\cov{\val[\namePPN]}{\ucs} \Leftrightarrow \val[\mar_0] \in \pres{\ucs}$.
%  This contradicts our initial supposition.
%
%  To prove that (2) $\Rightarrow$ (1), suppose that there exists $\val$ such that
%  $\val[\mar_0] \in \pres{\ucs} \cap \downc{\val_\omega(\mar_0)}$.
%  Thus, $\val[\mar_0] \in \pres{\ucs}$.
%  From the proof of the correctness of $\back$ \todo{cite/ref}, we have that
%  $\cov{\val[\namePPN]}{\ucs} \Leftrightarrow \val[\mar_0] \in \pres{\ucs}$.
%\end{proof}

\paragraph{Effectiveness}
The $\Vcov$ set may be infinite.
However,
%by using the vector representation of valuations,
by \Cref{theo:pre-upc} we obtain an upward-closed set that can be represented by its minimal elements (\Cref{theo:upward-closed-set-representation}).
%$\downc{\val_\omega(\mar_0)}$ is downward-closed and

Indeed,
$\pres{\ucs}$ are the markings that allow to cover $\ucs$.
%From this set, those which agree with $\mar_0$ on its non-parametric places form an over-approximation of 
Since $\ucs$ is upward-closed, by monotonicity, $\pres{\ucs}$ is upward-closed too (\Cref{theo:pre-upc}) \todo{need a proof?}.
Let $\minp{\pres{\ucs}} = \{\mar_1, \dots, \mar_k\}$.
Then,
\begin{align*}
  \mar_0 \in \pres{\ucs} \Leftrightarrow \ &
    \mar_1 \preceq \mar_0 \vee \dots \vee \mar_k \preceq \mar_0 \\
  \Leftrightarrow \ &
    (\mar_1(p_1) \leq \mar_0(p_1) \wedge \mar_1(p_2) \leq \mar_0(p_2) \wedge \dots) \\
  &\vee \dots \\
  &\vee
    (\mar_k(p_1) \leq \mar_0(p_1) \wedge \mar_k(p_2) \leq \mar_0(p_2) \wedge \dots)
\end{align*}

From this expression, we can remove the conjunctions that can not be true for any valuation, that is, those related to a minimal marking that can not be covered by $\mar_0$ due to a non-parametric place: $\mar \in \minp{\pres{\ucs}}$ with
\(
  \exists p \in \places : \mar_0(p) \notin \parameters \text{ and } \mar(p) > \mar_0(p)
\).
The other conjunctions provide us with the minimal valuations of $\vcov{\namePPN}{\ucs}$:
for each conjunction, each contained proposition is either a tautology (whenever $\mar_0(p) \notin \parameters$), or a lower bound on a parameter (whenever $\mar_0(p) \in \parameters$).
\todo{It is easy to see that the set of valuation thus obtained is correct and complete.}

\todo{thanks to the bisimulation relation described on section .. one can use this method to compute vcov on a PreTPPN}.

\todo{thanks to the cosimulation relation described on section .. one can use this method to compute .. on a PostTPPN}

%%\paragraph{Effectivness}
%The $\Vcov$ set may be infinite.
%However,
%by using the vector representation of valuations, we obtain an upward-closed set that can be represented by its minimal elements.
%%$\downc{\val_\omega(\mar_0)}$ is downward-closed and
%
%Indeed,
%$\pres{\ucs}$ are the markings that allow to cover $\ucs$.
%%From this set, those which agree with $\mar_0$ on its non-parametric places form an over-approximation of 
%Since $\ucs$ is upward-closed, by monotonicity, $\pres{\ucs}$ is upward-closed too \todo{need a proof?}.
%Let $\minp{\pres{\ucs}} = \{\mar_1, \dots, \mar_k\}$.
%That is,
%\begin{align*}
%  \mar \in \pres{\ucs} \Leftrightarrow \ &
%    \mar_1 \preceq \mar \vee \dots \vee \mar_k \preceq \mar \\
%  \Leftrightarrow \ &
%    (\mar_1(p_1) \leq \mar(p_1) \wedge \mar_1(p_2) \leq \mar(p_2) \wedge \dots) \\
%  &\vee
%    (\mar_2(p_1) \leq \mar(p_1) \wedge \mar_2(p_2) \leq \mar(p_2) \wedge \dots) \\
%  &\vee \dots
%\end{align*}
%$\forall \param \in \parameters$,
%let $\mar_0^{-1}(\param)$ be the set places $p$ such that $\mar_0(p) = \param$.
%Without loss of generality, consider that the parameters $\parameters = \{\param_1, \dots, \param_i\}$ in $\mar_0$ are used in, and only in, the $j$ first places.
%Then,
%\begin{align*}
%  \val[\mar_0] \in \pres{\ucs} \Leftrightarrow \ &
%    \mar_1 \preceq \val[\mar_0] \vee \dots \vee \mar_k \preceq \val[\mar_0] \\
%  \Leftrightarrow \ &
%    (\mar_1(p_1) \leq \val[\mar_0](p_1) \wedge \mar_1(p_2) \leq \val[\mar_0](p_2) \wedge \dots) \\
%  &\vee
%    (\mar_2(p_1) \leq \val[\mar_0](p_1) \wedge \mar_2(p_2) \leq \val[\mar_0](p_2) \wedge \dots) \\
%  &\vee \dots
%\end{align*}
%
%
%
%Since parameters are used on the initial marking only,
%we are looking for the valuations such that 
%
%
%\cref{lemm:wqo}
