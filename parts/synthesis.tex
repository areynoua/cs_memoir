\todo{change introduction and many other parts as this was not planned at first}

The use of parameter opens the way to the computation of parameters values ensuring some properties in the system.
This is called the \emph{synthesis problem}.

\begin{defi}[Coverability synthesis problem]
  \label{defi:cov-synth-prblm}
  Given a \ac{PPN} $\defPPN$ and an upward-closed set $\ucs$ of markings on $\places$,
  compute the set $\vcov{\namePPN}{\ucs}$ of valuations of $\parameters$
  such that
  $\cov{\val[\namePPN]}{\ucs} = \top \Leftrightarrow \val \in \vcov{\namePPN}{\ucs}$.
\end{defi}

\subsection{$\vback$ to solve the coverability synthesis problem}

\subsubsection{$\vback$ on P-PPNs}

We present a simple algorithm based on $\back$ to compute $\Vcov$ for all P-PPN \namePPN and goal sets of markings.
%It is close to the algorithm introduced in \cite{David17} for the ``direct computation of the coverability synthesis set for PreT-PPNs''.

Given a P-PPN $\defPPN$ and a goal upward-closed set of markings $\ucs$,
first compute $\pres{\ucs}$.
Then we have
\(
  \vcov{\namePPN}{\ucs} =
  \setComp{\val}{
    \val[\mar_0] \in \pres{\ucs}
    %\cap \downc{\val_\omega(\mar_0)}
  }
\).

The termination and the correctness follow from the termination and the correctness of $\back$.

%The termination comes from the termination of $\back$.
%
%\begin{lemm}[Correctness]
%  Given a P-PPN \namePPN and an upward-closed set $\ucs$ of markings of \namePPN,
%  then
%  %this two propositions are equivalent:\todo{check words proposition and equivalent}
%  \[
%    \text{(1) } \cov{\val[\namePPN]}{\ucs}
%    \Leftrightarrow
%    \text{(2) } \val[\mar_0] \in \pres{\ucs} \cap \downc{\val_\omega(\mar_0)}
%  \]
%\end{lemm}
%
%\begin{proof}
%  We prove (1) $\Rightarrow$ (2) by contradiction.
%  Suppose $\cov{\val[\namePPN]}{\ucs} = \top$ and
%  $\val[\mar_0] \notin \pres{\ucs} \cap \downc{\val_\omega(\mar_0)}$.
%  By definition of $\omega$,
%  for all valuation $\val$,
%  $\val[\mar_0] \in \downc{\val_\omega(\mar_0)}$.
%  Thus, $\val[\mar_0] \notin \pres{\ucs}$.
%  However, from the proof of the correctness of $\back$ \todo{cite/ref}, we have that
%  $\cov{\val[\namePPN]}{\ucs} \Leftrightarrow \val[\mar_0] \in \pres{\ucs}$.
%  This contradicts our initial supposition.
%
%  To prove that (2) $\Rightarrow$ (1), suppose that there exists $\val$ such that
%  $\val[\mar_0] \in \pres{\ucs} \cap \downc{\val_\omega(\mar_0)}$.
%  Thus, $\val[\mar_0] \in \pres{\ucs}$.
%  From the proof of the correctness of $\back$ \todo{cite/ref}, we have that
%  $\cov{\val[\namePPN]}{\ucs} \Leftrightarrow \val[\mar_0] \in \pres{\ucs}$.
%\end{proof}

\paragraph{Effectiveness}
The $\Vcov$ set may be infinite.
However,
%by using the vector representation of valuations,
by \Cref{theo:pre-upc} we obtain an upward-closed set that can be represented by its minimal elements (\Cref{theo:upward-closed-set-representation}).
%$\downc{\val_\omega(\mar_0)}$ is downward-closed and

Indeed,
$\pres{\ucs}$ are the markings that allow to cover $\ucs$.
%From this set, those which agree with $\mar_0$ on its non-parametric places form an over-approximation of
Since $\ucs$ is upward-closed, by monotonicity, $\pres{\ucs}$ is upward-closed too (\Cref{theo:pre-upc}) \todo{need a proof?}.
Let $\minp{\pres{\ucs}} = \{\mar_1, \dots, \mar_k\}$.
Then,
\begin{align*}
  \mar_0 \in \pres{\ucs} \Leftrightarrow \ &
    \mar_1 \preceq \mar_0 \vee \dots \vee \mar_k \preceq \mar_0 \\
  \Leftrightarrow \ &
    (\mar_1(p_1) \leq \mar_0(p_1) \wedge \mar_1(p_2) \leq \mar_0(p_2) \wedge \dots) \\
  &\vee \dots \\
  &\vee
    (\mar_k(p_1) \leq \mar_0(p_1) \wedge \mar_k(p_2) \leq \mar_0(p_2) \wedge \dots)
\end{align*}

From this expression, we can remove the conjunctions that can not be true for any valuation, that is, those related to a minimal marking that can not be covered by $\mar_0$ due to a non-parametric place: $\mar \in \minp{\pres{\ucs}}$ with
\(
  \exists p \in \places : \mar_0(p) \notin \parameters \text{ and } \mar(p) > \mar_0(p)
\).
The other conjunctions provide us with the minimal valuations of $\vcov{\namePPN}{\ucs}$:
for each conjunction, each contained proposition is either a tautology (whenever $\mar_0(p) \notin \parameters$), or a lower bound on a parameter (whenever $\mar_0(p) \in \parameters$).
\todo{It is easy to see that the set of valuation thus obtained is correct and complete.}

\todo{thanks to the bisimulation relation described on section .. one can use this method to compute vcov on a PreTPPN}.

%%\paragraph{Effectivness}
%The $\Vcov$ set may be infinite.
%However,
%by using the vector representation of valuations, we obtain an upward-closed set that can be represented by its minimal elements.
%%$\downc{\val_\omega(\mar_0)}$ is downward-closed and
%
%Indeed,
%$\pres{\ucs}$ are the markings that allow to cover $\ucs$.
%%From this set, those which agree with $\mar_0$ on its non-parametric places form an over-approximation of
%Since $\ucs$ is upward-closed, by monotonicity, $\pres{\ucs}$ is upward-closed too \todo{need a proof?}.
%Let $\minp{\pres{\ucs}} = \{\mar_1, \dots, \mar_k\}$.
%That is,
%\begin{align*}
%  \mar \in \pres{\ucs} \Leftrightarrow \ &
%    \mar_1 \preceq \mar \vee \dots \vee \mar_k \preceq \mar \\
%  \Leftrightarrow \ &
%    (\mar_1(p_1) \leq \mar(p_1) \wedge \mar_1(p_2) \leq \mar(p_2) \wedge \dots) \\
%  &\vee
%    (\mar_2(p_1) \leq \mar(p_1) \wedge \mar_2(p_2) \leq \mar(p_2) \wedge \dots) \\
%  &\vee \dots
%\end{align*}
%$\forall \param \in \parameters$,
%let $\mar_0^{-1}(\param)$ be the set places $p$ such that $\mar_0(p) = \param$.
%Without loss of generality, consider that the parameters $\parameters = \{\param_1, \dots, \param_i\}$ in $\mar_0$ are used in, and only in, the $j$ first places.
%Then,
%\begin{align*}
%  \val[\mar_0] \in \pres{\ucs} \Leftrightarrow \ &
%    \mar_1 \preceq \val[\mar_0] \vee \dots \vee \mar_k \preceq \val[\mar_0] \\
%  \Leftrightarrow \ &
%    (\mar_1(p_1) \leq \val[\mar_0](p_1) \wedge \mar_1(p_2) \leq \val[\mar_0](p_2) \wedge \dots) \\
%  &\vee
%    (\mar_2(p_1) \leq \val[\mar_0](p_1) \wedge \mar_2(p_2) \leq \val[\mar_0](p_2) \wedge \dots) \\
%  &\vee \dots
%\end{align*}
%
%
%
%Since parameters are used on the initial marking only,
%we are looking for the valuations such that
%
%
%\cref{lemm:wqo}

\subsubsection{$\vback$ on Post-PPNs}

Given a PostT-PPN $\defPPN[1]$,
one can construct a weakly co-similar P-PPN $\defPPN[2]$ as described in \Cref{sec:postt-ppn-to-p-ppn} and computes $\vback$ on it.
Technically, since the \namePPN[2] has more places than \namePPN[1], the domain of the markings is not the same for the two PNs.
The ``silent'' places added into \namePPN[2] are like an internal machinery and do not have a meaning for the studied system: we do not have requirements on its.
Thus we use as goal set for \namePPN[2] the upward-closed set that contains all the markings on $\places_2$ that match a marking in $\ucs$ on their common places:
$
  \ucse = \setComp
            {\mar}
            {\exists \mar' \in \ucs
            : \forall p \in \places_1
            : \mar(p) = \mar'(p)}
$.
Note that this set can be effectively computed:
\begin{align*}
  \mar' \in \minp{\ucs}
  \Leftrightarrow
  \mar \in \minp{\ucse} \\
  \text{with }
  \mar(p) = \begin{cases}
    \mar'(p)  &\tif p \in \places_1 \\
    0         &\tothrw
  \end{cases}
\end{align*}

Moreover, we define an equivalence relation
$\equiv \subseteq \naturals^{\card{\places_1}} \times \naturals^{\card{\places_2}}$
such that
$
  \mar_1 \equiv \mar_2
  \Leftrightarrow
  \forall p \in \places_1 \cap \places_2 : \mar_1(p) = \mar_2(p)
$ and we denote by $\rela_\equiv$ the relation on two sets of markings where $\equiv$ replace the usual component wise equality $=$ used in the relation $\rela$.
For example, $\setm_1 \subseteq_\equiv \setm_2$ denotes that the for all marking $\mar_1$ in $\setm_1$ there is a marking  $\mar_2$ in $\setm_2$ such that $\mar_1 \equiv \mar_2$.
In particular $\ucs =_\equiv \ucse$.

Surely, the co-similarity ensures that
$\vcov{\namePPN[1]}{\ucs} = \vcov{\namePPN[2]}{\ucse}$
and this result (namely: co-similarity preserves coverability) may be applicable to any \ac{WSTS}.
\todo{Ça m'ennuie de travailler dessus puisque ça a déjà sûrement été fait... mais je n'ai pas trouvé où}

%Here we establish $\vcov{\namePPN[1]}{\ucs} = \vcov{\namePPN[2]}{\ucs}$, from which the correctness of the procedure follows,
%\todo{bon bref faut formuler qu'on fait ça mais seulement dans notre cas précis: avec la construction donnée}

We provide a proof restricted to our specific case.

\begin{proof}[\proofname\ of correctness of $\vback$ on PostT-PPN]
  To prove that this method is correct is to prove that $\vcov{\namePPN[1]}{\ucs} = \vcov{\namePPN[2]}{\ucse}$,
  where $\namePPN[1]$ is the studied PostT-PPN and $\namePPN[2]$ is the P-PPN constructed like described in \Cref{sec:postt-ppn-to-p-ppn}.

  First we show that $\vcov{\namePPN[1]}{\ucs} \subseteq \vcov{\namePPN[2]}{\ucse}$.
  Since
  for any valuation $\val$,
  $\val[{\namePPN[2]}]$ simulates $\val[{\namePPN[1]}]$,
  we know that
  for any sequence of transitions $\seqt[1]$
  leading to a marking $\mar_1$ in any $\val[{\namePPN[1]}]$
  :
  $\mar_{0,1} \fire[{\val[{\namePPN[1]}]}]{\seqt[1]} \mar_1$,
  there exists a sequence $\seqt[2]$
  such that
  $\mar_{0,2} \fire[{\val[{\namePPN[2]}]}]{\seqt[2]} \mar_2$
  with $\mar_1 \simeq \mar_2$.
  It follows that $\vcov{\namePPN[1]}{\ucs} \subseteq \vcov{\namePPN[2]}{\set}$ for a set $\set$ isomorphic to $\ucs$.
  Moreover, looking at the simulation relation used in the proof of \Cref{theo:simulation-p-ppn-postt-ppn}, we observe that the markings involved in the simulation relation are identical on the places of \namePPN[1].
  This implies $\set =_\equiv \ucs$
  and $\vcov{\namePPN[1]}{\ucs} \subseteq \vcov{\namePPN[2]}{\ucse}$.

  Secondly, $\vcov{\namePPN[2]}{\ucs} \subseteq \vcov{\namePPN[1]}{\ucse}$.
  Indeed,
  we may follow a similar reasoning and observe that
  the simulation relation used in proof of \Cref{theo:simulation-postt-ppn-p-ppn} ensures
  $\mar_2 \simeq \mar_1 \implies \forall p \in \places_1 : \mar_1(p) \geq \mar_2(p)$.
\end{proof}

%Here we show how the co-similarity \Cref{theo:cosimulation-p-ppn-postt-ppn} allows to conclude to the correctness of the method by deducing
%$\vcov{\namePPN[2]}{\ucs} \subseteq \vcov{\namePPN[1]}{\ucs}$ from \Cref{theo:simulation-p-ppn-postt-ppn} and
%$\vcov{\namePPN[1]}{\ucs} \subseteq \vcov{\namePPN[2]}{\ucs}$ from        \Cref{theo:simulation-postt-ppn-p-ppn}.
%
%The weak simulation relation between \namePPN[2] and \namePPN[1]
