We have seen in \Cref{sec:some-uses-of-pn} that \acp{PN} are used in a wide range of areas.
They are commonly used either to design safe systems or to verify existing ones.
These uses require that the modelling of the system be complete.
That is, for the design of a model, it must be entirely designed to be analysable.
On the other hand, when checking an existing system, if a desired property does not hold, the correction must be done by hand.

With the introduction of parameters, some variables unknown at the design stage can be integrated into the model without having to be arbitrarily set.
Moreover, if during the verification a desired property is found not to hold, it is possible to check if the change of parameters alone can solve the problem, or if the structure of the Petri net also needs to be modified.
Going further, the use of parameters in the model can be used to determine the ``safest values'' for a system, or to synthesise the values that allow a given strategy to be followed.

We can therefore say that parameters can simplify the \emph{design} of a system. Indeed, since it is possible to keep unknown values, modelling can be done step by step, with the possibility to check the model at each step.
In addition, the design can be partially automated by parameter synthesis.
This approach gives a new interest in this model in fields as varied as chemistry, construction processes, financial loans\etc
\cite{David17} contains an illustrative example about a production line where \acp{PPN} allow to easily express the constraints and the optimisation objective.

There are also many advantages of using parameters when it comes to \emph{verification}.
For example, it allows to verify some properties simultaneously on many systems that differs only by parameters values.
See for example the ``Financial Loan'' example from \cite{David17}.
Here the personal loan contract is modelled as an abstract \ac{PPN} and each signature of an actual contract is a concrete \ac{PN} obtained by defining the values of the \ac{PPN}'s parameters.
