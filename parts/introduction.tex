\acp{PN} are a mathematical and graphical model introduced by Carl Adam Petri in 1962 \citep{Petri62,Petri66}. It was successfully used to analyse systems in a wide range of domains, especially for the formal verification of asynchronous systems.

In their standard definition, \acp{PN} are instantiated through many natural numbers which may represent, for example, the amount of resource needed for a given action to be carried out.

The introduction of parameters into the model to avoid the need to state these values explicitly%
\footnote{One can find in the literature many other way to use parameters in \acp{PN}. For example, place and~/~or transitions may also be parameters in order to dynamically change the network structure, like in \cite{Christensen97}.}
may have several benefits:
it may allow to perform analysis of a whole family of \acp{PN} in an efficient way, like in \cite{Abdulla13}, or to model dynamic changes in the system, as introduced by \cite{Badouel99} as a subclass of reconfigurable nets.

The use of parameters increases the modelling power of \acp{PN} but also make some basic coverability problems undecidable in the general case \cite{David17}.

We adopt the parametric Petri net model introduced by \cite{David17}, which seems the most general, and we study the existing results and algorithms for plain Petri nets to find out whether or not they still hold or how to adapt them to the parametric model.

The rest of the document is as follows:
In this first part, we define the plain Petri net model (\lang{i.e.} the classical one) and the parametric model.
We then briefly motivate our study, \todo{give concrete examples of applications,} and give an overview of the previous works on parametrisation of \acp{PN}.
Finally we place the \ac{PN} model in a broader model family: the \acp{WSTS}.\\
In a second part, we recall, first, the classical results that we will study on this new model, second, the results already obtained for the parameterized Petri net model as we have defined them.\\
Then, we focus on the parametric Petri net model to establish whether the results related to the coverability problem in the plain Petri net model still hold or if the algorithms may be adapted to this new model.

\acresetall

% vim: set spell spelllang=en :
