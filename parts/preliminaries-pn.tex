We now present the results relating to the coverability problem on the plain \acp{PN} which seem to us the most interesting.

To introduce these results, we need some additional definitions.
They will be given for plain \acp{PN}, but most of them are naturally extended to \acp{PPN}.

The \emph{coverability set} of an initialized \ac{PN} is an over-approximation of the reachability set that is precise enough to solve the coverability problem, and is, therefore, interesting for our study.
In order to define it formally, we need to know about the upward and downward closure of a (set of) marking(s).

\begin{defi}[Upward- and downward-closure on markings]
  Let $S \subseteq \mathbb{N}^{|P|}$ be a set of markings on the places $P$:
  \begin{itemize}
    \item The \emph{upward-closure} of $S$, noted $\upc(S)$, is the set
      $\{\marq \in \mathbb{N}^{|P|} \mid \exists \marqp \in S : \marqp \pleq \marq\}$,
    \item The \emph{downward-closure} of $S$, noted $\downc(S)$, is the set
      $\{\marq \in \mathbb{N}^{|P|} \mid \exists \marqp \in S : \marq \pleq \marqp\}$.
  \end{itemize}
  The closure of a marking \marq is the closure of the singleton $\{\marq\}$.
\end{defi}

For instance, with $\marq = (1, 2, 3)$, we have that its upward-closure is $\upc(\marq) = \{(i, j, k) \mid i \geq 1, j \geq 2, k \geq 3\}$.

\begin{defi}[Upward- and downward-closed set of markings]
  A set $S$ of markings is said \emph{upward-closed} if $S = \upc(S)$.
  It is said \emph{downward-closed} if $S = \downc(S)$.
\end{defi}

\begin{defi}[Coverability set \citep{Finkel87,Finkel90}]
  Given an initialized \ac{PN} $\tupleN = \PTm$, a \emph{coverability set} $S$ of \tupleN is a set of markings such that $\downc(S) = \downc(\Post^*(\marqi))$.
  
  The minimal coverability set is thus $\downc(\Post^*(\marqi))$ \todo{one-line demonstration required?} and is noted $\Cover(\tupleN)$.
\end{defi}

\todo{examples}



Sometimes the number of token in a place is unbounded (\textit{c.f.} the place boundedness problem).
Therefore, the reachability and coverability sets are infinite.
This is due to the existence of an increasing self-covering sequence.
\begin{defi}[Self-covering sequence]
  Given an initialized \ac{PN} $\tupleN = \PTm$,
  a self-covering sequence is a sequence of the form:
  \[
    \marqi \fire{\rho} \marq_i \fire{\sigma} \marq_j
  \]
  with $\rho$ and $\sigma$ two sequences of transitions of $T$
  and with $\marq_i \pleq \marq_j$.
\end{defi}

Note that, since $\marq_i \pleq \marq_j$, $\sigma$ is firable from $\marq_j$.
In addition, the monotonicity of \acp{PN} ensure that, with $\marqp_j$ given by $\marq_j \fire{\sigma} \marqp_j$, we have $\marq_j \pleq \marqp_j$.
Thus, we see that it is a sufficient condition for the \emph{non-termination} of the system (the system may be able to fire transitions infinitely often).
In fact, because $\pleq$ is a well quasi order (Lemma~\ref{lemm:wqo}), one can find in any infinite sequence $\marqi \fire{} \marq_1 \fire{} \dots$ two markings $\marq_i$ and $\marq_j$ such that $\marq_i \pleq \marq_j$.
And so, any infinite sequence is self-covering, and the existence of such a sequence is also necessary for the non-termination of the system.

\begin{defi}[Increasing self-covering sequence]
  Given an initialized \ac{PN} $\tupleN = \PTm$,
  an increasing self-covering sequence is a sequence of the form:
  \[
    \marqi \fire{\rho} \marq_i \fire{\sigma} \marq_j
  \]
  with $\rho$ and $\sigma$ two sequences of transitions of $T$
  and with $\marq_i \prec \marq_j$.
\end{defi}

Let $Q \subseteq P$ be the set of places $Q = \{q \in P \mid \marq_i(q) < \marq_j(q)\}$.
$Q \neq \emptyset$ since $\marq_i \prec \marq_j$.

With a similar reasoning to the above, we see that having such a sequence ensure that one can reach a marking $\marqp_j$ given by $\marq_j \fire{\sigma} \marqp_j$ such that $\marq_j \prec \marqp_j$.
Because of the constant effect of transitions, we know that $\forall q \in Q : \marq_j(q) < \marqp_j(q)$.
The unboundedness of the places in $Q$ follows.
\todo{A more formal proof that the existence of an increasing self-covering sequence is a necessary and sufficient condition for unboundedness on places is either to be done here or to reference.}



An \omark is a way to represent a set of markings which have the same number of tokens in some places, and may have any number of tokens, potentially an infinity, in the other places.
They are useful to effectively represent potentially infinite downward-closed set, like a coverability set or the sets of markings of an increasing self-covering sequence.

\begin{defi}[\omark]
  We define $\omega$ to be such that:
  $\omega \notin \mathbb{N}$
  and for all constant $c \in \mathbb{N}$:
  \begin{itemize}
    \item $c \leq \omega$
    \item $\omega + c = \omega$
    \item $\omega - c = \omega$
  \end{itemize}

  \emph{An \omark} \marq over a set of places $P$ is a function $\marq : P \mapsto \mathbb{N} \cup \{\omega\}$ that associates $\marq(p)$ tokens to each place $p \in P$.

  With $\mathbb{P}$ a set of parameters, $\omega \notin \mathbb{P}$,
  \emph{a parametric \omark} \marq over a set of places $P$ is a function $\marq : P \mapsto \mathbb{N} \cup \mathbb{P} \cup \{\omega\}$ that associates $\marq(p)$ tokens to each place $p \in P$.
\end{defi}

Note that an \omark \marq is a parametric \omark where $\marq(p) \in \mathbb{N} \cup \{\omega\}$ for all places $p \in P$.
Similarly, a parametric marking \marq is a parametric \omark where $\marq(p) \neq \omega$ for all places $p \in P$.
As for parametric markings, we often refer to a parametric \omark simply as \omark.

Having an \omark $\marq \in \Cover(\tupleN)$ denotes that for all marking $\marq_1$ such that $\marq_1(p) = \marq(p) \forall p \in \{p \mid \marq(p) \neq \omega\}$, there exists an marking $\marq_2$ in the reachability set of \tupleN such that $\marq_1 \pleq \marq_2$.

\subsection{Karp and Miller procedure}

The Karp and Miller algorithm\cite{Karp69} is a classical algorithm to compute a coverability set of an initialized \ac{PN}.
More precisely, it constructs a coverability tree and uses an acceleration function to systematically detect self-covering sequences.

\begin{defi}[Coverability tree]
  Given a \ac{PN} $\tupleN = \PTm$, a coverability tree $\mathcal{T}$ of \tupleN is a labelled tree $\mathcal{T} = \langle N, B, n_0, \Lambda\rangle$ where:
  \begin{itemize}
    \item $N$ is the set of nodes of the tree.%, is a set of \omark of \tupleN such that $\downc(N) = \Cover(\tupleN)$.
    \item $n_0 \in N$ is the root of the tree, \textit{i.e.} $\nexists n \in N$ such that $(n, n_0) \in B$.
    \item $\Lambda : N \mapsto (\mathbb{N} \cup \{\omega\})^{|P|}$ is a labelling function that associate to each node a \textomega-marking of \tupleN.
    \item $B \subseteq N \times N$, the set of edges, is such that:
      \begin{itemize}
        \item with $(n_1, n_2) \in N^2$, if there exists an edge $(n_1, n_2) \in B$ then there exists a sequence $\sigma$ of transitions of $T$ such that $\Lambda(n_1) \fire{\sigma} \Lambda(n_2)$, and
        \item for all node $n \in N \setminus n_0$, there exists a path from the root to $n$, that is, there exists a sequence of edges of $B$ of the form $((n_0, n_1), (n_1, n_2), \dots, (n_{i}, n))$, $i > 1$, and
        \item there is no cycles, that is, there is no sequences of edges of $B$ of the form $((n_1, n_2), (n_2, n_3), \dots, (n_i, n_1))$.
      \end{itemize}
  \end{itemize}
\end{defi}

\todo{describe K\&M procedure}

\subsection{\ac{EEC} algorithm}
\label{sec:ecc}

\citep{Geeraerts07thesis, Geeraerts06}

\removed{A backward algorithm \citep{Finkel90, Abdulla96}}
%\subsection{A backward algorithm \citep{Finkel90, Abdulla96}}
%
%We will now present an algorithm to solve the coverability problem for a marking \marq of a \ac{PN} $\tupleN = \PTm$.
%
%This algorithm was introduced by Abdulla \textit{et al.} \cite{Abdulla96} for well structured transition systems, a more general class of models which includes \acp{PN}.
%It is close of the one introduced earlier in \cite{Finkel90}.
%
%Recall the definition for a marking of being coverable.
%\coverability*
%
%For convenience, we will use an other equivalent definition.
%
%\begin{defi}[Coverability]
%  Given an initialized \ac{PN} \NPTm and an upward-closed set $U$, $U$ is said coverable if there exists a marking \marqp such that $\marqp \in U$ and $\marqi \fire{*} \marqp$.
%\end{defi}
%
%By choosing $\upc(\marq)$ as $U$, these two definitions set out the same instance of the coverability problem.
%With a set of markings considered in the first definition, $U$ may be the union of their upward-closure in the second.
%
%We say it is a backward algorithm in the sense that it is based on the computation of the set $\Pre^*(\marq)$ and answer by checking whether $\marqi \in \Pre^*(\marq)$; unlike a forward approach which would have calculated the reachability set and conclude by checking whether \marq was in it. In other words, it compute all the configurations that can reach $U$ in any number of steps.
%
%The calculation is a fixed point algorithm that compute the increasing sequence, for the inclusion relation, of sets of markings: $(R_n)_{n \in \mathbb{N}}$, with $R_0 = U$ and $R_{n+1} = \Pre(R_n) \cup R_n$.
%Thus, $R_n$ is the set of markings from which there exists a sequence of at most $n$ transitions which may be fired and that cover $U$.
%Because, with $U$ an upward-closed set of markings, $\Pre(U)$ is upward-closed too%
%\footnote{This is due to the monotonicity of \acp{PN}, \todo{see for example cite\{someone\}}},
%and because the union of two upward-closed sets is an upward-closed set,
%$R_n$ is upward-closed for all $n$.
%
%\todo{summarize correctness and termination from \cite{Abdulla96}}

% vim: set spell spelllang=en :
