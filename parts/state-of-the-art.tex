The use of parameters in formal verification systems is a well developed topic in the literature.

With regard to \ac{PN}, parameters have been introduced with many different roles.
Some works, like \cite{Christensen97} use parameters as places or transitions, for example to make it possible to change a place into a more complex subnet and thus allow different levels of abstractions to be considered.
In \cite{Lindqvist91}, parameters are used on the markings to obtain concise parametrised reachability trees, but not to realize formal verifications on these parametric systems.

\cite{Badouel99} introduce parameters as the weight of arcs to model changes in a system.
The parameters have a finite valuation domain and verifications are performed on these parametrized systems.
Systems with quantitative parameters with infinite valuation domains are analysed in \cite{Abdulla13}.

Our work is in the line with \cite{David17} which use discrete parameters as arc weights and in the markings.
\cite{David17} provide also a proof for the non decidability of \Ucov and \Ecov, and define several subclasses of \ac{PPN} for which these problems are decidable.

Other similar approaches are the \acp{PN} with parametric initial markings, like in [\todo{Modeling with Generalized Stochastic Petri Nets, Marsan 94, On parametric P/T nets and their modelling power, Chiola 91}].

\todo{$\omega$-markings \cite{Geeraerts15}}

% vim: spell spelllang=en :
