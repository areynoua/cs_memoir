By reduction from the halting problem as well as the counter boundedness problem, \cite{David17} has shown that \Ucov and \Ecov are undecidable on \ac{PPN}.
This motivates the introduction of natural subclasses of \acp{PPN} where parametric coverability problems are decidable.

Namely,
PreT-PPNs are \acp{PPN} where parameters are used only as input weight in $\inm$,
PostT-PPNs are \acp{PPN} where parameters are used only as output weight in $\outm$,
and P-PPNs are \ac{PPN} where parameters are used only in the initial marking $\mari$.
These subclasses were extensively studied in \cite{David17} where are given
some links between P-PPN and PostT-PPN,
an adaptation of the Karp and Miller Algorithm to solve the \Ucov problem on PreT-\acp{PPN},
and another to solve the \Ecov problem on PostT-\acp{PPN}%.
%as well as the complexity of these problems.
We will present these results as they contain ideas used in the sequel.

\subsection{Links between P-PPNs and PostT-PPNs}
\label{sec:p-ppn-and-postt-ppn}

Considering the synthesis of valuations for P-PPNs, one idea that comes naturally is to apply $\back$ from a marking to be covered.
We also have the intuition that this same result can be obtained in a similar way for PostT-PPNs. 
This leads us to look for links between these two models.

\paragraph{Simulations on Petri nets}
\label{sec:postt-ppn-to-p-ppn}

We will show how to construct a P-PPN that simulates a given PostT-PPN, that is, one that has ``the same'' behaviour.

\begin{defi}[Simulation on Petri nets]
  Given two \acp{PPN}, $\defPPN[1]$ and $\defPPN[2]$, a relation $\rela \subseteq (\naturals \cup  \parameters_1)^{\card{P_1}} \times (\naturals \cup  \parameters_2)^{\card{P_2}}$ is a \emph{simulation} if $\forall (\mar_1, \mar_2) \in (\naturals \cup  \parameters_1)^{\card{P_1}} \times (\naturals \cup  \parameters_2)^{\card{P_2}} : $
  \begin{gather*}
    (\mar_1, \mar_2) \in \rela \\
    \Updownarrow\\
    \forall (t_1,\mar'_1), t_1 \in \transitions_1, \mar_1 \fire{t_1} \marp_1 \>:\>
    \exists (t_2,\mar'_2), t_2 \in \transitions_2, \mar_2 \fire{t_2} \marp_2, (\mar'_1, \mar'_2) \in \rela.
  \end{gather*}
  If there exists such a relation $\rela$ with $(\mar_{0,1}, \mar_{0,2}) \in \rela$, we say that \namePPN[2] \emph{simulates} \namePPN[1].
\end{defi}

Actually, we will divide the set of transitions of our new P-PPN into two separate subsets: usual or observable transitions, and silent transitions.
The latter form a machinery that is not taken into account for the result of the simulation, then called a ``weak simulation'':
\begin{defi}[Weak simulation on Petri nets]
  Given two \acp{PPN}, $\defPPN[1]$ and $\defPPN[2]$,
  where $T_1 = T_{u1} \cup T_{s1}, T_{u1} \cap T_{s1} = \emptyset$,
  and   $T_2 = T_{u2} \cup T_{s2}, T_{u2} \cap T_{s2} = \emptyset$,
  a relation $\simeq \subseteq (\naturals \cup  \parameters_1)^{\card{P_1}} \times (\naturals \cup  \parameters_2)^{\card{P_2}}$ is a \emph{weak simulation} if $\forall (\mar_1, \mar_2) \in (\naturals \cup  \parameters_1)^{\card{P_1}} \times (\naturals \cup  \parameters_2)^{\card{P_2}} : $
  \begin{gather*}
    (\mar_1, \mar_2) \in \simeq \\
    \Updownarrow\\
    \forall (\seqt_1,\mar'_1), \seqt_1 \in \transitions_1^*, \mar_1 \fire{\seqt_1} \mar'_1 \>:\>
    \exists (\seqt_2,\mar'_2), \seqt_2 \in \transitions_2^*, \mar_2 \fire{\seqt_2} \mar'_2, (\mar'_1, \mar'_2) \in \simeq.
  \end{gather*}
  where $\seqt_i = \transitions_{si}^* + (t_{ui}) + \transitions_{si}^*,\enspace t_{ui} \in \transitions_{ui}$.% $\seqt_i$ is a sequence of zero or more transition from $T_{si}$ followed by a transition of $T_{ui}$ followed by zero or more transition of $T_{si}$.

  If there exists such a relation $\simeq$ with $(\mar_{0,1}, \mar_{0,2}) \in \simeq$, noted $\mar_{0,1} \simeq \mar_{0,2}$ we say that \namePPN[2] \emph{weakly simulates} \namePPN[1].
\end{defi}

\begin{defi}[Weak co-simulation]
  Given two \acp{PPN}, \namePPN[1] and \namePPN[2],
  they are said weakly co-similar if \namePPN[1] weakly simulates \namePPN[2] and \namePPN[2] weakly simulates \namePPN[1].
\end{defi}

\begin{defi}[Weak bisimulation]
  Given two \acp{PPN}, $\defPPN[1]$ and $\defPPN[2]$,
  they are said weakly bisimilar if there exists a weak simulation relation $\sim$ with $\mar_{0,1} \sim \mar_{0,2}$ and $\mar_{0,2} \sim \mar_{0,1}$.
\end{defi}

\todo{We will see later that to establish that there is a simulation between two networks makes it possible to deduce many results.}

\paragraph{From PostT-PPNs to P-PPNs}
\label{sec:from-postt-ppn-to-p-ppn}

\begin{figure}[htbp]
  \centering
  \begin{tikzpicture}[auto,x=0.12\linewidth,y=0.11\linewidth,
  silentp/.style={draw=blue,fill=blue!20},
  silentt/.style={draw=blue,fill=blue},
  silente/.style={draw=blue},
  parame/.style={},
  otherp/.style={draw=black!50,fill=black!10},
  othert/.style={draw=black!50,fill=black!50},
  othere/.style={draw=black!50},
  addede/.style={draw=red}
  ]
  \node [place,fill=blue!50!red!33,draw=blue!50!red] (l) [label=195:\texttt{lock},tokens=1]   at (1,3) {};
  \node [place,silentp] (f) [label=0:\texttt{firing}$_{t_1}$] at (3,3) {};
  \node [place,silentp] (t2) [label=270:\texttt{tank2}$_{t_1,p_1}$]  at (2,0.5) {};
  \node [place,silentp] (t1) [label=270:\texttt{tank1}$_{t_1,p_1}$]  at (2,1.5) {$a$};
  \node [place] (p1) [label=315:$p_1$]      at (4,1.5) {};
  \node [place,otherp] (p1p) [label=45:$p'_1$] at (3.5,4) {};
  \node [place,otherp] (p2)  [label=135:$p_2$] at (-1,4) {};

  \node [transition] (T1) [label=${t_1}$] at (2,4) {}
    edge [pre,  bend right, silente] (l)
    edge [post, bend left,  silente] (f)
    edge [post, othere]              (p1p);
  \node [transition,silentt] (D) [label=\texttt{done}$_{t_1}$] at (2,3) {}
    edge [pre,  silente]             (f)
    edge [post, silente]             (l);
  \node [transition,silentt] (R) [label=225:\texttt{reload}$_{t_1,p_1}$] at (1,1.5) {}
    edge [pre,  bend right, silente] (t2)
    edge [pre,  bend left,  silente] (l)
    edge [post, silente]             (t1)
    edge [post, bend right, silente] (l);
  \node [transition,silentt] (F) [label=315:\texttt{fire}$_{t_1,p_1}$] at (3,1.5) {}
    edge [pre,  bend left,  silente] (f)
    edge [pre,  silente]             (t1)
    edge [post, silente]             (p1)
    edge [post, bend right, silente] (f)
    edge [post, bend left,  silente] (t2);
  \node [transition,othert] (T2) [label=${t_2}$] at (0,4) {}
    edge [pre,  bend left,  addede] (l)
    edge [post, bend right, addede] (l)
    edge [post, othere] (p2);

  \begin{scope}[shift={(0,0.25)}]
    \node [place, label=45:$p'_1$, otherp] (pp1p) at (-3,1.9) {};
    \node [place, label=315:$p_1$]         (pp1)  at (-3,1.1) {};
    \node [place, label=45:$p_2$,  otherp] (pp2)  at (-3,3) {};

    \node [transition, label=$t_1$] at (-3.8,1.5) {}
      edge [post, othere] (pp1p)
      edge [post, parame] node[below] {$a$} (pp1);

    \node [transition, label=$t_2$, othert] at (-3.8,3) {}
      edge [post, othere] (pp2);
    
    \draw [decorate,decoration={brace,mirror}] (-2.4,0.4) -- (-2.4,3.7);
    \draw [->,decorate,decoration={coil,aspect=0}] (-2.3,2) -- (-1.7,2);
  \end{scope}
  \draw [decorate,decoration={brace}] (-1.6,-0.2) -- (-1.6,4.7);

\end{tikzpicture}

  \par
  \caption{From PostT-PPN to P-PPN}
  \label{fig:posttppn-to-pppn}
\end{figure}

We can now construct for any PostT-PPN a P-PPN that weakly simulates it.

Given a PostT-PPN $\namePPN[1] = \tuple{\places_1, \transitions_1, \parameters, \mar_{0,1}}$, create the new P-PPN $\namePPN[2] = \tuple{\places_2, \transitions_2, \parameters, \mar_{0,2}}$ like this.
Starting with an identical \ac{PPN}, first add a new place labelled \texttt{lock}, and, to all the transitions that does not have a parametrised output arc, add an input arc from, and an output one to \texttt{lock}.
Then, for all parametrised transition $t$, replace the parametrised output arcs by the silent subnet depicted in blue on figure~\ref{fig:posttppn-to-pppn}:
create a place \texttt{firing}$_t$ that will contains the ``lock token'' while simulating the firing of the parametrised transition, and a silent transition \texttt{done}$_t$ to release the lock.
then for all parametrised output arc to a place $p$, replace it by the places \texttt{tank1}$_{t,p}$, \texttt{tank2}$_{t,p}$ and the silent transitions \texttt{reload}$_{t,p}$, \texttt{fire}$_{t,p}$ as depicted.

Formally,
\begin{align*}
  \places_2 &= \bigcup \begin{cases}
      \places_1 \\
      \{ \mathtt{lock} \} \\
      \setComp{\mathtt{firing}_t}{\exists p \in \places_1 \text{ s.t. } \outm[\namePPN[1]][p,t] \in \parameters} \\
      \setComp{\mathtt{tank1}_{t,p}, \mathtt{tank2}_{t,p}}{\outm[\namePPN[1]][p,t] \in \parameters}
    \end{cases}\\
  \transitions_2 &= \bigcup \begin{cases}
      \transitions_1 \\
      \setComp{\mathtt{done}_t}{\exists p \in \places_1 \text{ s.t. } \outm[\namePPN[1]][p,t] \in \parameters} \\
      \setComp{\mathtt{fire}_{t,p}, \mathtt{reload}_{t,p}}{\outm[\namePPN[1]][p,t] \in \parameters}
    \end{cases}\\
  \inm[\namePPN[2]][p,t] &= \begin{cases}
      \inm[\namePPN[1]][p,t] &\text{ whenever } p \in \places_1 \wedge t \in \transitions_1 \\
      1 &\text{ whenever } p = \mathtt{lock}
          \wedge t \in \transitions_1 \cup \setComp{ \mathtt{reload}_{t',p'}}{t' \in \transitions_1, p' \in \places_1}\\
        &\quad\text{ or } p = \mathtt{firing}_{t'}
          \wedge t \in \{\mathtt{done}_{t'}\} \cup \setComp{\mathtt{fire}_{t',p'}}{p' \in \places_1} \\
        &\quad\text{ or } p = \mathtt{tank1}_{t',p'} \wedge t = \mathtt{fire}_{t',p'} \\
        &\quad\text{ or } p = \mathtt{tank2}_{t',p'} \wedge t = \mathtt{reload}_{t',p'} \\
      0 &\text{ otherwise }
    \end{cases} \\
  \outm[\namePPN[2]][p,t] &= \begin{cases}
      \outm[\namePPN[1]][p,t] &\text{ whenever } p \in \places_1 \wedge t \in \transitions_1 \wedge \outm[\namePPN[1]][p,t] \notin \parameters \\
      1 &\text{ whenever } p = \mathtt{lock}
          \wedge t \in \transitions_1 \wedge \nexists p' \text{ s.t. } \outm[\namePPN[1]][p',t] \in \parameters \\
        &\quad\text{ or } p = \mathtt{lock}
          \wedge t \in \setComp{\mathtt{done}_{t'}}{t' \in \transitions_1} \cup \setComp{\mathtt{reload}_{t',p'}}{t' \in \transitions_1, p' \in \places_1} \\
        &\quad\text{ or } p = \mathtt{firing}_{t'}
          \wedge t \in \{t'\} \cup \setComp{\mathtt{fire}_{t',p'}}{p' \in \places_1} \\
        &\quad\text{ or } p = \mathtt{tank1}_{t',p'} \wedge t = \mathtt{reload}_{t',p'} \\
        &\quad\text{ or } p = \mathtt{tank2}_{t',p'} \wedge t = \mathtt{fire}_{t',p'} \\
      0 &\text{ otherwise }
    \end{cases}\\
  \mar_{0,2}(p) &= \begin{cases}
      \mar_{0,1}(p)             &\text{ if } p \in \places_1 \\
      1                         &\text{ if } p = \mathtt{lock} \\
      \outm[\namePPN[1]][p',t'] &\text{ if } p = \mathtt{tank1}_{t',p'} \\
      0                         &\text{ otherwise }
    \end{cases}
\end{align*}

This construction comes from \cite{David17}, to which we have added the uniqueness of the \texttt{lock} place for the whole Petri net and its connection with all the transitions, even the non-parametrised ones.
This does not fundamentally change the properties of the newly built network, in that it also provides a simulation, but it reduces the size of its state tree.

\begin{lemm}
  \label{theo:simulation-p-ppn-postt-ppn}
  For all valuation $v$ on $\parameters$, $v(\namePPN[2])$ weakly simulates $v(\namePPN[1])$.
\end{lemm}

\begin{proof}
  Let $\simeq$ be the weak simulation relation that witnesses it, \ie $\mar_{0,1} \simeq \mar_{0,2}$.
  This relation surely exists: we recursively show that it contains all the pairs $(\mar_1, \mar_2)$ such that
  \begin{align*}
    \mar_1 &\in \posts[\val]{\mar_{0,1}}\\
    \mar_2(p) &= \begin{cases}
        \mar_1(p) &\text{ if } p \in \places_1\\
        1         &\text{ if } p \text{ is } \mathtt{lock} \\
        \val[a]\text{, with $a$ the parameter from $t$ to $p'$ in $\namePPN[1]$} &\text{ if $p$ is }\mathtt{tank1}_{t,p'} \\
        0 &\text{ otherwise.}
      \end{cases}
  \end{align*}

  This holds for $(\mar_{0,1}, \mar_{0,2})$.

  Now, consider $\mar_1 \simeq \mar_2$.
  Thus, the induction hypothesis ensures that for all $\mar_1 \fire[{\val[{\namePPN[1]}]}]{t} \mar'_1$ the following sequence of transitions $\seqt$ is enabled in $\mar_2$:
  \begin{align*}
    \seqt = (t)
      &+ \outm[\namePPN[1]][p_{i_1},t] \cdot (\mathtt{fire}_{t,p_{i_1}})
       + \outm[\namePPN[1]][p_{i_2},t] \cdot (\mathtt{fire}_{t,p_{i_2}})
       + \dots \\
      &+ (\mathtt{done}_{t}) \\
      &+ \outm[\namePPN[1]][p_{i_1},t] \cdot (\mathtt{reload}_{t,p_{i_1}})
       + \outm[\namePPN[1]][p_{i_2},t] \cdot (\mathtt{reload}_{t,p_{i_2}})
       + \dots
  \end{align*}
  with $p_{i_1}, p_{i_2}, \dots$ the places $p_{i_j}$ of $\places_1$ such that $\outm[\namePPN[1]][p_{i_j}] \in \parameters$.
  Moreover, this sequence leads to a marking $\mar'_2$, $\mar_2 \fire[{\val[{\namePPN[2]}]}]{\seqt} \mar'_2$, for which the induction hypothesis holds.
  Thus $\mar'_1 \simeq \mar'_2$ and we can construct inductively the weak simulation relation.
  %First consider $v(\namePPN[1])$ and $v(\namePPN[1]')$ where $\namePPN[1]'$ is the network augmented by the \texttt{lock} place and its arcs from and to the non-parametrised transitions.
  %The simulation is obvious here: it contains all the pair of markings $(\mar_1, \mar_2)$ such that $\mar_1 \in \Post^*_v(\mar_{0,1})$ and
  %\[
  %  \mar_2(p) =
  %    \begin{cases}
  %      1 &\text{ if $p$ is \texttt{lock},} \\
  %      \mar_1(p) &\text{ otherwise.}
  %    \end{cases}
  %\]
  %
  %Now consider $v(\namePPN[2])$ and focus on the parametrised transitions.
  %Each time an arc from $t$ to $p$ parametrised by $a$ ($\matO_{\namePPN[1]}(p)(t) = a$) is involved in a firing in $v(\namePPN[1])$, $v(a)$ tokens are created in $p$.
  %Regarding $v(\namePPN[2])$, the sequence $(t, \mathtt{fire}_{t,p}, \mathtt{done}_t, \mathtt{reload}_{t,p})$, where $\mathtt{fire}_{t,p}$ and $\mathtt{reload}_{t,p}$ are repeated $a$ times, creates $v(a)$ tokens in $p$ and reset the sub-net constructed for the simulation\todo{rewrite with regular notation}. Note that the rest of the network is not affected. Thus, monotonicity ensures that we have a weak simulation.
\end{proof}

Looking at the subnet in Figure~\ref{fig:posttppn-to-pppn},
one guesses that $\posts[\val({\namePPN[2]})]{\mar_{0,2}}$ does not contain “many” markings that are not involved in the weak simulation.
Indeed, by unfolding the transition relation on the created subnet for $\outm[\namePPN[1]][p,t] = a$, one can observe that the reachable markings when $t$ is fired only once are of the form:
\begin{align}
  \label{eqn:widget-reachable-markings}
  \mar_2(p') =
  \begin{cases}
    l   &\text{ if } p' \text{ is } \mathtt{lock}, \\
    f   &\text{ if } p' \text{ is } \mathtt{firing}_t, \\
    a_1 &\text{ if } p' \text{ is } \mathtt{tank1}_{t,p}, \\
    a_2 &\text{ if } p' \text{ is } \mathtt{tank2}_{t,p}, \\
    a_3 &\text{ if } p' \text{ is } p
  \end{cases}
  \text{\qquad with }
  \left\{
    \begin{aligned}
      l + f &= 1 \\
      a_1 + a_2 &= \val[a] \\
      a_3 \leq a_1 &\leq \val[a]
    \end{aligned}
  \right.
\end{align}

Thus, for all reachable marking $\mar_2 \in \Post^*_{v(\namePPN[2])}(\mar_{0,2})$ of \namePPN[2] there exists a reachable marking $\mar_1 \in \Post^*_{v(\namePPN[1])}(\mar_{0,1})$ of \namePPN[1] that ``covers'' it for all the common places: $\forall p \in \places_1, \mar_1(p) \geq \mar_2(p)$.
Actually, \namePPN[1] also simulates \namePPN[2].
\namePPN[1] and \namePPN[2] are \emph{weakly co-similar}.

\begin{lemm}
  \label{theo:simulation-postt-ppn-p-ppn}
  For all valuation $v$ on $\parameters$, $v(\namePPN[1])$ weakly simulates $v(\namePPN[2])$.
\end{lemm}

\begin{proof}
  We inductively create the weak simulation by keeping the following invariant:
  \( \mar_2 \simeq \mar_1 \Rightarrow \forall p \in \places_1, \mar_2(p) \geq \mar_1(p) \).
  Note that this property ensures that any visible transition enabled in $\mar_2$ is enabled in $\mar_1$.
  
  This holds for $(\mar_{0,2}, \mar_{0,1})$.

  Given a pair $\mar_2 \simeq \mar_1$ of the weak simulation, for all $\mar'_2$ given by
  $\mar_2 \fire[{\val(\namePPN[2])}]{t} \mar'_2$
  we add into the relation the pair $(\mar'_2, \mar'_1)$ where
  \begin{itemize}
    \item $\mar'_1$ is given by $\mar_1 \fire[{\val(\namePPN[1])}]{t} \mar'_1$ if $t \in \transitions_1$,
    \item $\mar'_1$ is $\mar_1$ otherwise, \ie if the $t$ is silent.
  \end{itemize}

  Indeed, if $t \in \transitions_1$, strong monotonicity of \ac{PN} ensures that the induction hypothesis holds for $(\mar'_2, \mar'_1)$.
  On the other hand, if $t$ is silent, either it is a transition $\mathtt{done}_{t'}$ or $\mathtt{reload}_{t',p'}$ and they do not change any place $p \in \places_1$; either it is a $\mathtt{fire}_{t',p'}$ transition and it adds a token in $p' \in \places_1$.
  However, $\mathtt{fire}_{t',p'}$ may be firered at most $\val[{\outm[{\namePPN[1]}]}]$ times each time $t'$ is fired once.
  (You can refer to~\cref{eqn:widget-reachable-markings} for more details.)
  This ensures that the induction hypothesis holds.
  %For all $\mar_{0,2} \fire[{\val(\namePPN[2])}]{t} \mar_{1,2}$, there exists $\mar_{1,1}$ such that $\mar_{0,1} \fire[{v(\namePPN[1])}]{t} \mar_{1,1}$.
  %After firing this first transition, either the transition was a non-parametrized one and we can repeat the operation, or it was parametrized and the lock disables the observable transitions until $\mathtt{done}_t$ has been fired.
  %In the latter case, any firable sequence ending with $\mathtt{done}_t$ is silent and produces a marking $\mar'_{1,2}$ covered by $\mar_{1,1}$.
  %Thus, by monotonicity and thanks to the lock, we know that once $\mathtt{done}_t$ has been fired, all the transitions enabled in $\mar'_{1,2}$ in \namePPN[2] are enabled in $\mar_{1,1}$ in \namePPN[1], except for some \texttt{reload} ones whose the effect is restricted to the \texttt{tank} places.
  %This can therefore be repeated and provides us with the desired weak-simulation.
\end{proof}

\paragraph{From P-PPN to PostT-PPN}
\label{sec:p-ppn-to-postt-ppn}

Symmetrically, from any P-PPN $\namePPN[1]$, one can construct a PostT-PPN $\namePPN[2]$ that weakly simulates it~\citep{David17}.
To do so, we create a silent parametric transition that produce the initial marking.
To ensure that it is fired first, we add a place from which an input arc goes into the newly created transition, and we put in it the only token of the initial marking of the new $\namePPN[2]$.

\begin{figure}[htbp]
  \centering
  \begin{tikzpicture}[auto,x=0.12\linewidth,y=0.11\linewidth,
  silentp/.style={draw=blue,fill=blue!20},
  silentt/.style={draw=blue,fill=blue},
  silente/.style={draw=blue}]

  \begin{scope}[shift={(-4,0)}]
    \node [place]          (p1) at (0,4) {$\param[1]$};
    \node [place,tokens=2] (p2) at (0,3) {};
    \node [place,tokens=1] (p3) at (0,2) {};
    \node [place]          (p4) at (0,1) {};
    \draw [decorate,decoration={brace,mirror}] (0.6,0.6) -- (0.6,4.4);
    \draw [->,decorate,decoration={coil,aspect=0}] (0.7,2.5) -- (1.3,2.5);
  \end{scope}

  \node [place,tokens=1,silentp]      (PI) at (-2,2.5) {};
  \node [place]                       (P1) at (0,4) {};
  \node [place]                       (P2) at (0,3) {};
  \node [place]                       (P3) at (0,2) {};
  \node [place]                       (P4) at (0,1) {};
  \draw [decorate,decoration={brace}] (-2.6,0.6) -- (-2.6,4.4);

  \node [transition, silentt] (TI) at (-1.1,2.5) {}
    edge [pre, silente]  (PI)
    edge [post, silente, bend left]  node[auto] {$\param[1]$} (P1)
    edge [post, silente, bend left]  node[auto] {$2$}         (P2)
    edge [post, silente, bend right]                          (P3);
\end{tikzpicture}

  \par
  \caption{From P-PPN to PostT-PPN}
  \label{fig:pppn-to-posttppn}
\end{figure}

Formally, given a P-PPN $\defPPN[1]$, we create a PostP-PPN $\defPPN[2]$ that provides a weak bisimulation, as follows:
\begin{align*}
  &\places_2 = \places_1 \cup           \{ p_0 \}
    \text{ with } p_0 \notin \places_1 \\
  &\transitions_2 = \transitions_1 \cup \{ t_0 \}
    \text{ with } t_0 \notin \transitions_1 \text{($t_0$ is  the only silent transition of both the nets)} \\
  &\inm[\namePPN[2]][p,t] = \begin{cases}
      \inm[\namePPN[1]][p,t] &\text{ whenever } p \in \places_1 \wedge t \in \transitions_1 \\
      1 &\text{ whenever } p = p_0 \wedge t = t_0 \\
      0 &\tothrw
    \end{cases} \\
  &\outm[\namePPN[2]][p,t] = \begin{cases}
      \outm[\namePPN[1]][p,t] &\text{ whenever } p \in \places_1 \wedge t \in \transitions_1 \\
      \mar_{0,1}(p) &\text{ whenever } p \in \places_1 \wedge t = t_0 \\
      0 &\tothrw
    \end{cases} \\
  &\mar_{0,2}(p) = \begin{cases}
      1 & \tif p = p_0 \\
      0 & \tothrw
    \end{cases}
\end{align*}

\begin{lemm}
  For any valuation $\val$, $\val[{\namePPN[1]}]$ and $\val[{\namePPN[2]}]$ are weakly bisimilar.
 \end{lemm}

\begin{proof}
  There surely exists a weak bisimulation relation that contains all the pairs:
  \begin{align*}
    \{ ( \mar_{0,1}, \mar_{0,2} ) \}
    \cup
    \{ (\mar, \mar') \}
  \end{align*}
  where $\mar'(p) = \begin{cases}\mar(p) &\tif p \in \places_1\\0&\tothrw\text{ (if $p = p_i$)}\end{cases}$
\end{proof}

\subsection{Karp and Miller algorithm for \Ecov on PostT-\ac{PPN}}
\label{sec:km-ecov-postt-ppn}

The adaptation of the Karp and Miller algorithm to solve the \Ecov problem on PostT-\acp{PPN} is the result of one key observation.
In a PostT-\ac{PPN}, a place may contains an arbitrary large number of tokens either because of the presence of a self-covering increasing sequence, as in a plain \ac{PN}, or because of an arbitrary large valuation.
In the latter case, the place is not necessary \emph{unbounded}, that is to say, once a valuation is given, the number of tokens the place may contains may be bounded, even if the bound is arbitrary large due to the arbitrary large values of the valuation.
This is the case for place $p_1$ in the net shown in \cref{fig:postt-ppn-bound}: obviously it can contain $\val(\param)$ tokens, whose value may be arbitrarily high; but the place is bounded to this value.
The number of tokens in place $p_2$, on the contrary, is not bounded.

\begin{figure}[htbp]
  \centering
  \begin{tikzpicture}[auto,x=0.12\linewidth,y=0.11\linewidth]
  \node [place,tokens=1] (p0) [label=$p_0$] at (-1,1) {};
  \node [place] (p1) [label=$p_1$] at (1,1) {};
  \node [place] (p2) [label=$p_2$] at (1,0) {};

  \node [transition] (t1) [label=$t_1$] at (0,1) {}
    edge [post] node [auto] {$\param[1]$} (p1)
    edge [pre] (p0);
  \node [transition] (t2) [label=$t_2$] at (0,0) {}
    edge [post] (p2);
\end{tikzpicture}

  \par
  \caption{An initialized PostT-\ac{PPN}}
  \label{fig:postt-ppn-bound}
\end{figure}

The adaptation of the Karp and Miller algorithm is therefore mainly the addition of a way to distinguish between these two cases.
It is done by introducing a new value allowed in the markings, noted $*$ and such that $* \notin \naturals$, $* \neq \omega$, and $\forall c \in \naturals$, we have:
\begin{itemize}
  \item $c < * < \omega$,
  \item $* - c = *$,
  \item $* + c = *$,
  \item $\omega - * = \omega$, and
  \item $\omega + * = \omega$.
\end{itemize}

For our purpose, we add: \todo{or not?}
\begin{itemize}
  \item $* + * = *$, and
  \item $\omega + \omega = \omega$.
\end{itemize}

$*$ being defined, the propagation of the $*$ still have to be ensured in the cases where all the input places of a transition are marked by a $*$, even if the output of the transition is not a parameter.
This is done by adapting the acceleration function.
We give it here before to explain the different cases.

Given a marking to accelerate $\mar$ with the set $\setm$ of markings as the base of the acceleration,
the values of the different places of the accelerated marking are determined as follows:
%\[
%  \Acc(\mar, \setm)(p) =
%  \begin{cases}
%    \omega  & \tif \exists \marp \in \setm : \marp \prec \mar \tand \marp(p) < \mar(p) \tand\\
%            & \phantom{\tif \exists \marp \in \setm : }
%              \forall p' \in \places, \mar'(p') \neq \omega: \effect{\mpath{\mar'}{\mar}}(p') \geq 0\\
%        % Cette dernière condition dit qu'un effet négatif n'est pas autorisé dans
%        % une place marquée de *. En effet dans ce cas il ne suffirait pas de répéter
%        % un morceau de séquence; mais il faudrait choisir une valuation plus grande.
%        % Cependant, je pense que pour répondre au E-cov problem on pourrait justement
%        % autoriser un effet négatif ici. À creuser.
%    *       & \tothrwand\\
%            & \text{ }\exists \mar' \in \setm : \mar' \prec \mar \tand \mar'(p) < \mar(p) \tand\\
%            & \phantom{\text{ }\exists \mar' \in \setm :}
%              \forall p' \in \places, \mar'(p') \in \naturals: \effect{\mpath{\mar'}{\mar}}(p') \geq 0 \\
%        % En fait la dernière condition ajoute aussi que si une place est accélérée
%        % entre le marquage choisi pour cette accélération et le marquage qu'on est en train
%        % d'accélérer, alors il faut vérifier que l'effet soit bien positif.
%        % Par exemple: <0,1> +a,+0 <*,1> -3,+1 <*,2>.
%        % Peut on accélérer le 2 si l'effet sur le * est -3?
%    \mar(p) & \tothrw
%  \end{cases}
%\]
% Accélération modifiée
\[
  \Acc(\mar, \setm)(p) =
  \begin{cases}
    \omega  & \tif \exists \marp \in \setm : \marp \prec \mar \tand \marp(p) < \mar(p) \tand\\
            & \phantom{\tif \exists \marp \in \setm : }
              \forall p' \in \places, \mar(p') \neq \omega: \effect{\mpath{\mar'}{\mar}}(p') \geq 0\\
        % Cette dernière condition dit qu'un effet négatif n'est pas autorisé dans
        % une place marquée de *. En effet dans ce cas il ne suffirait pas de répéter
        % un morceau de séquence; mais il faudrait choisir une valuation plus grande.
        % Cependant, je pense que pour répondre au E-cov problem on pourrait justement
        % autoriser un effet négatif ici. À creuser.
    *       & \tothrwand\\
            & \text{ }\exists \mar' \in \setm : \mar' \prec \mar \tand \mar'(p) < \mar(p) \tand\\
            & \phantom{\text{ }\exists \mar' \in \setm :}
              \forall p' \in \places, \mar(p') \in \naturals: \effect{\mpath{\mar'}{\mar}}(p') \geq 0 \\
        % En fait la dernière condition ajoute aussi que si une place est accélérée
        % entre le marquage choisi pour cette accélération et le marquage qu'on est en train
        % d'accélérer, alors il faut vérifier que l'effet soit bien positif.
        % Par exemple: <0,1> +a,+0 <*,1> -3,+1 <*,2>.
        % Peut on accélérer le 2 si l'effet sur le * est -3?
    \mar(p) & \tothrw
  \end{cases}
\]
\todo{introduce \textbackslash mpath}

This new acceleration function, $\Acc$, distinguishes three cases:
\begin{itemize}
  \item Either there exists $\marp \in \setm$ such that $\marp \fire{\sigma} \mar$, $\marp \prec \mar$, and for all place $p'$ such that $\marp(p') \neq \omega$ we have $\effect{\sigma}(p') \geq 0$.\\
    This case corresponds to the classic acceleration and
    for all $p$ such that $\marp(p) < \mar(p)$
    we have:
    \[
      \Acc(\mar, \setm)(p) = \omega
    \]
\item Or these first conditions does not hold but there is a marking
  that meets these conditions, except that the effect of the sequence of transitions is negative in some places $p'$ with $\mar(p') = *$,
  \ie,
  there exists $\marp \in \setm$ such that $\marp \fire{\sigma} \mar$, $\marp \prec \mar$, and for all place $p$ such that $\marp(p) \notin \{\omega, *\}$ we have $\effect{\sigma}(p) \geq 0$.\\
    This case handle the propagation of the $*$ mentioned above.
    Indeed, we can choose an arbitrary large valuation to be able to fire this sequence an arbitrary large number of times; however, for a given valuation, this number of repetition is bounded.
    So, if $\marp(p) < \mar(p)$
    we have:
    \[
      \Acc(\mar, \setm)(p) = *
    \]
  For example, given the marked PPN depicted on \cref{fig:postt-ppn-example}, consider the sequence $\mari = (1, 1, 1, 1) \fire{t_1} \mar' = (0, *, 2, 1) \fire{t_2} \mar = (0, *, 3, 1)$.
    We do have $\mar' \prec \mar$ and $\mar'(p_3) < \mar(p_3)$, but $\effect{\mpath{\mar'}{\mar}}(p_2) < 0$.
    This shows that this sequence does not witness that $p_3$ is unbounded,
    however its bound may be made arbitrary large,
    thus $\Acc(\mar, \{\mari, \mar'\}) = (0, *, *, 1)$.
  \item Or no one of the previous cases holds.
    In this case $\Acc(\mar, \setm)(p) = \mar(p)$.
\end{itemize}

\begin{figure}[htbp]
  \centering
  \begin{tikzpicture}[auto,x=0.12\linewidth,y=0.11\linewidth]
  \node [place,tokens=1] (p1) [label=$p_1$] at (0,2) {};
  \node [place,tokens=1] (p2) [label=$p_2$] at (2,3) {};
  \node [place,tokens=1] (p3) [label=-45:$p_3$] at (2,1) {};
  \node [place,tokens=1] (p4) [label=225:$p_4$] at (-0.5,0) {};

  \node [transition] (t1) [label=112:$t_1$] at (1,2) {}
    edge [pre]                            (p1)
    edge [post, bend left]  node [above] {$\param$} (p2)
    edge [post, bend right]                        (p3);
  \node [transition,horizontal] (t2) [label=15:$t_2$] at (3,2) {}
    edge [pre,  bend right]  node [above] {$2$}     (p2)
    edge [post, bend left ]                        (p3);
  \node [transition] (t3) [label=-22:$t_3$] at (1,0) {}
    edge [pre, bend right]  node [auto] {$3$}      (p3)
    edge [post]                                    (p4)
    edge [post, bend left]                         (p1);
\end{tikzpicture}

  \par
  \caption{An initialized PostT-\ac{PPN}}
  \label{fig:postt-ppn-example}
\end{figure}

Intuitively, the second case formalise the idea that one can remove some tokens from a place with $*$ tokens without being an issue for the existence of a valuation that allows to cover a given marking.

Keeping in mind that we are looking for the existence of a valuation, as large as it may be, that allows to cover a (set of) marking(s) of the \ac{PPN} \namePPN, one can now perform the Karp and Miller algorithm, with the adapted acceleration function, on the plain \ac{PN} $v(\namePPN)$ where $v$ is the $*$-valuation that maps every parameter to $*$.
We refer to this adatption of the Karp and Miller algorithm as the \emph{Karp and Miller algorithm for PostT-PPNs}.

The above considerations along with the analyse of the Karp and Miller algorithm provided in \cref{sec:km} should allow the reader to grasp how this procedure works.
Thus we give here a brief proof of the termination and correctness of the modified procedure.

\begin{theo}
  \label{theo:km-ecov-postt-ppn-termination}
  The Karp and Miller algorithm for PostT-PPNs
  %(the Karp and Miller algorithm used on the $*$-valuation of a PostT-PPN, with $\Acc$ as acceleration function)
  terminates and produces a finite tree.
\end{theo}

\begin{proof}
  %We prove that the frontier can not grow indefinitely.
  Since the number of transition is finite, the algorithm build a tree whose each node has a finite number of children.
  Thus, from the \todo{Köning Lemma}, the tree is finite iff it does not contains an infinite branch $\branch$. 
  Suppose that this branch exists.
  From \todo{lemma ..}, we can extract an infinite subsequence, respecting the order of the infinite branch, such that the labels from this branch $\defInfSeq[\seq][\mar][0][1]$ are sorted according to $\preceq$.
  Since the algorithm ensures that they are distincts, we have:
  \( \elemsInfSeq[\mar][0][1][\prec \mar_i \prec \dotsb \prec \mar_j \prec \dotsb][\prec][\dotsb] \)
  
  Since $\mar_i$ and $\mar_j$ are from the same branch and $\mar_i \prec \mar_j$, $\mar_j$ has at least one place marked by $\omega$ or $*$ more than $\mar_i$ does.
  This reasoning can be repeated on the sequence starting in $\mar_j$.
  However, as the number of places is finite, we will eventually reach a marking where all places will be marked with an $\omega$, \ie, which can not have a successor in the sequence.
  This contradicts the hypothesis that an infinite branch exists.
\end{proof}

\begin{theo}
  \label{theo:km-ecov-postt-ppn-correctness}
  Given a marked PostT-PPN \namePPN and a marking $\mar$ of \namePPN,
    then \qquad (1)
    there exists a valuation $v$ such that $\mar$ is coverable in \namePPN
    \qquad iff \qquad (2)
    there is a node $n$ of the Karp and Miller tree produced by the algorithm for PostT-PPNs
      such that $\Lab(n)$ cover $\mar$.
\end{theo}

\begin{proof}
  We start by proving that (1) $\Rightarrow$ (2).
  Let $\mar$ be a marking of $\cover{\val[\namePPN]}$ and $\defSeqt{0}{n-1}$ a sequence of transitions that witnesses it:
  \(\mari \fire{\seqt} \mar', \mar \preceq \mar'\).
  Let $\bodySeqm{1}{n}$ be the markings : \(\mari \fire{t_0} \mar_1 \fire{t_1} \dots \fire{t_{n-1}} \mar_n = \mar'\).
  First, for each parametric transition in $\seqt$, replace their parametric effect(s) by $*$ in the sequence of markings.

  Then, apply the following operations:
  while there exists a pair of markings \((\mar_i, \mar_j), i < j, \mar_i \preceq \mar_j\) in the sequence:
  \begin{itemize}
    \item if $\mar_i = \mar_j$, remove all the nodes after $\mar_j$,
    \item otherwise, if for all place $p'$ such that $\mar_{j-1}(p') \neq \omega$, we have $\effect{\mpath{\mar_i}{\mar_j}}(p') \geq 0$, then, for each place $p$ such that $\mar_i(p) < \mar_j(p)$, replace $\mar_k(p)$ by $\omega$ for all the markings $\mar_k, k \geq j$ from $\mar_j$ (included) to the end of the sequence.
      %if the effect is positive on the places 
      %for each place $p$ such that $\mar_i(p) < \mar_j(p)$, replace $\mar_k(p)$ by $\omega$ for all the markings $\mar_k$ from $\mar_j$ (included) to the end of the sequence.
    \item otherwise, for each place $p$ such that $\mar_i(p) < \mar_j(p)$, replace $\mar_k(p)$ by $*$ for all the markings $\mar_k, k \geq j$ from $\mar_j$ (included) to the end of the sequence.
      \todo{triple-check}
  \end{itemize}

  At the end, the sequence obtained is the sequence of labels in some path $\treepath{n}$ from the $n_0$ to $n$ and the final label cover $\mar$.

  To prove the other direction, suppose that there are $\mar$ and $n$ such that $\mar \in \downc{\lab{n}}$.
  Let $\treepath{n} = \bodySeq[n]{0}{m}$ be the nodes of the branch to $n$.
  Assume, without loss of generality, that the first $J$ places, and only them, of $\lab{n}$ are marked with $\omega$.
  Also assume that in $\treepath{n}$, $\omega$'s are introduced in the order $\range{1}{J}$.
  Then, for each $j \in \range{1}{J}$, there exists a subsequence of transitions
  %$\seqt[2]_j = \labt{\treepath[n_j]{n^\omega_j}}$
  $\labt{\treepath[n_j]{n^\omega_j}}$
  whose the effect is positive in $p_i$ and non negative in $p_{j+1}$ through $p_{\card{\places}}$.
  The negative effect of those transitions on the places $p_1$ through $p_J$ may be undone as seen in the proof of \cref{theo:km-correctness} to obtain, for all $j \in \range{1}{J}$, an increasing self covering sequence $\seqt[2]_j$.
  \todo{introduce here the equation from KM?}
  So we can construct a sequence of the form:
  \[ \tts{\mar}{n} = \labt{\treepath{n_1}} + k_1 \cdot \iscs{1} + \dots + k_J \cdot \iscs{J} + \labt{\treepath[n^\omega_J]{n}} \]
  with the \( k_j \geq 0 \),
  and
  such that \( \mari \fire{\tts{\mar}{n}} \mar'\). %, \mar \preceq \mar' \).
  With a given valuation $\val$, we have
  \[
    \begin{cases}
      \mar'(p) \geq \mar(p) &\tif \lab{n}(p) =   \omega \\
      \mar'(p) =    \mar(p) &\tif \lab{n}(p) \in \naturals \\
      \mar'(p) =    c + c_{p, \param_1} \val[\param_1] + c_{p, \param_2} \val[\param_2] + \dots
                            &\tif \lab{n}(p) =   *
    \end{cases}
  \]
  The value of the various $c$ are determined by the sequence of transition.
  It is easy to see that we can choose a valuation to cover any marking $\mar \in \downc{\lab{n}}$.
  In addition, the value of the markings in the places $p$ such that $\lab{n}(p) = *$ is bounded for a fixed marking.
  %Let $\labt{\treepath{n}} = \bodySeqt{1}{n}$
\end{proof}

\todo{This proof is not the one given in David 17}

\subsection{Karp and Miller algorithm for \Ucov on PreT-\ac{PPN}}
\label{sec:km-ucov-pret-ppn}

Here is a symmetrical situation.
The parameters are used to indicate the number of tokens required for transitions to fire, and we want to determine if a (set of) marking(s) is coverable for all the possible valuations of the parameters, as large as they may be.
Therefore, the adaptation of the Karp and Miller algorithm lies on considering that a transition with parametric input arcs is enabled if and only if the corresponding places are marked by $\omega$.
This ensure that the transition is regarded as enabled only if it may actually fire whatever the valuation.
The other direction of the implication holds too, that is to say that, if a transition has a parametric input arc whose the corresponding place is bounded, there exists a valuation that does not enable the transition.
Indeed, recall that all simultaneous unbounded places in a \ac{PN} appear marked by $\omega$ in at least one label of the Karp and Miller tree for this \ac{PN}.

% vim: set spell spelllang=en_gb,fr :
