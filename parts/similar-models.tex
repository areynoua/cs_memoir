Many reactive systems are naturally modelled as infinite-state systems.
Some infinite-states models are known to allow some automatic formal verification from model-checking techniques.
Among them are \acp{PN}, but also Lossy FIFO systems \citep{Abdulla98}, Broadcast protocols \citep{Emerson98}...
All are \acp{WSTS} (but there exists other famous infinite-states systems, like timed-automata \citep{Alur94}).
\todo{check whether timed-automata are \acp{WSTS}.}

\acp{WSTS} are outside of the range of this work, but we will sometimes refer to them to distinguish between techniques specific to the \ac{PN} model, or usable in a wider range of contexts.

For now, let us just say that \acp{PN} are \acp{WSTS} with stronger requirements on the transition relation.
Namely, \acp{WSTS} are transition systems whose the set of states is well-quasi-ordered (\cref{defi:well-quasi-order}) and whose transition relations is monotonic (\cref{defi:monotonicity}) with respect to the well-quasi-ordering.
All this will be defined in \Cref{sec:the-pn-model}.

