\todo{Clearly moving the sections introduced some inconsistencies}

Many reactive systems are naturally modelled as infinite-state systems.
Some infinite-states models are known to allow some automatic formal verification from model-checking techniques.
Among them are \acp{PN}, but also Lossy FIFO systems \citep{Abdulla98}, Broadcast protocols \citep{Emerson98}...
All are \acp{WSTS} (but there exists other famous infinite-states systems, like timed-automata \citep{Alur94}).
\todo{check whether timed-automata are \acp{WSTS}.}

\acp{WSTS} are outside of the range of this work, but we will sometimes refer to them to distinguish between techniques specific to the \ac{PN} model, or usable in a wider range of contexts.

For now, let us just say that \acp{PN} are \acp{WSTS} with stronger requirements on the transition relation.
Namely, \acp{WSTS} are transition systems whose the set of states is well-quasi-ordered and whose transition relations is monotonic with respect to the well-quasi-ordering.
All this will be defined in \Cref{sec:the-pn-model}.

\mov{In a few words, \acp{WSTS} are transition systems whose set of states are well-quasi-ordered and whose transition relations is monotonic with respect to the well-quasi-ordering.

The monotonicity property differs from the strong monotonicity defined above for \acp{PN} by the fact that the second state may be found after many steps.
Formally:
\begin{defi}[Monotonicity]
  \label{defi:monotonicity}
  A transition system is said \emph{monotonic} whenever its transition relation is monotonic.

  A transition relation $\fire{} \subseteq (\set \times \set)$ over a $\leq$-well-quasi-ordered set $\set$ is monotonic if, and only if, for all $\sit_1$, $\sit_2$, and $\sit_3$ from $\set$ such that $\sit_1 \leq \sit_2$ and $\sit_1 \rightarrow \sit_3$ there exists $\sit_4 \in \set$ such that $\sit_2 \fire{*} \sit_4$.
\end{defi}}
