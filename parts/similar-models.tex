Many reactive systems are naturally modeled as infinite-state systems.
Some infinite-states models are known to allow some automatic formal verification from model-checking techniques.
Among them are \acp{PN}, but also Lossy FIFO systems \citep{Abdulla98}, Broadcast protocols \citep{Emerson98}...
All are \acp{WSTS} (but there exists other famous infinite-states systems, like timed-automata \citep{Alur94}).
\todo{check whether timed-automata are \acp{WSTS}.}

\acp{WSTS} are outside of the range of this work, but we will sometimes refer to them to distinguish between techniques specific to the \ac{PN} model, or usable in a wider range of contexts.
\todo{Check english writting.}

In a few words, \acp{WSTS} are transition systems whose set of states are well quasi-ordered and whose transition relations is monotonic with respect to the well quasi-ordering.

The monotonicity property differ from the strong monotonicity defined above for \acp{PN} by the fact that the second state may be found after many steps. Formally:
\begin{defi}[Monotonicity]
  A transition system is said \emph{monotonic} whenever \todo{or `when'?} its transition relation is monotonic.

  A transition relation $\rightarrow \subseteq (\set \times \set)$ over a $\leq$-well quasi-ordered set $S$ is monotonic if, and only if, for all $s_1$, $s_2$, and $s_3$ from $\set$ such that $s_1 \leq s_2$ and $s_1 \rightarrow s_3$ there exists $s_4 \in \set$ such that $s_2 \fire{*} s_4$.
\end{defi}
