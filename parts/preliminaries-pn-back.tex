There exists a simple way to solve the coverability problem for all the examples of \acp{WSTS} above mentioned.
This algorithm was introduced by Abdulla \lang{et al.} \citep{Abdulla96}.
It is close of the one introduced earlier in \cite{Finkel90}.
%Even if we do not see how to use it in the context of \ac{PPN}, we mention it here because it helps to grasp, by comparison with the Karp and Miller algorithm presented in the following section, where does lie the difficulties of the coverability problem.

Relying on the definition~\ref{defi:upclocovprblm} of the coverability problem, given an upward-closed set of markings $\ucs$, the algorithm computes $\pres{\ucs}$ by iterating the $\Pre$ operator.
Thus we compute a sequence $(\setm_i)_{i \geq 0}$ where $\setm_i$ is the set of markings from where $\ucs$ is reachable in $i$ or less steps.


\begin{algorithm}
  \caption{$\back$}
  \label{algo:back}

  \begin{algorithmic}
    \Require $\ucs$ the upward-closed goal set of markings
    \State $\setm \gets \ucs$
    \Repeat
      \State $\setm'\gets \setm$
      \State $\setm \gets \setm \cup \pre{\setm}$
    \Until{$\setm' = \setm$}
    \Ensure $\setm = \pres{\ucs}$
  \end{algorithmic}
\end{algorithm}


Although the sets handled by the algorithm are infinite, it is known that the algorithm is effective.
Indeed, all the sets are upward-closed (\Cref{theo:pre-upc}) and can be manipulated through their minimal elements (see \Cref{theo:upward-closed-set-representation,theo:dickson}, and \cite{Ganty09, valk1985residue}).

The termination and correction of the algorithm were proved in \cite{Abdulla96}.% Here are the main elements of the proofs.


The termination is ensured by the existence of a fixed point in the sequence of upward-closed set of markings $(\setm_i)_{i \geq 0}$:
\begin{gather*}
  \setm_0 = \ucs \\
  \forall i > 0 : \setm_i = \setm_{i-1} \cup \pre{\setm_{i-1}}
\end{gather*}
From this fixpoint $\setm_j$, we have $\setm_k = \setm_j$ for all $k \geq j$.
Thus, $\setm_j = \pres{\ucs}$, \ie when the fixed point is reached, we have $\pres{\ucs}$.
This ensures the correction.

If $\mari \in \pres{\ucs}$, one can conclude positively.
Otherwise, the result is negative.
We call this algorithm $\back$.

This approach is elegant and general, but often inefficient in practice.
It is well-known that a forward exploration of the state of space (\ie, in this context, using $\Post$ instead of $\Pre$) is usually more efficient \citep{Henzinger98}.
We now present a forward algorithm, but whose the application is restricted to \acp{PN}.
