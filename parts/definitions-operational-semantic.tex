Given a \ac{PN} $\defNonMarkedPN$ and a marking \mar on \namePN, a transition $t \in \transitions$ is said \emph{enabled} by \mar if $\forall p \in \places : \mar(p) \geq \inw[t][p]$.
An enabled transition can be \emph{fired} to produce a new marking \marp such that $\forall p \in \places : \marp(p) = \mar(p) - \inw[t][p] + \outw[t][p]$.
This is denoted by $\mar \fire{t} \marp$.

Here are some additional notations:
\begin{itemize}
  \item $\mar \fire{} \marp$ denotes that there exists $t \in \transitions$ such that $\mar \fire{t} \marp$.
  \item $\mar \fire{\seqt} \marp$ where $\seqt$ is a sequence of transitions $\defSeqt{1}{n-1}, t_i \in \transitions, i \in \range{1}{n-1}$ denotes that there exists a sequence of markings $\bodySeqm{1}{n}$ such that : $\mar = \mar_1 \fire{t_1} \cdots \fire{t_{n-1}} \mar_n = \marp$.
  \item $\mar \fire{*} \marp$ denotes that there exists a sequence of transition $\seqt$ such that $\mar \fire{\seqt} \marp$.
    Note that the $\fire{*}$ relation is the reflexive and transitive closure of the relation $\fire{}$.
\end{itemize}

It is important to note that the effect of a transition is to add or remove a constant number of tokens at each place and does not depend on the marking from which it is fired.
A \ac{PN} transition is said to have a \emph{constant effect}.
Thus, the effect of a transition in a \ac{PN} is a function $\effect{t}$ that maps each place $p$ to $- \inw[t][p] + \outw[t][p]$.
This definition is extended to any sequence of transitions $\seqt$ as $\effect{\seqt}[p] = \sum_{t \in \seqt} -\inw[t][p] + \outw[t][p]$.

The following well-known results will be useful in the sequel.

\begin{lemm}[Strong monotonicity of PN]
  \label{lemm:strong-monotonicity-of-pn}
  The \ac{PN} model is \emph{strongly monotonic with regard to $\preceq$}.
  That is, for all \ac{PN} $\defPN$, for all transition $t \in \transitions$ and for all markings $\mar_1, \mar_2, \mar_3$ of \namePN such that $\mar_1 \preceq \mar_2$ and $\mar_1 \fire{t} \mar_3$, there exists a unique marking $\mar_4$ of \namePN such that $\mar_2 \fire{t} \mar_4$ and $\mar_3 \preceq \mar_4$.
\end{lemm}

\begin{proof}
  $t$ is enabled in $\mar_1$.
	Thus, for all $p \in \places : \inw[t][p] \leq \mar_1(p)$.
	Since $\mar_1 \preceq \mar_2$, we have for all place $p \in \places$ : $\inw[t][p] \leq \mar_1(p) \leq \mar_2(p)$.
	Thus $t$ is enabled in $\mar_2$.
	Since the effect of transitions is constant, $\mar_1 \preceq \mar_2 \Rightarrow \mar_3 \preceq \mar_4$.
\end{proof}

\begin{lemm}[Strict strong monotonicity of PN]
  The \ac{PN} model has \emph{strict strong monotonicity with regard to $\preceq$}.
	That is, for all \ac{PN} $\defPN$, for all transition $t \in \transitions$ and for all markings $\mar_1, \mar_2, \mar_3$ of \namePN such that $\mar_1 \prec \mar_2$ and $\mar_1 \fire{t} \mar_3$, there exists a marking $\mar_4$ of \namePN such that $\mar_2 \fire{t} \mar_4$ and $\mar_3 \prec \mar_4$.
\end{lemm}

The proof is very similar to the one of the \cref{lemm:strong-monotonicity-of-pn}.

We give a general definition of the \emph{monotonicity} property in transitions systems in \cref{defi:monotonicity}.

\begin{defi}
  Given an \ac{PN} $\defNonMarkedPN$ and a marking \mar of \namePN:
  \begin{itemize}
    \item $\post{\mar} = \setComp{\mar'}{\mar \fire{} \mar'}$ is the set of one-step successors of \mar,
    \item $\pre{\mar} = \setComp{\mar'}{\mar' \fire{} \mar}$ is the set of one-step predecessors of \mar,
    \item $\posts{\mar} = \setComp{\marp}{\mar \fire{*} \marp}$ is the set of successors of \mar, in any number of step.
      With $\mari$ the initial marking of \namePN, $\posts{\mari}$ is the \emph{reachability set} of \namePN.%, denoted \reach{\nameNP}.
    \item $\pres{\mar} = \setComp{\marp}{\marp \fire{*} \mar}$ is the set of predecessors of \mar, in any number of step.
  \end{itemize}
\end{defi}

These operators are naturally extended to sets of markings as the union of the sets obtained by applying the operator on each marking of the sets.
That is, with $\markings$ a set of markings of \namePN,
$\post{\markings} = \setComp{\marp}{\exists \mar \in \markings : \mar \fire{} \marp}$.

For example, regarding the \ac{PPN} shown on figure \ref{fig:parametric-petri-net-example},
$\post{\vect{0,1,0}} = \{\vect{b, 1, 0}\}$
and
$\posts{\vect{0,1,0}} = \setComp{\vect{i, 1, 0}}{i \in \naturals} \cup \setComp{\vect{i, 0, 1}}{i \in \naturals}$.

All of this applies to \ac{PPN} through valuations of the parameters:
\begin{defi}[Instantiation of \acp{PPN}]
  Let $\defPPN$ be a \ac{PPN} and $\val : \parameters \mapsto \naturals$ be a \emph{$\naturals$-valuation}, or simply valuation, on $\parameters$.
  Then $\val[\namePPN]$ is defined as the \ac{PN} obtained by replacing each parameter $\param \in \parameters$ by $\val[\param]$.
  Thus, we have $\val[\namePPN] = \tuple{\places, \transitions, \mari'}$ such that:
  \begin{itemize}
    \item $\inm[\val(\namePPN)][p, t] =
      \begin{cases}
        \phantom{\val(}\inm[\namePPN][p, t]  & \text{if } \inm[\namePPN][p, t] \in \naturals \\
                 \val( \inm[\namePPN][p, t]) & \text{if } \inm[\namePPN][p, t] \in \parameters
      \end{cases}$
    \item $\outm[\val(\namePPN)][p, t] =
      \begin{cases}
        \phantom{\val(}\outm[\namePPN][p, t]  & \text{if } \outm[\namePPN][p, t] \in \naturals \\
                 \val( \outm[\namePPN][p, t]) & \text{if } \outm[\namePPN][p, t] \in \parameters
      \end{cases}$
    \item $\mari'(p) =
      \begin{cases}
        \phantom{\val(}\mari(p)  & \text{if } \mari(p) \in \naturals \\
                 \val( \mari(p)) & \text{if } \mari(p) \in \parameters
      \end{cases}$
  \end{itemize}
\end{defi}

Given $\namePPN$ a \ac{PPN} and a valuation $\val$, one can therefore instantiate a \ac{PN} $\val[\namePPN]$ from $\namePPN$ and apply the semantic described above.
When the \ac{PPN} under consideration is clear from the context, $\inm[\val]$ is used to denote $\inm[\val(\namePPN)]$ and $\outm[\val]$ to denote $\outm[\val(\namePPN)]$.
We write $\fire[\val]{t}$,
         $\fire[\val]{}$,
         $\fire[\val]{\sigma}$,
         $\fire[\val]{*}$,
         $\Post_v$,
         $\Pre_v$,
         $\Post^*_v$
     and $\Pre^*_v$
to denote $\fire{t}$,
          $\fire{}$,
          $\fire{\sigma}$,
          $\fire{*}$,
          $\Post$,
          $\Pre$,
          $\Post^*$
      and $\Pre^*$
on the plain \ac{PN} $\val(\namePPN)$.

This makes it possible to formally represent a system and interactions between its components.

Moreover, when working on set of valuations, it is often convenient to set an order on the parameters and to see the valuations as vectors.
Thus, for example with $\parameters = \{a, b\}$, we can consider the parameters in lexicographical order: $\parameters = (a, b)$.
Then the valuation $\val$ such that $\val(a) = 1$ and $\val(b) = 2$ may be seen as the vector $\val = (1,2)$.
Many definitions given for the markings are valid for any finite dimensional vector (that we call \emph{tuples}) whose the domain is a totally ordered countable set%
\footnote{A \emph{countable set} is a set that is finite or has the same size as $\naturals$.}.
So, by analogy, we may use $\preceq$, $\Min$, $\Max$… on valuations.

We will now define some properties that the model may have and that are usually of interest to show that the modelled system meets some requirements.
