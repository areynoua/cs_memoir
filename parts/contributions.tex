In this part we will study the possibilities to adapt the existing results for plain \ac{PN} to the \ac{PPN} model.

During this research, two results were particularly useful.
First, the adaptation of the Karp and Miller algorithm proposed by \cite{David17} is based primarily on a modification of the acceleration function. This modification can be taken up as a first point for the adaptation of other algorithms that also use an acceleration function. %In addition, this modification can also be seen as a change in the transition function. As a result, it can also be the basis for the search for adaptations in many other algorithms.

Second, it is interesting to note that some valuations have remarkable properties with respect to covering problems.
Actually, the Karp and Miller algorithm adaptations are possible thanks to these properties.

We will first cover these properties.
Then we will discuss three classic algorithms.

% vim: set spell spelllang=en :
