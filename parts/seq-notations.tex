We make a parenthesis here to introduce the notations on the sequences we will use during this work.

With $\set$ a set, a sequence of elements of $\set$ is a list of its elements in a given order.
We usually write it within parentheses.

Given two sequences $\defSeq{1}{n}$ and $\defSeq[\seq'][\sit']{1}{n'}$ of elements from $\set$, a natural $c$ and two indices $i$, $j \in \range{1}{n}, i < j$ and an object $\sit$, we use the following notations:
\begin{itemize}
  \item $\seq \concat \seq' = (\elemsSeq{1}{n}, \elemsSeq[\sit']{1}{n'})$ is the concatenation of $\seq$ and $\seq'$,
  \item $c \cdot \seq = (\overbrace{\overbracket{\elemsSeq{1}{n}}, \overbracket{\elemsSeq{1}{n}}, {\dots}}^{c \text{ times}})$ is the concatenation of $\seq$ with itself, repeated $c$ times,
  \item $\seq[i]$ is the $i$th element of $\seq$, so here $\seq[i] = \sit_i$, 
  \item $\set^*$ is the set of all the sequences of elements of $\set$, that is a sequence of zero or more elements from $\set$; by extension,
  \item $\sit^*$ denotes $\{\sit\}^*$, that is a sequence of zero or more repetitions of $\sit$,
  \item $\slice{\sit_i}{\sit_j} = \bodySeq{i}{j}$ is the subsequence of $\seq$ starting from $\sit_i$ and ending with $\sit_j$.\\
    There may be several occurrences of the given elements in $\seq$.
    We consider by convention: for the end bound, its last occurrence; and for the beginning bound, its last occurrence occurring before the end bound.
    If this is not possible, either because one of the given elements is not in the sequence, or because there is no occurrence for the start bound before the end bound, then the denoted sub-sequence is empty,
  \item $\slice{}{\sit_j} = \slice{\seq[1]}{\sit_j}$ stands for the \emph{prefix}, \lang{i.e.} beginning, of the sequence, until the last occurrence of $\sit_j$,
  \item $\slice{\sit_i}{} = \slice{\sit_i}{\seq[\card{\seq}]}$ is the end of the sequence, from the last occurrence of $\sit_i$ to the end.
\end{itemize}
