Although it was not originally designed for this purpose, the Karp and Miller algorithm \cite{Karp69} is a classical algorithm to compute a coverability set of an initialized \ac{PN}.
More precisely, \emph{it constructs a coverability tree} and uses an acceleration function to systematically detect self-covering sequences, and thus ensures the termination.

\begin{defi}[Coverability tree]
  Given a \ac{PN} $\defPN$, a coverability tree \nameT of \namePN is a labelled tree $\defT$ where:
  \begin{itemize}
    \item $\nodes$ is the set of nodes of the tree.%, is a set of \omark of \namePN such that $\downc{\nodes} = \cover{\namePN}$.

    \item $n_0 \in \nodes$ is the root of the tree, \ie $\nexists n \in \nodes$ such that $(n, n_0) \in \edges$.

    \item $\Lab : \nodes \mapsto (\naturals \cup \{\omega\})^{\card{\places}}$ is a labelling function that associate to each node an \omark of \namePN.

    \item $\edges \subseteq \nodes \times \nodes$, the set of edges, is such that:
      \begin{itemize}
        \item with $(n_1, n_2) \in \nodes^2$, if there exists an edge $(n_1, n_2) \in \edges$ then there exists a sequence $\seqt$ of transitions of $\transitions$ such that $\lab{n_1} \fire{\seqt} \lab{n_2}$,
        \item for all node $n \in \nodes \setminus \{n_0\}$, there exists a path from the root to $n$, that is, there exists a sequence of edges of $\edges$ of the form $((n_0, n_1), (n_1, n_2), \dots, (n_{i}, n))$, $i \geq 1$, and
        \item there are no cycles, that is, there are no sequences of edges of $\edges$ of the form $((n_1, n_2), (n_2, n_3), \dots, (n_i, n_1))$.
      \end{itemize}
  \end{itemize}
  and such that $\downc{\setComp{\lab{n}}{n \in \nodes}} = \cover{\namePN}$.
\end{defi}

In addition, we will often use the following notations.
\begin{itemize}
  \item We denote by $\treepath{n}$ the path in the tree from the root to $n$, that is the sequence of nodes from $n_0$ to $n$.
    $\treepath{n}$ is called \emph{the branch to $n$} and the nodes of $\treepath{n}$ are the \emph{ancestors} of $n$.
		The \emph{depth} of $n$ is its number of ancestors minus one ($|\treepath{n}| - 1$).

    With $m \in \treepath{n}$, $\treepath[m]{n}$ is the sequence of nodes from $m$ to $n$.
    If $m \notin \treepath{n}$, $\treepath[m]{n}$ is the empty sequence.
  %$\treepath{n}$ is the sequence of nodes from the path, $\treepathe{n}$ is the sequence of edges.
  \item $\forall n \in \nodes \setminus \{n_0\}$, $\parent{n}$ is the direct ancestor, or \emph{parent}, of $n$ in $\nameT$, that is, the last but one node in $\treepath{n}$.
  \item \removed{given a sequence of nodes, for any node $n$ of the sequence but the last one, $\child{n}$ is the next node in the sequence.}
\end{itemize}

The algorithm (see~\Cref{algo:km}) constructs the tree $\nameT$ as follows:
The root $n_0$ of the tree is labelled with \mari.
A frontier $\front$ is defined to be the set of unprocessed nodes of the tree and is initialised to $\{n_0\}$.
Then, while $\front$ is non-empty, a node $n$ is chosen from $\front$ to be processed:
(1) it is removed from $\front$, and (2) if there is no node $n'$ in $\treepath{n}$ such that $\lab{n} = \lab{n'}$, then for all \omark $\mar \in \post{\lab{n}}$, (2.1) a node labelled with $\kmacc{\mar, \treepath{n}}$ is added to the frontier and (2.2) to the tree as a child of $n$.

\begin{algorithm}
  \caption{The Karp and Miller algorithm}
  \label{algo:km}

  \begin{algorithmic}
    \State $n_0 \gets $ new Node, $\lab{n_0} \gets \mar_0$
    \State $\front \gets \{n_0\}$
    \While{$\front \neq \emptyset$}
      \State $n \gets$ pop$(\front)$ \Comment{(1)}
      \If{$\nexists n' \in \treepath{n} \text{ s.t. } \lab{n} = \lab{n'}$} \Comment{(2)}
        \ForAll{$\mar \in  \post{\lab{n}}$}
          \State $n'' \gets $ new Node, $\lab{n''} \gets \kmacc{\mar, \treepath{n}}$
          \State push$(\front, n'')$ \Comment{(2.1)}
          \State push$($children$(n_0), n'')$ \Comment{(2.2)}
        \EndFor
      \EndIf
    \EndWhile
  \end{algorithmic}
\end{algorithm}


We attach to the Karp and Miller tree $\nameT$ the mapping $\Labt : \nodes \setminus \{n_0\} \mapsto \transitions$ that gives for all the nodes of the tree (but the root) the transition used to create it at step (2) of the Karp and Miller algorithm.\\
By abuse of notation, we will note
$\labt{(n_1, n_2, ..., n_m)}$
%with $n_i, i \in \{1, ..., m\}$ some nodes of $\nameT$
with $n_i \in \nodes$
the sequence
% TODO: remove this false line break
\\
$(\labt{n_2}, ..., \labt{n_m})$.

To keep $\nodes$ finite, the Karp and Miller algorithm exploits the strong monotonicity of \acp{PN} to introduce \omarks through an \emph{acceleration function} $\KMAcc$.
This function takes a marking \mar to accelerate with a set of markings $\setm$ as a base for the acceleration and returns a marking such that:
\[
  \KMAcc(\mar,\setm)(p) =
  \begin{cases}
    \omega  &\text{if } \exists \mar' \in \setm : \mar' \prec \mar \text{ and } \mar'(p) < \mar(p) \\
    \mar(p) &\text{otherwise}
  \end{cases}
\]

The acceleration function is said to accelerate a marking if the first case holds for one or more places.
These places are said \emph{accelerated}.

The correctness and the termination of the algorithm lies on the strong monotonicity of \acp{PN}, and was proved by Karp and Miller in their work \cite{Karp69}.

To prove that the Karp and Miller Tree $\nameT$ is a coverability tree of \namePN, we show:
\begin{enumerate}
  \item that one can construct a sequence from \mari to any marking \mar “covered by the tree”, given a node $n$ such that $\mar \in \downc{\lab{n}}$, and
  \item that any marking of $\cover{\namePN}$ is “covered by at least one node of the tree”.
\end{enumerate}

Finally, we prove that $\nameT$ is finite and that the algorithm terminates.

Although very different in form, the proofs are based on those stated in \cite{Karp69}.
We take this opportunity to give the reader a firm understanding of how the algorithm works.
For a shorter proof, refer to \cite{Karp69}.

\begin{lemm}
  \label{theo:km-correctness}
  Given an initialized \ac{PN} $\defPN$ and a node $n$ of its Karp and Miller tree $\defT$,
  for any marking $\mar \in \downc{\lab{n}}$ there exists a sequence of transitions $\seqt \in \transitions^*$ and a marking $\mar'$ such that $\mari \fire{\seqt} \mar'$ and $\mar \preceq \mar'$.
\end{lemm}

\begin{proof}
  We show that there exists a function $\Tts : \naturals^{\card{\places}} \times \nodes \mapsto \transitions^* : \mar, n \rightarrow \seqt$ that, given a marking $\mar$ and a node $n$ such that $\mar \in \downc{\lab{n}}$, gives such a sequence $\seqt$.

  Before we look at $\Tts$, here is one key observation on the Karp and Miller tree:
  since the Karp and Miller algorithm never remove $\omega$s, we have
  $\forall n \in \nodes \setminus \{n_0\}$, $\op{\lab{\parent{n}}} \subseteq \op{\lab{n}}$,
  \ie all the \oplaces on a marking stay marked with an $\omega$ in all its descendants.

  The idea of the construction of $\tts{\mar}{n}$ is that for any \oplace $p_i$ of $\lab{n}$, one can extract from $\treepath{n}$ a self-covering sequence $\iscs{i}$ strictly increasing in $p_i$:
  $\effect{\iscs{i}}(p_i) > 0$ and $\effect{\iscs{i}}(p) \geq 0$ for all $p \in \places$.
  If $\lab{n}$ does not contain any \oplace, then the sequence given by $\labt{\treepath{n}}$ is an actual execution of the system (since no acceleration have occurred on the branch) and we define $\tts{\mar}{n}$ to be $\labt{\treepath{n}}$.
  Further down in the branch we may encounter the first node with some \oplaces.
  These $\omega$s reveal the presence of increasing self-covering sequences that we can repeat to cover the desired marking.
  If new $\omega$s appear later, the  increasing self-covering sequence is not obvious, since it may be that the effect of the sequence implicitly detected by the acceleration is negative for some places already accelerated before.
  However, this negative effect on this place may be counterbalanced by the repetition of the sequence that allowed this previous acceleration.
  We now detail the different cases.
  %We now show it in greater details.

  %Without loss of generality, we assume that $\mari$ does not map any place to $\omega$, and thus $\tts{\mar}{n_0} = ()$.
  Since $\mari$ does not have any \oplace, $\tts{\mar}{n_0} = ()$.

  %We now recursively show that $\tts{\mar}{n}$ exists for all such $(\mar, n)$ pair where $n \neq n_0$.
  We voluntary adopt a very prolific style here, in order to allow an easy and in-depth understanding of how the algorithm work.

  So assume $\mar$ and $n$ given and $\mar \in \downc{\lab{n}}$.
  When we deal with a given node, we'll need a specific order on the places, so we adopt the following conventions:
  \begin{itemize}
    \item with $p_i \in \op{\lab{n}}$,
      \begin{itemize}
        \item $n^\omega_i$ denotes the first node of $\treepath{n}$ such that $\lab{n_i}(p_i) = \omega$, and
        \item $n_i$ denotes the node used by $\KMAcc$ to accelerate $p_i$ in $\lab{n^\omega_i}$, \ie, \todo{the last} node of $\treepath{n^\omega_i}$ such that $\lab{n_i} \prec \lab{n^\omega_i} \text{ and } \lab{n_i}(p_i) \neq \omega$.
      \end{itemize}
    \item with $i$ and $j \in \{1, …, \card{\places}\}$, we have:
      \begin{itemize}
        \item $i < j$ whenever $p_i \in \op{\lab{n}} \wedge p_j \notin \op{\lab{n}}$:
          the \oplaces are before the \noplaces,
        \item $i < j$ whenever $p_i$ and $p_j \in \op{\lab{n}}$ and $\treepath{n^\omega_i}$ is a prefix of $\treepath{n^\omega_j}$:
          the places accelerated first come first,
        \item $i < j$ whenever $p_i$ and $p_j \in \op{\lab{n}}$, $n^\omega_i = n^\omega_j$ and $\treepath{n_i}$ is a prefix of $\treepath{n_j}$:
          the places accelerated thanks to nodes nearer of the root come first,
        \item otherwise, the order between $i$ and $j$ may be fixed arbitrarily.
      \end{itemize}
    \item $\ab{n} = \{1, ..., I\}$ is the possibly empty set of the indices of the places that were accelerated ``before'' $n$: $i \in \ab{n}$ iff $p_i \in \op{\lab{\parent{n}}}$,
    \item $\na{n} = \{I+1, ..., J\}$ is the possibly empty set of the indices of the places accelerated ``at'' $n$: $j \in \na{n}$ iff $p_j \in \op{\lab{n}}$ and $p_j \notin \ab{n}$.%$p_j \notin \op{\lab{\parent{n}}}$.
  \end{itemize}

  In the rest of the proof, we will consistenlty use $I$ and $J$ as the largest indices of $\ab{n}$ and $\na{n}$.

  Thus, with $n$ clear from the context this ordering fixed, $n^\omega_1$ corresponds to the first node of $\treepath{n}$ where an acceleration occured.

  \todo{example}

  Moreover, $\forall i \in \ab{n} \cup \na{n}$, we will denote by $\iscs{i}$ the self-covering sequence increasing in $p_i$ extracted from $\treepath{n}$.
  Actually, the work is twofold: when dealing with a node $n$ we have to determine $\iscs{j}$ for all $j \in \na{n}$, and how many times at least should we repeat $\iscs{i}$ for all $i \in \ab{n} \cup \na{n}$, \lang{i.e}, we are looking for a sequence of the form:
  \[ \tts{\mar}{n} = \labt{\treepath{n_1}} + k_1 \cdot \iscs{1} + \dots + k_J \cdot \iscs{J} + \labt{\treepath[n^\omega_J]{n}} \]

  Recall that
  $\forall n \in \nodes \setminus \{n_0\}$, $\op{\lab{\parent{n}}} \subseteq \op{\lab{n}}$,
  (all the \oplaces on a marking stay marked with an $\omega$ in all its descendants).
  We use this property to realise an induction on the nodes of the branch.
  However, we do not state it explicitly for the sake of clarity.

  First consider that $\ab{n} = \emptyset$.
  If $\na{n} = \emptyset$, there was no acceleration on $\treepath{n}$ and by construction of $\nameT$, $\mari \fire{\labt{\treepath{n}}} \lab{n}$.
  So we define $\tts{\mar}{n}$ to be $\labt{\treepath{n}}$ whenever $\ab{n} = \emptyset$ and $\na{n} = \emptyset$.\\
  This is a special case of the one where $\na{n}$ may be not empty.
  In that case, $\mar''$ given by $\mari \fire{\labt{\treepath{n}}} \mar''$ agrees with $\lab{n}$ in its \noplaces.
  To reach $\mar'$ that covers \mar in all its places, we explicitly repeat the increasing self-covering sequences implicitly detected by the acceleration function.
  By construction of $\nameT$ we know that $\ab{n} = \emptyset$ (thus~$I = 0$), and, for all $j \in \na{n} = \{I+1 = 1, ..., J\}$, $\lab{n_j} \prec \mar''$.
  Therefore:
  \begin{align*}
    &\makebox[0pt][l]{\,\(\labt{\treepath[n_j]{n^\omega_j}}\text{ is enabled in }\mar''\text{ (by strong monotonicity), }\)} \\
    &\text{For all the places, the effect is non-negative:}\\
    &\effect{\labt{\treepath[n_j]{n^\omega_j}}}(p) \geq 0 &&\text{ for all }j \in \na{n}\text{ and }p \in \places \text{, and} \\
    &\text{For all the accelerated places, the effect is positive:}\\
    &\effect{\labt{\treepath[n_j]{n^\omega_j}}}(p_j) > 0  &&\text{ for all }j \in \na{n} \text{.}
  \end{align*}
  (Note that $n^\omega_j = n$.)
  Thus, for all such $j$, $\iscs{j} = \labt{\treepath[n_j]{n^\omega_j}}$ is the increasing self-covering sequence we were looking for.
  Moreover, % whenever $\ab{n} = \emptyset$,
  %$\seqt_j = \labt{\treepath[n_j]{n}}$ for all $j \in \na{n}$
  %and
  there exists some naturals $k_j$ such that we can define $\tts{\mar}{n}$ to be:
  \[ \tts{\mar}{n} = \labt{\treepath{n_1}} + k_1 \cdot \iscs{1} + \dots + k_J \cdot \iscs{J} \quad\text{ if }\ab{n} = \emptyset \]
  Actually, one can easily compute such $k_j$ to obtain the shortest sequence of this form.
  With $\seqt_j = \labt{\treepath{n_1}} + k_1 \cdot \iscs{1} + … + k_j \cdot \iscs{j}$, we have:
  \[
    k_j =
    \begin{cases}
      \left\lceil
        \frac{\mar(p_j) - \mari(p_j) - \effect{\labt{\treepath{n_1}}}(p_j)}
             {\effect{\iscs{j}}}
      \right\rceil
      &\text{ if } j = 1\\
      \\
      \max\left(0,
        \left\lceil
          \frac{\mar(p_j) - \mari(p_j) - \effect{\seqt_{j-1}}(p_j)}
              {\effect{\iscs{j}}}
        \right\rceil
      \right)
      &\text{ if } j \in \{2, ..., J\}
    \end{cases}
  \]

  The chosen order ensures that with $j_1 \leq j_2$ we have $\card{\iscs{j_1}} \geq \card{\iscs{j_2}}$ and so any other order would give a sequence longer or as long as the one computed here.

  Now consider the case where $\ab{n} \neq \emptyset$.
  %By definition, a node in this case have at least one ancestor lying in the base case.
  Once again, let us focus first on the simpler case where $\na{n} = \emptyset$.
  Here one can easily compute the $\preceq$-lowest marking $\mar''$ of $\downc{\lab{\parent{n}}}$ such that $\mar'' \fire{\labt{n}} \mar'$ (with $\mar \preceq \mar'$)
  and that agrees with $\lab{\parent{n}}$ for its \noplaces. It is given by:
  \[
    \mar''(p) = \begin{cases}
      \lab{\parent{n}}(p)
        &\text{ if } p \notin \op{\lab{\parent{n}}} \\
      \max(\mar(p) - \effect{\labt{n}}(p), \inw[\labt{n}][p])
        &\text{ otherwise}
    \end{cases}
  \]

  Indeed, %(1)
  $\labt{n}$ is enabled in $\mar''$:
  for the \noplaces $p$ of $\lab{\parent{n}}$, the Karp and Miller algorithm ensures that $I_{\labt{n}}(p) \leq \mar''(p)$;
  and for the \oplaces $p$ of $\lab{\parent{n}}$, the chosen value is at least $I_{\labt{n}}(p)$.\\
  Moreover, %(2)
  $\mar \preceq \mar'$:
  for the \noplaces $p$ of $\lab{\parent{n}}$, we have $\mar'(p) = \mar''(p) + \effect{\labt{n}}(p) = \lab{n}(p) \geq \mar(p)$ by choice of $n$ and \mar;
  and for the \oplaces $p$ of $\lab{\parent{n}}$, we have $\mar'(p) = \mar''(p) + \effect{\labt{n}}(p) \geq \mar(p)$ by construction of $\mar''$.\\
  Thus, we define
  $\tts{n}{\mar}$ to be $\tts{\parent{n}}{\mar''} + (\labt{n})$
  whenever
  $\na{n} = \emptyset \wedge \ab{n} \neq \emptyset$.
  Note that $\tts{\parent{n}}{\mar''}$ may effectively be computed since we are implicitly defining $\tts$ as an induction on the nodes of the branch.

  In the general case, $\na{n}$ may be not empty.
  There a difficulty arises from the fact that $\labt{\treepath[n_j]{n^\omega_j}}, j \in \na{n}$ may not be a self-covering sequence since its effect may be negative for some places $p_i$ (with $i \in \ab{n}$).
  This difficulty is solved by integrating enough repetitions of $\iscs{i}$ along with $\labt{\treepath[n_j]{n^\omega_j}}$ in $\iscs{j}$ to counterbalance it.

  \todo{example}

  Here are two ways to show that it can be done.
  First, it may be easier to convince oneself of this by noting that, for all such $i$, there is a marking $\mar''$ such that $\tts{\mar''}{n^\omega_i}$ has enough tokens in $p_i$ to undo the negative effect of $\labt{\treepath[n_j]{n^\omega_j}}$ in this place. But the calculation of $\mar''$ is technical.\\
   However, the computation of an actual self-covering sequence increasing in $p_j$, to use instead of $\treepath[n_j]{ n^\omega_j}$, is easy.
   %With only one $i$, it is given by:
   % \todo{à corriger}
   %\[
   % \iscs{j} = \labt{\treepath[n_j]{n_i} + l_{j,i} \cdot \treepath[n_i]{n^\omega_i} + \treepath[n^\omega_i]{n^\omega_j}}
   % \text{ with } l_{j,i} =
   % \left\lceil \frac{-\effect{\labt{\treepath[n_j]{n^\omega_j}}}(p_i)}
   %                  { \effect{\labt{\treepath[n_i]{n^\omega_i}}}(p_i)} \right\rceil \text{.}
   %\]

   %In the case where there are several different $i$'s, the sequence can be obtained by considering the $i$'s one by one.
   Starting from $\treepath[n_j]{n^\omega_j}$, for each $i \in \ab{n}$, we compute a path that undo a potential negative effect in $p_i$.
   Then, the sequence we are looking for is $\iscs{j} = \labt{\seqt_{j,I}}$ given by: $\forall i \in \{0\} \cup \ab{n} = \{0, ..., I\}$
  \[
    \seqt_{j,i} =
    \begin{cases}
      \seqt_{j,0} = \treepath[n_j]{n^\omega_j}
        &\text{ if } i = 0 \\
      \seqt_{j,i} = \seqt_{j,i-1}
        &\text{ if } i > 0 \text{ and } \effect{\labt{\seqt_{j,i-1}}}(p_i) \geq 0 \\
      \seqt_{j,i} = \slice[\seqt_{j,i-1}]{}{u} + l_{j,i} \cdot \iscs{i} + \slice[\seqt_{j,i-1}]{v}{}
        &\text{ otherwise}
    \end{cases}
  \]
  where $u = \elem{\iscs{i}}{1}$ is the first element of $\iscs{i}$ and $v = \elem{\iscs{i}}{\card{\iscs{i}}}$ is its last element,
  and
  \[
    l_{j,i} = 
    \left\lceil \frac{-\effect{\labt{\seqt_{j,i-1}}}(p_i)}
                     { \effect{\labt{\iscs{i}}}(p_i)} \right\rceil \text{.}
  \]
  \todo{example}

  This provides us with a mean to compute an actual increasing self-covering sequence for every accelerated place $p_j, j \in \na{n}$.

  Thus, as before we define $\Tts$ to be
  \[ \tts{\mar}{n} = \labt{\treepath{n_1}} + k_1 \cdot \iscs{1} + \dots + k_J \cdot \iscs{J} \]
  %with $m = n_i : \min_{i \in \range{1}{J}}(\card{\treepath{n_i}})$ is the first node of the branch used in a increasing self-covering sequence $\iscs{}$.
  Here however, since $j_1 < j_2$ does not implies $\card{\iscs{j_1}} \geq \card{\iscs{j_2}}$ anymore, another order may lead to shorter sequences.



  To sum up, given a \ac{PN}, its Karp and Miller tree, a node $n$ from it and a marking $\mar \in \downc{n}$, we have
  \[
    \mari \fire{\tts{\mar}{n}} \mar', \mar \preceq \mar'
  \]
  with
  \[
    \tts{\mar}{n} = \labt{\treepath{n_1}} + k_1 \cdot \iscs{1} + \dots + k_J \cdot \iscs{J} + \labt{\treepath[n^\omega_J]{n}}
  \]
  where each increasing self-covering sequences $\iscs{j}$ is computed in the context of the first node of the branch where it is accelerated: $n^\omega_j$;
  where the $k_j$ are computed in the context of the last node where an acceleration occurred $n^\omega_J$, considering a marking $\mar''$ such that
  \( \mar'' \fire{\labt{\treepath[n^\omega_J]{n}}} \mar' \).
\end{proof}

\begin{lemm}
  Given an initialized \ac{PN} $\defPN$ and any of its coverable marking $\mar \in \cover{\namePN}$,
  there exists a node $n$ of its Karp and Miller tree $\defT$, such that $\mar \preceq \lab{n}$.
\end{lemm}

\begin{proof}
  Let \mar be a marking of $\cover{\namePN}$ and $\defSeqt{0}{n-1}$ a sequence of transitions that witnesses it: \(\mari \fire{\seqt} \mar', \mar \preceq \mar'\).
  Let $\bodySeqm{1}{n}$ be the markings : \(\mari \fire{t_0} \mar_1 \fire{t_1} \dots \fire{t_{n-1}} \mar_n = \mar'\).

  Then, apply the following operations on this sequence of markings:
  while there exists a pair of markings \((\mar_i, \mar_j), i < j, \mar_i \preceq \mar_j\) in the sequence:
  \begin{itemize}
    \item if $\mar_i = \mar_j$, remove all the nodes after $\mar_j$,
    \item otherwise, for each place $p$ such that $\mar_i(p) < \mar_j(p)$, replace $\mar_k(p)$ by $\omega$ for all the markings $\mar_k, k \geq j$ from $\mar_j$ (included) to the end of the sequence.
  \end{itemize}

  At the end, the sequence obtained is the sequence of labels in some path $\treepath{n}$ from the $n_0$ to $n$ and the final label cover $\mar$.
\end{proof}

\begin{lemm}
  \label{theo:km-tree-finiteness}
  For any \ac{PN}, its Karp and Miller tree is finite.
\end{lemm}

Based on the Kőnig's Infinity lemma, we will prove that, to be infinite, a tree needs either an infinite branch or a node with infinitely many children.
From there we will proof the finiteness of the Karp and Miller tree.

\begin{lemm}[Kőnig's Infinity lemma \cite{konig1927schlussweise} applied to trees]
	\label{theo:konig}
	Let $\nameT$ be a tree of infinite depth where the number of nodes at each depth is finite.
	Then $\nameT$ contains an infinite branch.
\end{lemm}

\begin{lemm}
  \label{theo:false-konig}
  Every infinite tree has a node with infinitely many children or contains an infinite branch.
\end{lemm}

\begin{proof}%[\proofname\ of \Cref{theo:false-konig}]
  Let $\nameT$ be an infinite tree where every node has a finite number of children.
  For every $c \in naturals$, let $\set_c$ be the set of nodes at depth $c$.
	Induction on $c$ shows that the sets $\set_c$ are finite.
	Thus, by \Cref{theo:konig}, $\nameT$ contains an infinite branch.
\end{proof}

\begin{proof}[\proofname\ of \Cref{theo:km-tree-finiteness}]
  Suppose that the Karp and Miller tree is infinite
  and recall that, by definition, the set of transitions of a PN is finite.
	Then, by contruction, every node of the Karp and Miller tree has a finite number of children.
  Thus, the tree has an inifinite branch.
  This contradicts \Cref{lemm:wqo}.
\end{proof}

\todo{example and explicits why it contradicts lemma 1}
 
The Karp and Miller tree has a lot of convenient properties that allow, among other, to answer the coverability problem as well as the simultaneous place unboundedness problem.

\begin{defi}[Simultaneous place unboundedness]
  Let $\defPN$ be a marked \ac{PN}.
  Given a set of place $\setp \subseteq \places$, \namePN is said $\setp$-simultaneous unbounded if and only if for any $c \in \naturals$ there exists \mar such that $\mari \fire* \mar$ and $\forall p \in \setp : \mar(p) \geq c$.
\end{defi}

Furthermore, the Karp and Miller algorithm can easily be adapted to some parametric problems \cite{David17}, as we will show in \autoref{sec:known-results-on-ppn}.

However, this tree, although finite, is often much larger than the minimal coverability set, and cannot be constructed in reasonable time.
As a consequence, many improvements were proposed, as well as other algorithms with different approaches.
