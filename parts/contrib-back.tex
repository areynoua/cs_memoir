The backward algorithm does not seem to be easy to adapt to the parametric coverability problems.
Indeed, it works only thanks to the iteration of the $\Pre$ operator.

In the case of PostT-\ac{PPN} and of the \Ecov problem, the idea would be to look for \mari from the markings to cover.
We have seen that, with a forward approach, one can consider only high valuations, that are better because they allow to cover more markings: the search from $\mari$ may go ``too far'' without being an issue.
Here however a high valuation may lead to a premature stop in a backward search, because it restricts ``more quickly'' the set of transitions that is enabled at each step.
In other words, contrary to forward approach, the backward approach has the deals with the 0 bound of the transition precondition. 

However, it may be the case that one can answer the \Ucov problem by applying the backward algorithm on the plain \ac{PN} obtained by the 0-valuation of all its parameter.
This will be a direction of our study.

The situation is symmetrical for PreT-\ac{PPN}, where the algorithm does not seem to be of much help for the \Ucov problem but where considering a 0-valuation may lead to a result about the \Ecov problem.

% vim: set spell spelllang=en :
