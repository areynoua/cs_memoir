The backward algorithm does not seem to be easy to adapt to the parametric coverability problems.
Indeed, it works only thanks to the iteration of the $\Pre$ operator.

In the case of PostT-\ac{PPN} and of the \Ecov problem, the idea would be to look for \mari from the markings to cover.
We have seen that, with a forward approach, one can consider only high valuations, that are better because they allow to cover more markings: the search from $\mari$ may go ``too far'' without being an issue.
Here however a high valuation may lead to a premature stop in a backward search, because it restricts ``more quickly'' the set of transitions that is enabled at each step.
In other words, contrary to forward approach, the backward approach has the deals with the 0 bound of the transition precondition. 

However, it may be the case that one can answer the \Ucov problem by applying the backward algorithm on the plain \ac{PN} obtained by the 0-valuation of all its parameter.
This will be a direction of our study.

The situation is symmetrical for PreT-\ac{PPN}, where the algorithm does not seem to be of much help for the \Ucov problem but where considering a 0-valuation may lead to a result about the \Ecov problem.

\rule{5cm}{1pt}

We will now give an adaptation of the general backward algorithm to solve the \Ucov problem on PostT-\ac{PPN}.

This proves the decidability of this problem for this subclass of \ac{PPN}\todo{ Check for no others proves and make this more apparent}.

Remember that $\back$ denotes the fixed-point algorithm described in \autoref{sec:backward-algorithm}.
With $v_0 : \mathbb{P} \mapsto \mathbb{N} : a \rightarrow 0 \forall a \in \mathbb{P}$ the valuation that maps all the parameters to 0, the new algorithm may be described as follows:
given an initialized \ac{PPN} \SPTPm and an upward-closed set of markings $U$, compute $\back(v_0(\PPN), U)$.

The termination of the algorithm follows from the termination of $\back$.
Intuitively, the correctness comes from the fact that $v_0$ is the worst valuation for the coverability problem.

\begin{proof}
  To prove the correctness of this algorithm is to prove that:
  \[
    \back(v_0(\PPN), U) = \top \Leftrightarrow \UcovOp(\PPN, U) = \top
  \]

$\UcovOp(\PPN, U) \Rightarrow \back(v_0(\PPN), U)$ is trivial. So we prove the other direction. Let us first rewrite the implication by applying the definition of $\back$: $\mari \in \Pre_{v_0}^*(U) \Rightarrow \UcovOp(\PPN, U)$.

%We may apply the defintion of $\back$ to the left-hand side, and the definition of $\UcovOp$ with the properties of $\back$ to the right-hand side to obtain:
%\[
%  \mari \in \Pre_{v_0}^*(U) \Leftrightarrow \mari \in \Pre_v^*(U) \forall v \in \{\mathbb{P} \mapsto \mathbb{N}\}
%\]

%\paragraph*{We prove that $\mari \in \Pre_{v_0}^*(U) \Rightarrow \UcovOp(\PPN, U)$:}
  We know that there exists a marking of $U$ reachable from \mari in $v_0(\PPN)$.
  Let $\marp$ be such a marking and $\sigma$ be a sequence of transitions $\sigma = (t_1, \dots, t_{n-1})$ such that $\mari \fire{\sigma}_{v_0} \marp$.
  Now observe that for all valuation $v\in \{\mathbb{P} \mapsto \mathbb{N}\}$, for all place $p \in P$, and for all transition $t \in T$: $\matI_{v_0(\PPN)}(p,t) = \matI_{v(\PPN)}(p,t)$ and $\matO_{v_0(\PPN)}(p,t) \leq \matO_{v(\PPN)}(p,t)$, that is all transitions consume the same amount of token than with the $v_0$ valuation and produce at least the same amount. This directly implies that: \todo{introduces Effect\_v}
  \[
    \forall v \in \{\mathbb{P} \mapsto \mathbb{N}\}, \forall p \in P, \forall t \in T :
  \quad
  \Effect_{v_0}(t)(p) \leq \Effect_v(t)(p)
  \]

By recurrence on the steps, this ensures that for any valuation $v$:
  \begin{enumerate}
    \item $\mari \fire{t_1}_v \mar_1 \fire{t_2}_v \dots \fire{t_{n-1}}_v \mar_n$: the sequence $\sigma$ is enabled \todo{introduce enabled for a seq of transitions } in $\mari$,
    \item $\marp \preceq \mar_n$: and the reached marking $\mar_n$ is greater or equal to $\marp$ according to $\preceq$.
  \end{enumerate}

  Because $U$ is upward-closed and $\marp \in U$, we know that $\mar_n \in U$ as well and so $U$ is coverable for any valuation.
\end{proof}

\todo{illustration}

This adaptation of $\back$ may also be used to solve the \Ecov problem in the PreT-\ac{PPN} model.
The idea is that, in this model, $v_0$ is the valuation that enables the most transitions and where the transitions produce the most tokens.
Thus, we hope that if there exist a valuation that allows to cover $U$, $v_0$ will also allows to cover $U$.

Formally we have to prove that, for any given \ac{PPN} \SPTPm and upward-closed set of markings $U$:
  \[
    \back(v_0(\PPN), U) = \top \Leftrightarrow \EcovOp(\PPN, U) = \top
  \]

\begin{proof}
  $\back(v_0(\PPN), U) \Rightarrow \EcovOp(\PPN, U)$ is trivial.

  By definition of \Ecov and $\back$, to prove that $\EcovOp(\PPN, U) \Rightarrow \back(v_0(\PPN), U)$ is to prove that
  \[
    \exists v \mid \mari \in \Pre^*_{v}(U) \Rightarrow \mari \in \Pre^*_{v_0}(U)
  \]
  \todo{Maybe write with parentheses?:}$ \left( \exists v \mid \mari \in \Pre^*_{v}(U) \right) \Rightarrow \mari \in \Pre^*_{v_0}(U) $

  Suppose that such a $v$ exists. We have a marking $\marp \in U$ and a sequence of transitions $\sigma = (t_1, \dots, t_{n-1})$ such that $\mari \fire{\sigma}_v \marp$.

  Once again let us compare the effect of the transitions in the \acp{PN} $v(\PPN)$ and $v_0(\PPN)$. Here, for any place $p$ and transition $t$: $\matI_{v(\PPN)}(p,t) \geq \matI_{v_0(\PPN)}(p,t)$ and $\matO_{v(\PPN)}(p,t) = \matO_{v_0(\PPN)}(p,t)$, that is all transitions produce the same amount of token than with the $v_0$ valuation but consume the same amount or more. So, if a marking enables a transition in $v(\PPN)$, it enables it in $v_0(\PPN)$ too. Moreover, $\Effect_v(t)(p) \leq \Effect_{v_0}(t)(p)$, thus, by recurrence on the transitions of $\sigma$, we have that:
  \begin{enumerate}
    \item $\mari \fire{t_1}_{v_0} \mar_1 \fire{t_2}_{v_0} \dots \fire{t_{n-1}}_{v_0} \mar_n$: the sequence $\sigma$ is enabled in $v_0$,
    \item $\marp \preceq \mar_n$: and the reached marking $\mar_n$ is greater or equal to $\marp$ according to $\preceq$.
  \end{enumerate}

  Because $U$ is upward-closed and $\marp \in U$, we know that $\mar_n \in U$ as well and so $U$ is coverable for any valuation.
\end{proof}

% vim: set spell spelllang=en :
